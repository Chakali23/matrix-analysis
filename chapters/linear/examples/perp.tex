\begin{enumerate}[label=\thesection.\arabic*,ref=\thesection.\theenumi]
\numberwithin{equation}{enumi}
\numberwithin{figure}{enumi}
\numberwithin{table}{enumi}

\item 
\item 
\item  Reduce the following equations into normal form. Find their perpendicular distances from the origin and angle between perpendicular and the positive $x$-axis.
\label{chapters/11/10/3/3}
\begin{enumerate}
	\item $x-\sqrt{3}y+8=0$ 
	\item $y-2=0$
	\item $x-y=4$
\end{enumerate}
\solution
\begin{enumerate}[label=\thesection.\arabic*,ref=\thesection.\theenumi]
\numberwithin{equation}{enumi}
\numberwithin{figure}{enumi}
\numberwithin{table}{enumi}

\item 
\item 
\item  Reduce the following equations into normal form. Find their perpendicular distances from the origin and angle between perpendicular and the positive $x$-axis.
\label{chapters/11/10/3/3}
\begin{enumerate}
	\item $x-\sqrt{3}y+8=0$ 
	\item $y-2=0$
	\item $x-y=4$
\end{enumerate}
\solution
\begin{enumerate}[label=\thesection.\arabic*,ref=\thesection.\theenumi]
\numberwithin{equation}{enumi}
\numberwithin{figure}{enumi}
\numberwithin{table}{enumi}

\item 
\item 
\item  Reduce the following equations into normal form. Find their perpendicular distances from the origin and angle between perpendicular and the positive $x$-axis.
\label{chapters/11/10/3/3}
\begin{enumerate}
	\item $x-\sqrt{3}y+8=0$ 
	\item $y-2=0$
	\item $x-y=4$
\end{enumerate}
\solution
\begin{enumerate}[label=\thesection.\arabic*,ref=\thesection.\theenumi]
\numberwithin{equation}{enumi}
\numberwithin{figure}{enumi}
\numberwithin{table}{enumi}

\item 
\item 
\item  Reduce the following equations into normal form. Find their perpendicular distances from the origin and angle between perpendicular and the positive $x$-axis.
\label{chapters/11/10/3/3}
\begin{enumerate}
	\item $x-\sqrt{3}y+8=0$ 
	\item $y-2=0$
	\item $x-y=4$
\end{enumerate}
\solution
\input{chapters/11/10/3/3/perp.tex}
\item Find the distance of the point $(-1,1)$ from the line $12\brak{x+6} = 5\brak{y-2}$. 
\label{chapters/11/10/3/4}
\input{chapters/11/10/3/4/line.tex}
\item Find the points on the x-axis, whose distances from the line $\frac{x}{3}+\frac{y}{4}=1$ are 4 units.
\label{chapters/11/10/3/5}
	\\
	\solution
\input{chapters/11/10/3/5/line.tex}
\item Find the distance between parallel lines
\label{chapters/11/10/3/6}
\begin{enumerate}
	\item $15x+8y-34=0$ and  $15x+8y+31=0$ \\
	\item  $l(x+y)+p=0$ and  $l(x+y)-r=0$
\end{enumerate}
	\solution
\input{chapters/11/10/3/6/dist.tex}
\item Find the coordinates of the foot of the perpendicular from $(-1, 3)$ to the line $3x-4y-16=0$.  
\label{chapters/11/10/3/14}
\\
\solution
\input{chapters/11/10/3/14/line.tex}
\item  If ${p}$ and ${q}$ are the lengths of perpendiculars from the origin to the lines ${x}\cos\theta - {y}\sin\theta =  {k}\cos2\theta$ and ${x}\sec\theta + {y}\cosec\theta = {k}$, respectively, prove that ${p}^2 + 4{q}^2 = {k}^2$
\label{chapters/11/10/3/16}
\\
\solution
\input{chapters/11/10/3/16/line.tex}
\item In the triangle $ABC$ with vertices $\vec{A} \brak{2, 3}$, $\vec{B} \brak{4, –1}$ and $\vec{C} \brak{1, 2}$, find the equation and length of altitude from the vertex $\vec{A}$.
\label{chapters/11/10/3/17}
\\
\solution
\input{chapters/11/10/3/17/line.tex}
\item If $p$ is the length of perpendicular from origin to the line whose intercepts on the axes are $a$ and $b$, then show that 
\begin{align}
	\frac{1}{p^2} = \frac{1}{a^2}+ \frac{1}{b^2}
\end{align}
\label{chapters/11/10/3/18}
\input{chapters/11/10/3/18/dist.tex}
\item What are the points on the y-axis whose distance from the line $\frac{x}{3}+\frac{y}{4}=1$ is 4 units.
\\
\solution
		\input{chapters/11/10/4/4/line.tex}
\item Find perpendicular distance from the origin to the line joining the points$(\cos\theta,\sin\theta)$ and $(\cos\phi,\sin\phi)$.
\\
\solution
		\input{chapters/11/10/4/5/dist.tex}
\item Find the equation of line which is equidistant from parallel lines $9x+6y-7=0$ and $3x+2y+6=0$.
\\
\solution
		\input{chapters/11/10/4/21/line.tex}
	\item Prove that the products of the lengths of the perpendiculars drawn from the points $\myvec{\sqrt{a^2-b^2}\\0}$ and $\myvec{-\sqrt{a^2-b^2} \\0} $ to the line $\frac{x}{a} \cos{\theta} + \frac{y}{b}\sin{\theta} =1 $ is $ b^2 $.
\\
    \solution 
		\input{chapters/11/10/4/23/dist.tex}
\item Find the equation of line  drawn perpendicular to the line $\frac{x}{4}+\frac{y}{6}=1$ through the point where it meets the y-axis \\
\solution
		\input{chapters/11/10/4/7/line.tex}
 \item  In each of the following cases, determine the direction cosines of the normal to
the plane and the distance from the origin.
\begin{enumerate}
	\item $z=2$ 
	\item $x + y + z = 1$
	\item $2x + 3y – z = 5$
	\item $5y + 8 = 0$
\end{enumerate}
    \solution
		\input{chapters/12/11/3/1/dist.tex}
\item
\input{chapters/12/11/4/3/vec4.tex}
\end{enumerate}

\item Find the distance of the point $(-1,1)$ from the line $12\brak{x+6} = 5\brak{y-2}$. 
\label{chapters/11/10/3/4}
%\documentclass[12pt]{article}
%\usepackage[cmex10]{amsmath}
%\usepackage{amsthm}
%\usepackage{mathrsfs}
%\usepackage{txfonts}
%\usepackage{stfloats}
%\usepackage{bm}
%\usepackage{cite}
%\usepackage{cases}
%\usepackage{subfig}
%\usepackage{longtable}
%\usepackage{multirow}
%\usepackage{enumitem}
%\usepackage{mathtools}
%\usepackage{steinmetz}
%\usepackage{tikz}
%\usepackage{circuitikz}
%\usepackage{verbatim}
%\usepackage{tfrupee}
%\usepackage[breaklinks=true]{hyperref}
%\usepackage{tkz-euclide} % loads  TikZ and tkz-base
%\providecommand{\brak}[1]{\ensuremath{\left(#1\right)}}
%\usepackage{atbegshi}
%\AtBeginDocument{\AtBeginShipoutNext{\AtBeginShipoutDiscard}}
%\usetikzlibrary{calc,math}
%\usepackage{listings}
%    \usepackage{color}                                            %%
%    \usepackage{array}                                            %%
 %   \usepackage{longtable}                                        %%
  %  \usepackage{calc}                                             %%
   % \usepackage{multirow}                                         %%
    %\usepackage{hhline}                                           %%
    %\usepackage{ifthen}                                           %%
  %optionally (for landscape tables embedded in another document): %%
    %\usepackage{lscape}     
%\usepackage{multicol}
%\usepackage{chngcntr}

%\DeclareMathOperator*{\Res}{Res}
%\renewcommand{\baselinestretch}{2}
%\renewcommand\thesection{\arabic{section}}
%\renewcommand\thesubsection{\thesection.\arabic{subsection}}
%\renewcommand\thesubsubsection{\thesubsection.\arabic{subsubsection}}


% correct bad hyphenation here
%\hyphenation{op-tical net-works semi-conduc-tor}
%\def\inputGnumericTable{}                                 %%

%\lstset{
%language=C,
%frame=single, 
%breaklines=true,
%columns=fullflexible
%}
%\begin{document}
%\newtheorem{theorem}{Theorem}[section]
%\newtheorem{problem}{Problem}
%\newtheorem{proposition}{Proposition}[section]
%\newtheorem{lemma}{Lemma}[section]
%\newtheorem{corollary}[theorem]{Corollary}
%\newtheorem{example}{Example}[section]
%\newtheorem{definition}[problem]{Definition}
%\newcommand{\BEQA}{\begin{eqnarray}}
%\newcommand{\EEQA}{\end{eqnarray}}
%\newcommand{\define}{\stackrel{\triangle}{=}}

%\bibliographystyle{IEEEtran}
%\bibliographystyle{ieeetr}
%\providecommand{\mbf}{\mathbf}
%\providecommand{\pr}[1]{\ensuremath{\Pr\left(#1\right)}}
%\providecommand{\qfunc}[1]{\ensuremath{Q\left(#1\right)}}
%\providecommand{\sbrak}[1]{\ensuremath{{}\left[#1\right]}}
%\providecommand{\lsbrak}[1]{\ensuremath{{}\left[#1\right.}}
%\providecommand{\rsbrak}[1]{\ensuremath{{}\left.#1\right]}}
%\providecommand{\brak}[1]{\ensuremath{\left(#1\right)}}
%\providecommand{\lbrak}[1]{\ensuremath{\left(#1\right.}}
%\providecommand{\rbrak}[1]{\ensuremath{\left.#1\right)}}
%\providecommand{\cbrak}[1]{\ensuremath{\left\{#1\right\}}}
%\providecommand{\lcbrak}[1]{\ensuremath{\left\{#1\right.}}
%\providecommand{\rcbrak}[1]{\ensuremath{\left.#1\right\}}}
%\theoremstyle{remark}
%\newtheorem{rem}{Remark}
%\newcommand{\sgn}{\mathop{\mathrm{sgn}}}
%\providecommand{\res}[1]{\Res\displaylimits_{#1}} 
%\providecommand{\mtx}[1]{\mathbf{#1}}
%\providecommand{\fourier}{\overset{\mathcal{F}}{\rightleftharpoons}}
%\providecommand{\system}{\overset{\mathcal{H}}{\longleftrightarrow}}
	%\newcommand{\solution}[2]{\textbf{Solution:}{#1}}
%\newcommand{\solution}{\noindent \textbf{Solution: }}
%\newcommand{\cosec}{\,\text{cosec}\,}
%\providecommand{\dec}[2]{\ensuremath{\overset{#1}{\underset{#2}{\gtrless}}}}
%\newcommand{\myvec}[1]{\ensuremath{\begin{pmatrix}#1\end{pmatrix}}}
%\newcommand{\mydet}[1]{\ensuremath{\begin{vmatrix}#1\end{vmatrix}}}
%\let\vec\mathbf
%\begin{center}
%\title{\textbf{Straight Lines}}
%\date{\vspace{-5ex}} %Not to print date automatically
%\maketitle
%\end{center}
%\setcounter{page}{1}
%\section*{11$^{th}$ Maths - Chapter 10}
%This is Problem-10 from Exercise 10.4
%\begin{enumerate}
%    \item If three lines whose equations are $y=m_1x+c_1$, $y=m_2x+c_2$ and $y=m_3x+c_3$ are concurrent, then show that $m_1(c_2-c_3)+m_2(c_3-c_1)+m_3(c_1-c_2) = 0.$\\
%    \solution 
    Given lines can be written as \begin{align}
       m_1x-y+c_1=0
    \end{align}
    \begin{align}
        m_2x-y+c_2=0
    \end{align}
    \begin{align}
        m_3x-y+c_3=0
        \label{eq:3}
    \end{align}
    
    
   The above lines can be written in the form of \begin{align}
        \Vec{n}^{\top}\Vec{x} = c
    \end{align}
   Therefore,
		\begin{align}
       \myvec{m_1&-1}\vec{x}=c_1
       \label{eq:5}
   \end{align} 
   \begin{align}
       \myvec{m_2&-1}\vec{x}=c_2
       \label{eq:6}
   \end{align}
   \begin{align}
       \myvec{m_3&-1}\vec{x}=c_3
       \label{eq:7}
   \end{align}
   Solving equations \eqref{eq:5}, \eqref{eq:6}and \eqref{eq:7}
		augumented matrix is
 \begin{align}
    \myvec{m_1&-1&c_1\\m_2&-1&c_2\\m_3&-1&c_3}\\
    \xleftrightarrow{R_2 \leftarrow m_1R_2-m_2R_1}
    \myvec{m_1&-1&c_1\\0&m_2-m_1&m_1c_2-m_2c_1\\m_3&-1&c_3}\\
    \xleftrightarrow{R_3 \leftarrow m_1R_3-m_3R_1}
    \myvec{m_1&-1&c_1\\0&m_2-m_1&m_1c_2-m_2c_1\\0&m_3-m_1&m_1c_3-m_3c_1}\\
    \xleftrightarrow{R_3 \leftarrow R_3\frac{m_2-m_1}{m_3-m_1}-R_2}
        \myvec{m_1&-1&c_1\\0&m_2-m_1&m_1c_2-m_2c_1\\0&0&$\brak{m_1c_3-m_3c_1}$$\brak{\frac{m_2-m_1}{m_3-m_1}}$-$\brak{m_1c_2-m_2c_1}$}
\end{align}
Now, for lines to be concurrent, then the third row should be equal to zero. \\

Therefore,
\begin{align}
\brak{m_1c_3-m_3c_1}\brak{\frac{m_2-m_1}{m_3-m_1}}-\brak{m_1c_2-m_2c_1}=0\\
\frac{\brak{m_1c_3-m_3c_1}\brak{m_2-m_1}-\brak{m_1c_2-m_2c_1}\brak{m_3-m_1}}{m_3-m_1}=0\\
\brak{m_1c_3-m_3c_1}\brak{m_2-m_1}-\brak{m_1c_2-m_2c_1}\brak{m_3-m_1}=0\\
m_2c_3-m_1c_3+m_3c_1-m_3c_2+m_1c_2-m_2c_1=0\\
m_1\brak{c_2-c_3}+m_2\brak{c_3-c_1}+m_3\brak{c_1-c_2} = 0
\end{align}
           Hence proved
%\begin{figure}[h]
 %   \centering
  %  \includegraphics[width=\columnwidth]{concurrent-1.png}
   % \caption{Straight Lines}
    %\label{fig:concurrent-1.png}
%\end{figure}
%\end{enumerate}
%\end{document}

\item Find the points on the x-axis, whose distances from the line $\frac{x}{3}+\frac{y}{4}=1$ are 4 units.
\label{chapters/11/10/3/5}
	\\
	\solution
%\documentclass[12pt]{article}
%\usepackage[cmex10]{amsmath}
%\usepackage{amsthm}
%\usepackage{mathrsfs}
%\usepackage{txfonts}
%\usepackage{stfloats}
%\usepackage{bm}
%\usepackage{cite}
%\usepackage{cases}
%\usepackage{subfig}
%\usepackage{longtable}
%\usepackage{multirow}
%\usepackage{enumitem}
%\usepackage{mathtools}
%\usepackage{steinmetz}
%\usepackage{tikz}
%\usepackage{circuitikz}
%\usepackage{verbatim}
%\usepackage{tfrupee}
%\usepackage[breaklinks=true]{hyperref}
%\usepackage{tkz-euclide} % loads  TikZ and tkz-base
%\providecommand{\brak}[1]{\ensuremath{\left(#1\right)}}
%\usepackage{atbegshi}
%\AtBeginDocument{\AtBeginShipoutNext{\AtBeginShipoutDiscard}}
%\usetikzlibrary{calc,math}
%\usepackage{listings}
%    \usepackage{color}                                            %%
%    \usepackage{array}                                            %%
 %   \usepackage{longtable}                                        %%
  %  \usepackage{calc}                                             %%
   % \usepackage{multirow}                                         %%
    %\usepackage{hhline}                                           %%
    %\usepackage{ifthen}                                           %%
  %optionally (for landscape tables embedded in another document): %%
    %\usepackage{lscape}     
%\usepackage{multicol}
%\usepackage{chngcntr}

%\DeclareMathOperator*{\Res}{Res}
%\renewcommand{\baselinestretch}{2}
%\renewcommand\thesection{\arabic{section}}
%\renewcommand\thesubsection{\thesection.\arabic{subsection}}
%\renewcommand\thesubsubsection{\thesubsection.\arabic{subsubsection}}


% correct bad hyphenation here
%\hyphenation{op-tical net-works semi-conduc-tor}
%\def\inputGnumericTable{}                                 %%

%\lstset{
%language=C,
%frame=single, 
%breaklines=true,
%columns=fullflexible
%}
%\begin{document}
%\newtheorem{theorem}{Theorem}[section]
%\newtheorem{problem}{Problem}
%\newtheorem{proposition}{Proposition}[section]
%\newtheorem{lemma}{Lemma}[section]
%\newtheorem{corollary}[theorem]{Corollary}
%\newtheorem{example}{Example}[section]
%\newtheorem{definition}[problem]{Definition}
%\newcommand{\BEQA}{\begin{eqnarray}}
%\newcommand{\EEQA}{\end{eqnarray}}
%\newcommand{\define}{\stackrel{\triangle}{=}}

%\bibliographystyle{IEEEtran}
%\bibliographystyle{ieeetr}
%\providecommand{\mbf}{\mathbf}
%\providecommand{\pr}[1]{\ensuremath{\Pr\left(#1\right)}}
%\providecommand{\qfunc}[1]{\ensuremath{Q\left(#1\right)}}
%\providecommand{\sbrak}[1]{\ensuremath{{}\left[#1\right]}}
%\providecommand{\lsbrak}[1]{\ensuremath{{}\left[#1\right.}}
%\providecommand{\rsbrak}[1]{\ensuremath{{}\left.#1\right]}}
%\providecommand{\brak}[1]{\ensuremath{\left(#1\right)}}
%\providecommand{\lbrak}[1]{\ensuremath{\left(#1\right.}}
%\providecommand{\rbrak}[1]{\ensuremath{\left.#1\right)}}
%\providecommand{\cbrak}[1]{\ensuremath{\left\{#1\right\}}}
%\providecommand{\lcbrak}[1]{\ensuremath{\left\{#1\right.}}
%\providecommand{\rcbrak}[1]{\ensuremath{\left.#1\right\}}}
%\theoremstyle{remark}
%\newtheorem{rem}{Remark}
%\newcommand{\sgn}{\mathop{\mathrm{sgn}}}
%\providecommand{\res}[1]{\Res\displaylimits_{#1}} 
%\providecommand{\mtx}[1]{\mathbf{#1}}
%\providecommand{\fourier}{\overset{\mathcal{F}}{\rightleftharpoons}}
%\providecommand{\system}{\overset{\mathcal{H}}{\longleftrightarrow}}
	%\newcommand{\solution}[2]{\textbf{Solution:}{#1}}
%\newcommand{\solution}{\noindent \textbf{Solution: }}
%\newcommand{\cosec}{\,\text{cosec}\,}
%\providecommand{\dec}[2]{\ensuremath{\overset{#1}{\underset{#2}{\gtrless}}}}
%\newcommand{\myvec}[1]{\ensuremath{\begin{pmatrix}#1\end{pmatrix}}}
%\newcommand{\mydet}[1]{\ensuremath{\begin{vmatrix}#1\end{vmatrix}}}
%\let\vec\mathbf
%\begin{center}
%\title{\textbf{Straight Lines}}
%\date{\vspace{-5ex}} %Not to print date automatically
%\maketitle
%\end{center}
%\setcounter{page}{1}
%\section*{11$^{th}$ Maths - Chapter 10}
%This is Problem-10 from Exercise 10.4
%\begin{enumerate}
%    \item If three lines whose equations are $y=m_1x+c_1$, $y=m_2x+c_2$ and $y=m_3x+c_3$ are concurrent, then show that $m_1(c_2-c_3)+m_2(c_3-c_1)+m_3(c_1-c_2) = 0.$\\
%    \solution 
    Given lines can be written as \begin{align}
       m_1x-y+c_1=0
    \end{align}
    \begin{align}
        m_2x-y+c_2=0
    \end{align}
    \begin{align}
        m_3x-y+c_3=0
        \label{eq:3}
    \end{align}
    
    
   The above lines can be written in the form of \begin{align}
        \Vec{n}^{\top}\Vec{x} = c
    \end{align}
   Therefore,
		\begin{align}
       \myvec{m_1&-1}\vec{x}=c_1
       \label{eq:5}
   \end{align} 
   \begin{align}
       \myvec{m_2&-1}\vec{x}=c_2
       \label{eq:6}
   \end{align}
   \begin{align}
       \myvec{m_3&-1}\vec{x}=c_3
       \label{eq:7}
   \end{align}
   Solving equations \eqref{eq:5}, \eqref{eq:6}and \eqref{eq:7}
		augumented matrix is
 \begin{align}
    \myvec{m_1&-1&c_1\\m_2&-1&c_2\\m_3&-1&c_3}\\
    \xleftrightarrow{R_2 \leftarrow m_1R_2-m_2R_1}
    \myvec{m_1&-1&c_1\\0&m_2-m_1&m_1c_2-m_2c_1\\m_3&-1&c_3}\\
    \xleftrightarrow{R_3 \leftarrow m_1R_3-m_3R_1}
    \myvec{m_1&-1&c_1\\0&m_2-m_1&m_1c_2-m_2c_1\\0&m_3-m_1&m_1c_3-m_3c_1}\\
    \xleftrightarrow{R_3 \leftarrow R_3\frac{m_2-m_1}{m_3-m_1}-R_2}
        \myvec{m_1&-1&c_1\\0&m_2-m_1&m_1c_2-m_2c_1\\0&0&$\brak{m_1c_3-m_3c_1}$$\brak{\frac{m_2-m_1}{m_3-m_1}}$-$\brak{m_1c_2-m_2c_1}$}
\end{align}
Now, for lines to be concurrent, then the third row should be equal to zero. \\

Therefore,
\begin{align}
\brak{m_1c_3-m_3c_1}\brak{\frac{m_2-m_1}{m_3-m_1}}-\brak{m_1c_2-m_2c_1}=0\\
\frac{\brak{m_1c_3-m_3c_1}\brak{m_2-m_1}-\brak{m_1c_2-m_2c_1}\brak{m_3-m_1}}{m_3-m_1}=0\\
\brak{m_1c_3-m_3c_1}\brak{m_2-m_1}-\brak{m_1c_2-m_2c_1}\brak{m_3-m_1}=0\\
m_2c_3-m_1c_3+m_3c_1-m_3c_2+m_1c_2-m_2c_1=0\\
m_1\brak{c_2-c_3}+m_2\brak{c_3-c_1}+m_3\brak{c_1-c_2} = 0
\end{align}
           Hence proved
%\begin{figure}[h]
 %   \centering
  %  \includegraphics[width=\columnwidth]{concurrent-1.png}
   % \caption{Straight Lines}
    %\label{fig:concurrent-1.png}
%\end{figure}
%\end{enumerate}
%\end{document}

\item Find the distance between parallel lines
\label{chapters/11/10/3/6}
\begin{enumerate}
	\item $15x+8y-34=0$ and  $15x+8y+31=0$ \\
	\item  $l(x+y)+p=0$ and  $l(x+y)-r=0$
\end{enumerate}
	\solution
\input{chapters/11/10/3/6/dist.tex}
\item Find the coordinates of the foot of the perpendicular from $(-1, 3)$ to the line $3x-4y-16=0$.  
\label{chapters/11/10/3/14}
\\
\solution
%\documentclass[12pt]{article}
%\usepackage[cmex10]{amsmath}
%\usepackage{amsthm}
%\usepackage{mathrsfs}
%\usepackage{txfonts}
%\usepackage{stfloats}
%\usepackage{bm}
%\usepackage{cite}
%\usepackage{cases}
%\usepackage{subfig}
%\usepackage{longtable}
%\usepackage{multirow}
%\usepackage{enumitem}
%\usepackage{mathtools}
%\usepackage{steinmetz}
%\usepackage{tikz}
%\usepackage{circuitikz}
%\usepackage{verbatim}
%\usepackage{tfrupee}
%\usepackage[breaklinks=true]{hyperref}
%\usepackage{tkz-euclide} % loads  TikZ and tkz-base
%\providecommand{\brak}[1]{\ensuremath{\left(#1\right)}}
%\usepackage{atbegshi}
%\AtBeginDocument{\AtBeginShipoutNext{\AtBeginShipoutDiscard}}
%\usetikzlibrary{calc,math}
%\usepackage{listings}
%    \usepackage{color}                                            %%
%    \usepackage{array}                                            %%
 %   \usepackage{longtable}                                        %%
  %  \usepackage{calc}                                             %%
   % \usepackage{multirow}                                         %%
    %\usepackage{hhline}                                           %%
    %\usepackage{ifthen}                                           %%
  %optionally (for landscape tables embedded in another document): %%
    %\usepackage{lscape}     
%\usepackage{multicol}
%\usepackage{chngcntr}

%\DeclareMathOperator*{\Res}{Res}
%\renewcommand{\baselinestretch}{2}
%\renewcommand\thesection{\arabic{section}}
%\renewcommand\thesubsection{\thesection.\arabic{subsection}}
%\renewcommand\thesubsubsection{\thesubsection.\arabic{subsubsection}}


% correct bad hyphenation here
%\hyphenation{op-tical net-works semi-conduc-tor}
%\def\inputGnumericTable{}                                 %%

%\lstset{
%language=C,
%frame=single, 
%breaklines=true,
%columns=fullflexible
%}
%\begin{document}
%\newtheorem{theorem}{Theorem}[section]
%\newtheorem{problem}{Problem}
%\newtheorem{proposition}{Proposition}[section]
%\newtheorem{lemma}{Lemma}[section]
%\newtheorem{corollary}[theorem]{Corollary}
%\newtheorem{example}{Example}[section]
%\newtheorem{definition}[problem]{Definition}
%\newcommand{\BEQA}{\begin{eqnarray}}
%\newcommand{\EEQA}{\end{eqnarray}}
%\newcommand{\define}{\stackrel{\triangle}{=}}

%\bibliographystyle{IEEEtran}
%\bibliographystyle{ieeetr}
%\providecommand{\mbf}{\mathbf}
%\providecommand{\pr}[1]{\ensuremath{\Pr\left(#1\right)}}
%\providecommand{\qfunc}[1]{\ensuremath{Q\left(#1\right)}}
%\providecommand{\sbrak}[1]{\ensuremath{{}\left[#1\right]}}
%\providecommand{\lsbrak}[1]{\ensuremath{{}\left[#1\right.}}
%\providecommand{\rsbrak}[1]{\ensuremath{{}\left.#1\right]}}
%\providecommand{\brak}[1]{\ensuremath{\left(#1\right)}}
%\providecommand{\lbrak}[1]{\ensuremath{\left(#1\right.}}
%\providecommand{\rbrak}[1]{\ensuremath{\left.#1\right)}}
%\providecommand{\cbrak}[1]{\ensuremath{\left\{#1\right\}}}
%\providecommand{\lcbrak}[1]{\ensuremath{\left\{#1\right.}}
%\providecommand{\rcbrak}[1]{\ensuremath{\left.#1\right\}}}
%\theoremstyle{remark}
%\newtheorem{rem}{Remark}
%\newcommand{\sgn}{\mathop{\mathrm{sgn}}}
%\providecommand{\res}[1]{\Res\displaylimits_{#1}} 
%\providecommand{\mtx}[1]{\mathbf{#1}}
%\providecommand{\fourier}{\overset{\mathcal{F}}{\rightleftharpoons}}
%\providecommand{\system}{\overset{\mathcal{H}}{\longleftrightarrow}}
	%\newcommand{\solution}[2]{\textbf{Solution:}{#1}}
%\newcommand{\solution}{\noindent \textbf{Solution: }}
%\newcommand{\cosec}{\,\text{cosec}\,}
%\providecommand{\dec}[2]{\ensuremath{\overset{#1}{\underset{#2}{\gtrless}}}}
%\newcommand{\myvec}[1]{\ensuremath{\begin{pmatrix}#1\end{pmatrix}}}
%\newcommand{\mydet}[1]{\ensuremath{\begin{vmatrix}#1\end{vmatrix}}}
%\let\vec\mathbf
%\begin{center}
%\title{\textbf{Straight Lines}}
%\date{\vspace{-5ex}} %Not to print date automatically
%\maketitle
%\end{center}
%\setcounter{page}{1}
%\section*{11$^{th}$ Maths - Chapter 10}
%This is Problem-10 from Exercise 10.4
%\begin{enumerate}
%    \item If three lines whose equations are $y=m_1x+c_1$, $y=m_2x+c_2$ and $y=m_3x+c_3$ are concurrent, then show that $m_1(c_2-c_3)+m_2(c_3-c_1)+m_3(c_1-c_2) = 0.$\\
%    \solution 
    Given lines can be written as \begin{align}
       m_1x-y+c_1=0
    \end{align}
    \begin{align}
        m_2x-y+c_2=0
    \end{align}
    \begin{align}
        m_3x-y+c_3=0
        \label{eq:3}
    \end{align}
    
    
   The above lines can be written in the form of \begin{align}
        \Vec{n}^{\top}\Vec{x} = c
    \end{align}
   Therefore,
		\begin{align}
       \myvec{m_1&-1}\vec{x}=c_1
       \label{eq:5}
   \end{align} 
   \begin{align}
       \myvec{m_2&-1}\vec{x}=c_2
       \label{eq:6}
   \end{align}
   \begin{align}
       \myvec{m_3&-1}\vec{x}=c_3
       \label{eq:7}
   \end{align}
   Solving equations \eqref{eq:5}, \eqref{eq:6}and \eqref{eq:7}
		augumented matrix is
 \begin{align}
    \myvec{m_1&-1&c_1\\m_2&-1&c_2\\m_3&-1&c_3}\\
    \xleftrightarrow{R_2 \leftarrow m_1R_2-m_2R_1}
    \myvec{m_1&-1&c_1\\0&m_2-m_1&m_1c_2-m_2c_1\\m_3&-1&c_3}\\
    \xleftrightarrow{R_3 \leftarrow m_1R_3-m_3R_1}
    \myvec{m_1&-1&c_1\\0&m_2-m_1&m_1c_2-m_2c_1\\0&m_3-m_1&m_1c_3-m_3c_1}\\
    \xleftrightarrow{R_3 \leftarrow R_3\frac{m_2-m_1}{m_3-m_1}-R_2}
        \myvec{m_1&-1&c_1\\0&m_2-m_1&m_1c_2-m_2c_1\\0&0&$\brak{m_1c_3-m_3c_1}$$\brak{\frac{m_2-m_1}{m_3-m_1}}$-$\brak{m_1c_2-m_2c_1}$}
\end{align}
Now, for lines to be concurrent, then the third row should be equal to zero. \\

Therefore,
\begin{align}
\brak{m_1c_3-m_3c_1}\brak{\frac{m_2-m_1}{m_3-m_1}}-\brak{m_1c_2-m_2c_1}=0\\
\frac{\brak{m_1c_3-m_3c_1}\brak{m_2-m_1}-\brak{m_1c_2-m_2c_1}\brak{m_3-m_1}}{m_3-m_1}=0\\
\brak{m_1c_3-m_3c_1}\brak{m_2-m_1}-\brak{m_1c_2-m_2c_1}\brak{m_3-m_1}=0\\
m_2c_3-m_1c_3+m_3c_1-m_3c_2+m_1c_2-m_2c_1=0\\
m_1\brak{c_2-c_3}+m_2\brak{c_3-c_1}+m_3\brak{c_1-c_2} = 0
\end{align}
           Hence proved
%\begin{figure}[h]
 %   \centering
  %  \includegraphics[width=\columnwidth]{concurrent-1.png}
   % \caption{Straight Lines}
    %\label{fig:concurrent-1.png}
%\end{figure}
%\end{enumerate}
%\end{document}

\item  If ${p}$ and ${q}$ are the lengths of perpendiculars from the origin to the lines ${x}\cos\theta - {y}\sin\theta =  {k}\cos2\theta$ and ${x}\sec\theta + {y}\cosec\theta = {k}$, respectively, prove that ${p}^2 + 4{q}^2 = {k}^2$
\label{chapters/11/10/3/16}
\\
\solution
%\documentclass[12pt]{article}
%\usepackage[cmex10]{amsmath}
%\usepackage{amsthm}
%\usepackage{mathrsfs}
%\usepackage{txfonts}
%\usepackage{stfloats}
%\usepackage{bm}
%\usepackage{cite}
%\usepackage{cases}
%\usepackage{subfig}
%\usepackage{longtable}
%\usepackage{multirow}
%\usepackage{enumitem}
%\usepackage{mathtools}
%\usepackage{steinmetz}
%\usepackage{tikz}
%\usepackage{circuitikz}
%\usepackage{verbatim}
%\usepackage{tfrupee}
%\usepackage[breaklinks=true]{hyperref}
%\usepackage{tkz-euclide} % loads  TikZ and tkz-base
%\providecommand{\brak}[1]{\ensuremath{\left(#1\right)}}
%\usepackage{atbegshi}
%\AtBeginDocument{\AtBeginShipoutNext{\AtBeginShipoutDiscard}}
%\usetikzlibrary{calc,math}
%\usepackage{listings}
%    \usepackage{color}                                            %%
%    \usepackage{array}                                            %%
 %   \usepackage{longtable}                                        %%
  %  \usepackage{calc}                                             %%
   % \usepackage{multirow}                                         %%
    %\usepackage{hhline}                                           %%
    %\usepackage{ifthen}                                           %%
  %optionally (for landscape tables embedded in another document): %%
    %\usepackage{lscape}     
%\usepackage{multicol}
%\usepackage{chngcntr}

%\DeclareMathOperator*{\Res}{Res}
%\renewcommand{\baselinestretch}{2}
%\renewcommand\thesection{\arabic{section}}
%\renewcommand\thesubsection{\thesection.\arabic{subsection}}
%\renewcommand\thesubsubsection{\thesubsection.\arabic{subsubsection}}


% correct bad hyphenation here
%\hyphenation{op-tical net-works semi-conduc-tor}
%\def\inputGnumericTable{}                                 %%

%\lstset{
%language=C,
%frame=single, 
%breaklines=true,
%columns=fullflexible
%}
%\begin{document}
%\newtheorem{theorem}{Theorem}[section]
%\newtheorem{problem}{Problem}
%\newtheorem{proposition}{Proposition}[section]
%\newtheorem{lemma}{Lemma}[section]
%\newtheorem{corollary}[theorem]{Corollary}
%\newtheorem{example}{Example}[section]
%\newtheorem{definition}[problem]{Definition}
%\newcommand{\BEQA}{\begin{eqnarray}}
%\newcommand{\EEQA}{\end{eqnarray}}
%\newcommand{\define}{\stackrel{\triangle}{=}}

%\bibliographystyle{IEEEtran}
%\bibliographystyle{ieeetr}
%\providecommand{\mbf}{\mathbf}
%\providecommand{\pr}[1]{\ensuremath{\Pr\left(#1\right)}}
%\providecommand{\qfunc}[1]{\ensuremath{Q\left(#1\right)}}
%\providecommand{\sbrak}[1]{\ensuremath{{}\left[#1\right]}}
%\providecommand{\lsbrak}[1]{\ensuremath{{}\left[#1\right.}}
%\providecommand{\rsbrak}[1]{\ensuremath{{}\left.#1\right]}}
%\providecommand{\brak}[1]{\ensuremath{\left(#1\right)}}
%\providecommand{\lbrak}[1]{\ensuremath{\left(#1\right.}}
%\providecommand{\rbrak}[1]{\ensuremath{\left.#1\right)}}
%\providecommand{\cbrak}[1]{\ensuremath{\left\{#1\right\}}}
%\providecommand{\lcbrak}[1]{\ensuremath{\left\{#1\right.}}
%\providecommand{\rcbrak}[1]{\ensuremath{\left.#1\right\}}}
%\theoremstyle{remark}
%\newtheorem{rem}{Remark}
%\newcommand{\sgn}{\mathop{\mathrm{sgn}}}
%\providecommand{\res}[1]{\Res\displaylimits_{#1}} 
%\providecommand{\mtx}[1]{\mathbf{#1}}
%\providecommand{\fourier}{\overset{\mathcal{F}}{\rightleftharpoons}}
%\providecommand{\system}{\overset{\mathcal{H}}{\longleftrightarrow}}
	%\newcommand{\solution}[2]{\textbf{Solution:}{#1}}
%\newcommand{\solution}{\noindent \textbf{Solution: }}
%\newcommand{\cosec}{\,\text{cosec}\,}
%\providecommand{\dec}[2]{\ensuremath{\overset{#1}{\underset{#2}{\gtrless}}}}
%\newcommand{\myvec}[1]{\ensuremath{\begin{pmatrix}#1\end{pmatrix}}}
%\newcommand{\mydet}[1]{\ensuremath{\begin{vmatrix}#1\end{vmatrix}}}
%\let\vec\mathbf
%\begin{center}
%\title{\textbf{Straight Lines}}
%\date{\vspace{-5ex}} %Not to print date automatically
%\maketitle
%\end{center}
%\setcounter{page}{1}
%\section*{11$^{th}$ Maths - Chapter 10}
%This is Problem-10 from Exercise 10.4
%\begin{enumerate}
%    \item If three lines whose equations are $y=m_1x+c_1$, $y=m_2x+c_2$ and $y=m_3x+c_3$ are concurrent, then show that $m_1(c_2-c_3)+m_2(c_3-c_1)+m_3(c_1-c_2) = 0.$\\
%    \solution 
    Given lines can be written as \begin{align}
       m_1x-y+c_1=0
    \end{align}
    \begin{align}
        m_2x-y+c_2=0
    \end{align}
    \begin{align}
        m_3x-y+c_3=0
        \label{eq:3}
    \end{align}
    
    
   The above lines can be written in the form of \begin{align}
        \Vec{n}^{\top}\Vec{x} = c
    \end{align}
   Therefore,
		\begin{align}
       \myvec{m_1&-1}\vec{x}=c_1
       \label{eq:5}
   \end{align} 
   \begin{align}
       \myvec{m_2&-1}\vec{x}=c_2
       \label{eq:6}
   \end{align}
   \begin{align}
       \myvec{m_3&-1}\vec{x}=c_3
       \label{eq:7}
   \end{align}
   Solving equations \eqref{eq:5}, \eqref{eq:6}and \eqref{eq:7}
		augumented matrix is
 \begin{align}
    \myvec{m_1&-1&c_1\\m_2&-1&c_2\\m_3&-1&c_3}\\
    \xleftrightarrow{R_2 \leftarrow m_1R_2-m_2R_1}
    \myvec{m_1&-1&c_1\\0&m_2-m_1&m_1c_2-m_2c_1\\m_3&-1&c_3}\\
    \xleftrightarrow{R_3 \leftarrow m_1R_3-m_3R_1}
    \myvec{m_1&-1&c_1\\0&m_2-m_1&m_1c_2-m_2c_1\\0&m_3-m_1&m_1c_3-m_3c_1}\\
    \xleftrightarrow{R_3 \leftarrow R_3\frac{m_2-m_1}{m_3-m_1}-R_2}
        \myvec{m_1&-1&c_1\\0&m_2-m_1&m_1c_2-m_2c_1\\0&0&$\brak{m_1c_3-m_3c_1}$$\brak{\frac{m_2-m_1}{m_3-m_1}}$-$\brak{m_1c_2-m_2c_1}$}
\end{align}
Now, for lines to be concurrent, then the third row should be equal to zero. \\

Therefore,
\begin{align}
\brak{m_1c_3-m_3c_1}\brak{\frac{m_2-m_1}{m_3-m_1}}-\brak{m_1c_2-m_2c_1}=0\\
\frac{\brak{m_1c_3-m_3c_1}\brak{m_2-m_1}-\brak{m_1c_2-m_2c_1}\brak{m_3-m_1}}{m_3-m_1}=0\\
\brak{m_1c_3-m_3c_1}\brak{m_2-m_1}-\brak{m_1c_2-m_2c_1}\brak{m_3-m_1}=0\\
m_2c_3-m_1c_3+m_3c_1-m_3c_2+m_1c_2-m_2c_1=0\\
m_1\brak{c_2-c_3}+m_2\brak{c_3-c_1}+m_3\brak{c_1-c_2} = 0
\end{align}
           Hence proved
%\begin{figure}[h]
 %   \centering
  %  \includegraphics[width=\columnwidth]{concurrent-1.png}
   % \caption{Straight Lines}
    %\label{fig:concurrent-1.png}
%\end{figure}
%\end{enumerate}
%\end{document}

\item In the triangle $ABC$ with vertices $\vec{A} \brak{2, 3}$, $\vec{B} \brak{4, –1}$ and $\vec{C} \brak{1, 2}$, find the equation and length of altitude from the vertex $\vec{A}$.
\label{chapters/11/10/3/17}
\\
\solution
%\documentclass[12pt]{article}
%\usepackage[cmex10]{amsmath}
%\usepackage{amsthm}
%\usepackage{mathrsfs}
%\usepackage{txfonts}
%\usepackage{stfloats}
%\usepackage{bm}
%\usepackage{cite}
%\usepackage{cases}
%\usepackage{subfig}
%\usepackage{longtable}
%\usepackage{multirow}
%\usepackage{enumitem}
%\usepackage{mathtools}
%\usepackage{steinmetz}
%\usepackage{tikz}
%\usepackage{circuitikz}
%\usepackage{verbatim}
%\usepackage{tfrupee}
%\usepackage[breaklinks=true]{hyperref}
%\usepackage{tkz-euclide} % loads  TikZ and tkz-base
%\providecommand{\brak}[1]{\ensuremath{\left(#1\right)}}
%\usepackage{atbegshi}
%\AtBeginDocument{\AtBeginShipoutNext{\AtBeginShipoutDiscard}}
%\usetikzlibrary{calc,math}
%\usepackage{listings}
%    \usepackage{color}                                            %%
%    \usepackage{array}                                            %%
 %   \usepackage{longtable}                                        %%
  %  \usepackage{calc}                                             %%
   % \usepackage{multirow}                                         %%
    %\usepackage{hhline}                                           %%
    %\usepackage{ifthen}                                           %%
  %optionally (for landscape tables embedded in another document): %%
    %\usepackage{lscape}     
%\usepackage{multicol}
%\usepackage{chngcntr}

%\DeclareMathOperator*{\Res}{Res}
%\renewcommand{\baselinestretch}{2}
%\renewcommand\thesection{\arabic{section}}
%\renewcommand\thesubsection{\thesection.\arabic{subsection}}
%\renewcommand\thesubsubsection{\thesubsection.\arabic{subsubsection}}


% correct bad hyphenation here
%\hyphenation{op-tical net-works semi-conduc-tor}
%\def\inputGnumericTable{}                                 %%

%\lstset{
%language=C,
%frame=single, 
%breaklines=true,
%columns=fullflexible
%}
%\begin{document}
%\newtheorem{theorem}{Theorem}[section]
%\newtheorem{problem}{Problem}
%\newtheorem{proposition}{Proposition}[section]
%\newtheorem{lemma}{Lemma}[section]
%\newtheorem{corollary}[theorem]{Corollary}
%\newtheorem{example}{Example}[section]
%\newtheorem{definition}[problem]{Definition}
%\newcommand{\BEQA}{\begin{eqnarray}}
%\newcommand{\EEQA}{\end{eqnarray}}
%\newcommand{\define}{\stackrel{\triangle}{=}}

%\bibliographystyle{IEEEtran}
%\bibliographystyle{ieeetr}
%\providecommand{\mbf}{\mathbf}
%\providecommand{\pr}[1]{\ensuremath{\Pr\left(#1\right)}}
%\providecommand{\qfunc}[1]{\ensuremath{Q\left(#1\right)}}
%\providecommand{\sbrak}[1]{\ensuremath{{}\left[#1\right]}}
%\providecommand{\lsbrak}[1]{\ensuremath{{}\left[#1\right.}}
%\providecommand{\rsbrak}[1]{\ensuremath{{}\left.#1\right]}}
%\providecommand{\brak}[1]{\ensuremath{\left(#1\right)}}
%\providecommand{\lbrak}[1]{\ensuremath{\left(#1\right.}}
%\providecommand{\rbrak}[1]{\ensuremath{\left.#1\right)}}
%\providecommand{\cbrak}[1]{\ensuremath{\left\{#1\right\}}}
%\providecommand{\lcbrak}[1]{\ensuremath{\left\{#1\right.}}
%\providecommand{\rcbrak}[1]{\ensuremath{\left.#1\right\}}}
%\theoremstyle{remark}
%\newtheorem{rem}{Remark}
%\newcommand{\sgn}{\mathop{\mathrm{sgn}}}
%\providecommand{\res}[1]{\Res\displaylimits_{#1}} 
%\providecommand{\mtx}[1]{\mathbf{#1}}
%\providecommand{\fourier}{\overset{\mathcal{F}}{\rightleftharpoons}}
%\providecommand{\system}{\overset{\mathcal{H}}{\longleftrightarrow}}
	%\newcommand{\solution}[2]{\textbf{Solution:}{#1}}
%\newcommand{\solution}{\noindent \textbf{Solution: }}
%\newcommand{\cosec}{\,\text{cosec}\,}
%\providecommand{\dec}[2]{\ensuremath{\overset{#1}{\underset{#2}{\gtrless}}}}
%\newcommand{\myvec}[1]{\ensuremath{\begin{pmatrix}#1\end{pmatrix}}}
%\newcommand{\mydet}[1]{\ensuremath{\begin{vmatrix}#1\end{vmatrix}}}
%\let\vec\mathbf
%\begin{center}
%\title{\textbf{Straight Lines}}
%\date{\vspace{-5ex}} %Not to print date automatically
%\maketitle
%\end{center}
%\setcounter{page}{1}
%\section*{11$^{th}$ Maths - Chapter 10}
%This is Problem-10 from Exercise 10.4
%\begin{enumerate}
%    \item If three lines whose equations are $y=m_1x+c_1$, $y=m_2x+c_2$ and $y=m_3x+c_3$ are concurrent, then show that $m_1(c_2-c_3)+m_2(c_3-c_1)+m_3(c_1-c_2) = 0.$\\
%    \solution 
    Given lines can be written as \begin{align}
       m_1x-y+c_1=0
    \end{align}
    \begin{align}
        m_2x-y+c_2=0
    \end{align}
    \begin{align}
        m_3x-y+c_3=0
        \label{eq:3}
    \end{align}
    
    
   The above lines can be written in the form of \begin{align}
        \Vec{n}^{\top}\Vec{x} = c
    \end{align}
   Therefore,
		\begin{align}
       \myvec{m_1&-1}\vec{x}=c_1
       \label{eq:5}
   \end{align} 
   \begin{align}
       \myvec{m_2&-1}\vec{x}=c_2
       \label{eq:6}
   \end{align}
   \begin{align}
       \myvec{m_3&-1}\vec{x}=c_3
       \label{eq:7}
   \end{align}
   Solving equations \eqref{eq:5}, \eqref{eq:6}and \eqref{eq:7}
		augumented matrix is
 \begin{align}
    \myvec{m_1&-1&c_1\\m_2&-1&c_2\\m_3&-1&c_3}\\
    \xleftrightarrow{R_2 \leftarrow m_1R_2-m_2R_1}
    \myvec{m_1&-1&c_1\\0&m_2-m_1&m_1c_2-m_2c_1\\m_3&-1&c_3}\\
    \xleftrightarrow{R_3 \leftarrow m_1R_3-m_3R_1}
    \myvec{m_1&-1&c_1\\0&m_2-m_1&m_1c_2-m_2c_1\\0&m_3-m_1&m_1c_3-m_3c_1}\\
    \xleftrightarrow{R_3 \leftarrow R_3\frac{m_2-m_1}{m_3-m_1}-R_2}
        \myvec{m_1&-1&c_1\\0&m_2-m_1&m_1c_2-m_2c_1\\0&0&$\brak{m_1c_3-m_3c_1}$$\brak{\frac{m_2-m_1}{m_3-m_1}}$-$\brak{m_1c_2-m_2c_1}$}
\end{align}
Now, for lines to be concurrent, then the third row should be equal to zero. \\

Therefore,
\begin{align}
\brak{m_1c_3-m_3c_1}\brak{\frac{m_2-m_1}{m_3-m_1}}-\brak{m_1c_2-m_2c_1}=0\\
\frac{\brak{m_1c_3-m_3c_1}\brak{m_2-m_1}-\brak{m_1c_2-m_2c_1}\brak{m_3-m_1}}{m_3-m_1}=0\\
\brak{m_1c_3-m_3c_1}\brak{m_2-m_1}-\brak{m_1c_2-m_2c_1}\brak{m_3-m_1}=0\\
m_2c_3-m_1c_3+m_3c_1-m_3c_2+m_1c_2-m_2c_1=0\\
m_1\brak{c_2-c_3}+m_2\brak{c_3-c_1}+m_3\brak{c_1-c_2} = 0
\end{align}
           Hence proved
%\begin{figure}[h]
 %   \centering
  %  \includegraphics[width=\columnwidth]{concurrent-1.png}
   % \caption{Straight Lines}
    %\label{fig:concurrent-1.png}
%\end{figure}
%\end{enumerate}
%\end{document}

\item If $p$ is the length of perpendicular from origin to the line whose intercepts on the axes are $a$ and $b$, then show that 
\begin{align}
	\frac{1}{p^2} = \frac{1}{a^2}+ \frac{1}{b^2}
\end{align}
\label{chapters/11/10/3/18}
\input{chapters/11/10/3/18/dist.tex}
\item What are the points on the y-axis whose distance from the line $\frac{x}{3}+\frac{y}{4}=1$ is 4 units.
\\
\solution
		%\documentclass[12pt]{article}
%\usepackage[cmex10]{amsmath}
%\usepackage{amsthm}
%\usepackage{mathrsfs}
%\usepackage{txfonts}
%\usepackage{stfloats}
%\usepackage{bm}
%\usepackage{cite}
%\usepackage{cases}
%\usepackage{subfig}
%\usepackage{longtable}
%\usepackage{multirow}
%\usepackage{enumitem}
%\usepackage{mathtools}
%\usepackage{steinmetz}
%\usepackage{tikz}
%\usepackage{circuitikz}
%\usepackage{verbatim}
%\usepackage{tfrupee}
%\usepackage[breaklinks=true]{hyperref}
%\usepackage{tkz-euclide} % loads  TikZ and tkz-base
%\providecommand{\brak}[1]{\ensuremath{\left(#1\right)}}
%\usepackage{atbegshi}
%\AtBeginDocument{\AtBeginShipoutNext{\AtBeginShipoutDiscard}}
%\usetikzlibrary{calc,math}
%\usepackage{listings}
%    \usepackage{color}                                            %%
%    \usepackage{array}                                            %%
 %   \usepackage{longtable}                                        %%
  %  \usepackage{calc}                                             %%
   % \usepackage{multirow}                                         %%
    %\usepackage{hhline}                                           %%
    %\usepackage{ifthen}                                           %%
  %optionally (for landscape tables embedded in another document): %%
    %\usepackage{lscape}     
%\usepackage{multicol}
%\usepackage{chngcntr}

%\DeclareMathOperator*{\Res}{Res}
%\renewcommand{\baselinestretch}{2}
%\renewcommand\thesection{\arabic{section}}
%\renewcommand\thesubsection{\thesection.\arabic{subsection}}
%\renewcommand\thesubsubsection{\thesubsection.\arabic{subsubsection}}


% correct bad hyphenation here
%\hyphenation{op-tical net-works semi-conduc-tor}
%\def\inputGnumericTable{}                                 %%

%\lstset{
%language=C,
%frame=single, 
%breaklines=true,
%columns=fullflexible
%}
%\begin{document}
%\newtheorem{theorem}{Theorem}[section]
%\newtheorem{problem}{Problem}
%\newtheorem{proposition}{Proposition}[section]
%\newtheorem{lemma}{Lemma}[section]
%\newtheorem{corollary}[theorem]{Corollary}
%\newtheorem{example}{Example}[section]
%\newtheorem{definition}[problem]{Definition}
%\newcommand{\BEQA}{\begin{eqnarray}}
%\newcommand{\EEQA}{\end{eqnarray}}
%\newcommand{\define}{\stackrel{\triangle}{=}}

%\bibliographystyle{IEEEtran}
%\bibliographystyle{ieeetr}
%\providecommand{\mbf}{\mathbf}
%\providecommand{\pr}[1]{\ensuremath{\Pr\left(#1\right)}}
%\providecommand{\qfunc}[1]{\ensuremath{Q\left(#1\right)}}
%\providecommand{\sbrak}[1]{\ensuremath{{}\left[#1\right]}}
%\providecommand{\lsbrak}[1]{\ensuremath{{}\left[#1\right.}}
%\providecommand{\rsbrak}[1]{\ensuremath{{}\left.#1\right]}}
%\providecommand{\brak}[1]{\ensuremath{\left(#1\right)}}
%\providecommand{\lbrak}[1]{\ensuremath{\left(#1\right.}}
%\providecommand{\rbrak}[1]{\ensuremath{\left.#1\right)}}
%\providecommand{\cbrak}[1]{\ensuremath{\left\{#1\right\}}}
%\providecommand{\lcbrak}[1]{\ensuremath{\left\{#1\right.}}
%\providecommand{\rcbrak}[1]{\ensuremath{\left.#1\right\}}}
%\theoremstyle{remark}
%\newtheorem{rem}{Remark}
%\newcommand{\sgn}{\mathop{\mathrm{sgn}}}
%\providecommand{\res}[1]{\Res\displaylimits_{#1}} 
%\providecommand{\mtx}[1]{\mathbf{#1}}
%\providecommand{\fourier}{\overset{\mathcal{F}}{\rightleftharpoons}}
%\providecommand{\system}{\overset{\mathcal{H}}{\longleftrightarrow}}
	%\newcommand{\solution}[2]{\textbf{Solution:}{#1}}
%\newcommand{\solution}{\noindent \textbf{Solution: }}
%\newcommand{\cosec}{\,\text{cosec}\,}
%\providecommand{\dec}[2]{\ensuremath{\overset{#1}{\underset{#2}{\gtrless}}}}
%\newcommand{\myvec}[1]{\ensuremath{\begin{pmatrix}#1\end{pmatrix}}}
%\newcommand{\mydet}[1]{\ensuremath{\begin{vmatrix}#1\end{vmatrix}}}
%\let\vec\mathbf
%\begin{center}
%\title{\textbf{Straight Lines}}
%\date{\vspace{-5ex}} %Not to print date automatically
%\maketitle
%\end{center}
%\setcounter{page}{1}
%\section*{11$^{th}$ Maths - Chapter 10}
%This is Problem-10 from Exercise 10.4
%\begin{enumerate}
%    \item If three lines whose equations are $y=m_1x+c_1$, $y=m_2x+c_2$ and $y=m_3x+c_3$ are concurrent, then show that $m_1(c_2-c_3)+m_2(c_3-c_1)+m_3(c_1-c_2) = 0.$\\
%    \solution 
    Given lines can be written as \begin{align}
       m_1x-y+c_1=0
    \end{align}
    \begin{align}
        m_2x-y+c_2=0
    \end{align}
    \begin{align}
        m_3x-y+c_3=0
        \label{eq:3}
    \end{align}
    
    
   The above lines can be written in the form of \begin{align}
        \Vec{n}^{\top}\Vec{x} = c
    \end{align}
   Therefore,
		\begin{align}
       \myvec{m_1&-1}\vec{x}=c_1
       \label{eq:5}
   \end{align} 
   \begin{align}
       \myvec{m_2&-1}\vec{x}=c_2
       \label{eq:6}
   \end{align}
   \begin{align}
       \myvec{m_3&-1}\vec{x}=c_3
       \label{eq:7}
   \end{align}
   Solving equations \eqref{eq:5}, \eqref{eq:6}and \eqref{eq:7}
		augumented matrix is
 \begin{align}
    \myvec{m_1&-1&c_1\\m_2&-1&c_2\\m_3&-1&c_3}\\
    \xleftrightarrow{R_2 \leftarrow m_1R_2-m_2R_1}
    \myvec{m_1&-1&c_1\\0&m_2-m_1&m_1c_2-m_2c_1\\m_3&-1&c_3}\\
    \xleftrightarrow{R_3 \leftarrow m_1R_3-m_3R_1}
    \myvec{m_1&-1&c_1\\0&m_2-m_1&m_1c_2-m_2c_1\\0&m_3-m_1&m_1c_3-m_3c_1}\\
    \xleftrightarrow{R_3 \leftarrow R_3\frac{m_2-m_1}{m_3-m_1}-R_2}
        \myvec{m_1&-1&c_1\\0&m_2-m_1&m_1c_2-m_2c_1\\0&0&$\brak{m_1c_3-m_3c_1}$$\brak{\frac{m_2-m_1}{m_3-m_1}}$-$\brak{m_1c_2-m_2c_1}$}
\end{align}
Now, for lines to be concurrent, then the third row should be equal to zero. \\

Therefore,
\begin{align}
\brak{m_1c_3-m_3c_1}\brak{\frac{m_2-m_1}{m_3-m_1}}-\brak{m_1c_2-m_2c_1}=0\\
\frac{\brak{m_1c_3-m_3c_1}\brak{m_2-m_1}-\brak{m_1c_2-m_2c_1}\brak{m_3-m_1}}{m_3-m_1}=0\\
\brak{m_1c_3-m_3c_1}\brak{m_2-m_1}-\brak{m_1c_2-m_2c_1}\brak{m_3-m_1}=0\\
m_2c_3-m_1c_3+m_3c_1-m_3c_2+m_1c_2-m_2c_1=0\\
m_1\brak{c_2-c_3}+m_2\brak{c_3-c_1}+m_3\brak{c_1-c_2} = 0
\end{align}
           Hence proved
%\begin{figure}[h]
 %   \centering
  %  \includegraphics[width=\columnwidth]{concurrent-1.png}
   % \caption{Straight Lines}
    %\label{fig:concurrent-1.png}
%\end{figure}
%\end{enumerate}
%\end{document}

\item Find perpendicular distance from the origin to the line joining the points$(\cos\theta,\sin\theta)$ and $(\cos\phi,\sin\phi)$.
\\
\solution
		\input{chapters/11/10/4/5/dist.tex}
\item Find the equation of line which is equidistant from parallel lines $9x+6y-7=0$ and $3x+2y+6=0$.
\\
\solution
		%\documentclass[12pt]{article}
%\usepackage[cmex10]{amsmath}
%\usepackage{amsthm}
%\usepackage{mathrsfs}
%\usepackage{txfonts}
%\usepackage{stfloats}
%\usepackage{bm}
%\usepackage{cite}
%\usepackage{cases}
%\usepackage{subfig}
%\usepackage{longtable}
%\usepackage{multirow}
%\usepackage{enumitem}
%\usepackage{mathtools}
%\usepackage{steinmetz}
%\usepackage{tikz}
%\usepackage{circuitikz}
%\usepackage{verbatim}
%\usepackage{tfrupee}
%\usepackage[breaklinks=true]{hyperref}
%\usepackage{tkz-euclide} % loads  TikZ and tkz-base
%\providecommand{\brak}[1]{\ensuremath{\left(#1\right)}}
%\usepackage{atbegshi}
%\AtBeginDocument{\AtBeginShipoutNext{\AtBeginShipoutDiscard}}
%\usetikzlibrary{calc,math}
%\usepackage{listings}
%    \usepackage{color}                                            %%
%    \usepackage{array}                                            %%
 %   \usepackage{longtable}                                        %%
  %  \usepackage{calc}                                             %%
   % \usepackage{multirow}                                         %%
    %\usepackage{hhline}                                           %%
    %\usepackage{ifthen}                                           %%
  %optionally (for landscape tables embedded in another document): %%
    %\usepackage{lscape}     
%\usepackage{multicol}
%\usepackage{chngcntr}

%\DeclareMathOperator*{\Res}{Res}
%\renewcommand{\baselinestretch}{2}
%\renewcommand\thesection{\arabic{section}}
%\renewcommand\thesubsection{\thesection.\arabic{subsection}}
%\renewcommand\thesubsubsection{\thesubsection.\arabic{subsubsection}}


% correct bad hyphenation here
%\hyphenation{op-tical net-works semi-conduc-tor}
%\def\inputGnumericTable{}                                 %%

%\lstset{
%language=C,
%frame=single, 
%breaklines=true,
%columns=fullflexible
%}
%\begin{document}
%\newtheorem{theorem}{Theorem}[section]
%\newtheorem{problem}{Problem}
%\newtheorem{proposition}{Proposition}[section]
%\newtheorem{lemma}{Lemma}[section]
%\newtheorem{corollary}[theorem]{Corollary}
%\newtheorem{example}{Example}[section]
%\newtheorem{definition}[problem]{Definition}
%\newcommand{\BEQA}{\begin{eqnarray}}
%\newcommand{\EEQA}{\end{eqnarray}}
%\newcommand{\define}{\stackrel{\triangle}{=}}

%\bibliographystyle{IEEEtran}
%\bibliographystyle{ieeetr}
%\providecommand{\mbf}{\mathbf}
%\providecommand{\pr}[1]{\ensuremath{\Pr\left(#1\right)}}
%\providecommand{\qfunc}[1]{\ensuremath{Q\left(#1\right)}}
%\providecommand{\sbrak}[1]{\ensuremath{{}\left[#1\right]}}
%\providecommand{\lsbrak}[1]{\ensuremath{{}\left[#1\right.}}
%\providecommand{\rsbrak}[1]{\ensuremath{{}\left.#1\right]}}
%\providecommand{\brak}[1]{\ensuremath{\left(#1\right)}}
%\providecommand{\lbrak}[1]{\ensuremath{\left(#1\right.}}
%\providecommand{\rbrak}[1]{\ensuremath{\left.#1\right)}}
%\providecommand{\cbrak}[1]{\ensuremath{\left\{#1\right\}}}
%\providecommand{\lcbrak}[1]{\ensuremath{\left\{#1\right.}}
%\providecommand{\rcbrak}[1]{\ensuremath{\left.#1\right\}}}
%\theoremstyle{remark}
%\newtheorem{rem}{Remark}
%\newcommand{\sgn}{\mathop{\mathrm{sgn}}}
%\providecommand{\res}[1]{\Res\displaylimits_{#1}} 
%\providecommand{\mtx}[1]{\mathbf{#1}}
%\providecommand{\fourier}{\overset{\mathcal{F}}{\rightleftharpoons}}
%\providecommand{\system}{\overset{\mathcal{H}}{\longleftrightarrow}}
	%\newcommand{\solution}[2]{\textbf{Solution:}{#1}}
%\newcommand{\solution}{\noindent \textbf{Solution: }}
%\newcommand{\cosec}{\,\text{cosec}\,}
%\providecommand{\dec}[2]{\ensuremath{\overset{#1}{\underset{#2}{\gtrless}}}}
%\newcommand{\myvec}[1]{\ensuremath{\begin{pmatrix}#1\end{pmatrix}}}
%\newcommand{\mydet}[1]{\ensuremath{\begin{vmatrix}#1\end{vmatrix}}}
%\let\vec\mathbf
%\begin{center}
%\title{\textbf{Straight Lines}}
%\date{\vspace{-5ex}} %Not to print date automatically
%\maketitle
%\end{center}
%\setcounter{page}{1}
%\section*{11$^{th}$ Maths - Chapter 10}
%This is Problem-10 from Exercise 10.4
%\begin{enumerate}
%    \item If three lines whose equations are $y=m_1x+c_1$, $y=m_2x+c_2$ and $y=m_3x+c_3$ are concurrent, then show that $m_1(c_2-c_3)+m_2(c_3-c_1)+m_3(c_1-c_2) = 0.$\\
%    \solution 
    Given lines can be written as \begin{align}
       m_1x-y+c_1=0
    \end{align}
    \begin{align}
        m_2x-y+c_2=0
    \end{align}
    \begin{align}
        m_3x-y+c_3=0
        \label{eq:3}
    \end{align}
    
    
   The above lines can be written in the form of \begin{align}
        \Vec{n}^{\top}\Vec{x} = c
    \end{align}
   Therefore,
		\begin{align}
       \myvec{m_1&-1}\vec{x}=c_1
       \label{eq:5}
   \end{align} 
   \begin{align}
       \myvec{m_2&-1}\vec{x}=c_2
       \label{eq:6}
   \end{align}
   \begin{align}
       \myvec{m_3&-1}\vec{x}=c_3
       \label{eq:7}
   \end{align}
   Solving equations \eqref{eq:5}, \eqref{eq:6}and \eqref{eq:7}
		augumented matrix is
 \begin{align}
    \myvec{m_1&-1&c_1\\m_2&-1&c_2\\m_3&-1&c_3}\\
    \xleftrightarrow{R_2 \leftarrow m_1R_2-m_2R_1}
    \myvec{m_1&-1&c_1\\0&m_2-m_1&m_1c_2-m_2c_1\\m_3&-1&c_3}\\
    \xleftrightarrow{R_3 \leftarrow m_1R_3-m_3R_1}
    \myvec{m_1&-1&c_1\\0&m_2-m_1&m_1c_2-m_2c_1\\0&m_3-m_1&m_1c_3-m_3c_1}\\
    \xleftrightarrow{R_3 \leftarrow R_3\frac{m_2-m_1}{m_3-m_1}-R_2}
        \myvec{m_1&-1&c_1\\0&m_2-m_1&m_1c_2-m_2c_1\\0&0&$\brak{m_1c_3-m_3c_1}$$\brak{\frac{m_2-m_1}{m_3-m_1}}$-$\brak{m_1c_2-m_2c_1}$}
\end{align}
Now, for lines to be concurrent, then the third row should be equal to zero. \\

Therefore,
\begin{align}
\brak{m_1c_3-m_3c_1}\brak{\frac{m_2-m_1}{m_3-m_1}}-\brak{m_1c_2-m_2c_1}=0\\
\frac{\brak{m_1c_3-m_3c_1}\brak{m_2-m_1}-\brak{m_1c_2-m_2c_1}\brak{m_3-m_1}}{m_3-m_1}=0\\
\brak{m_1c_3-m_3c_1}\brak{m_2-m_1}-\brak{m_1c_2-m_2c_1}\brak{m_3-m_1}=0\\
m_2c_3-m_1c_3+m_3c_1-m_3c_2+m_1c_2-m_2c_1=0\\
m_1\brak{c_2-c_3}+m_2\brak{c_3-c_1}+m_3\brak{c_1-c_2} = 0
\end{align}
           Hence proved
%\begin{figure}[h]
 %   \centering
  %  \includegraphics[width=\columnwidth]{concurrent-1.png}
   % \caption{Straight Lines}
    %\label{fig:concurrent-1.png}
%\end{figure}
%\end{enumerate}
%\end{document}

	\item Prove that the products of the lengths of the perpendiculars drawn from the points $\myvec{\sqrt{a^2-b^2}\\0}$ and $\myvec{-\sqrt{a^2-b^2} \\0} $ to the line $\frac{x}{a} \cos{\theta} + \frac{y}{b}\sin{\theta} =1 $ is $ b^2 $.
\\
    \solution 
		\input{chapters/11/10/4/23/dist.tex}
\item Find the equation of line  drawn perpendicular to the line $\frac{x}{4}+\frac{y}{6}=1$ through the point where it meets the y-axis \\
\solution
		%\documentclass[12pt]{article}
%\usepackage[cmex10]{amsmath}
%\usepackage{amsthm}
%\usepackage{mathrsfs}
%\usepackage{txfonts}
%\usepackage{stfloats}
%\usepackage{bm}
%\usepackage{cite}
%\usepackage{cases}
%\usepackage{subfig}
%\usepackage{longtable}
%\usepackage{multirow}
%\usepackage{enumitem}
%\usepackage{mathtools}
%\usepackage{steinmetz}
%\usepackage{tikz}
%\usepackage{circuitikz}
%\usepackage{verbatim}
%\usepackage{tfrupee}
%\usepackage[breaklinks=true]{hyperref}
%\usepackage{tkz-euclide} % loads  TikZ and tkz-base
%\providecommand{\brak}[1]{\ensuremath{\left(#1\right)}}
%\usepackage{atbegshi}
%\AtBeginDocument{\AtBeginShipoutNext{\AtBeginShipoutDiscard}}
%\usetikzlibrary{calc,math}
%\usepackage{listings}
%    \usepackage{color}                                            %%
%    \usepackage{array}                                            %%
 %   \usepackage{longtable}                                        %%
  %  \usepackage{calc}                                             %%
   % \usepackage{multirow}                                         %%
    %\usepackage{hhline}                                           %%
    %\usepackage{ifthen}                                           %%
  %optionally (for landscape tables embedded in another document): %%
    %\usepackage{lscape}     
%\usepackage{multicol}
%\usepackage{chngcntr}

%\DeclareMathOperator*{\Res}{Res}
%\renewcommand{\baselinestretch}{2}
%\renewcommand\thesection{\arabic{section}}
%\renewcommand\thesubsection{\thesection.\arabic{subsection}}
%\renewcommand\thesubsubsection{\thesubsection.\arabic{subsubsection}}


% correct bad hyphenation here
%\hyphenation{op-tical net-works semi-conduc-tor}
%\def\inputGnumericTable{}                                 %%

%\lstset{
%language=C,
%frame=single, 
%breaklines=true,
%columns=fullflexible
%}
%\begin{document}
%\newtheorem{theorem}{Theorem}[section]
%\newtheorem{problem}{Problem}
%\newtheorem{proposition}{Proposition}[section]
%\newtheorem{lemma}{Lemma}[section]
%\newtheorem{corollary}[theorem]{Corollary}
%\newtheorem{example}{Example}[section]
%\newtheorem{definition}[problem]{Definition}
%\newcommand{\BEQA}{\begin{eqnarray}}
%\newcommand{\EEQA}{\end{eqnarray}}
%\newcommand{\define}{\stackrel{\triangle}{=}}

%\bibliographystyle{IEEEtran}
%\bibliographystyle{ieeetr}
%\providecommand{\mbf}{\mathbf}
%\providecommand{\pr}[1]{\ensuremath{\Pr\left(#1\right)}}
%\providecommand{\qfunc}[1]{\ensuremath{Q\left(#1\right)}}
%\providecommand{\sbrak}[1]{\ensuremath{{}\left[#1\right]}}
%\providecommand{\lsbrak}[1]{\ensuremath{{}\left[#1\right.}}
%\providecommand{\rsbrak}[1]{\ensuremath{{}\left.#1\right]}}
%\providecommand{\brak}[1]{\ensuremath{\left(#1\right)}}
%\providecommand{\lbrak}[1]{\ensuremath{\left(#1\right.}}
%\providecommand{\rbrak}[1]{\ensuremath{\left.#1\right)}}
%\providecommand{\cbrak}[1]{\ensuremath{\left\{#1\right\}}}
%\providecommand{\lcbrak}[1]{\ensuremath{\left\{#1\right.}}
%\providecommand{\rcbrak}[1]{\ensuremath{\left.#1\right\}}}
%\theoremstyle{remark}
%\newtheorem{rem}{Remark}
%\newcommand{\sgn}{\mathop{\mathrm{sgn}}}
%\providecommand{\res}[1]{\Res\displaylimits_{#1}} 
%\providecommand{\mtx}[1]{\mathbf{#1}}
%\providecommand{\fourier}{\overset{\mathcal{F}}{\rightleftharpoons}}
%\providecommand{\system}{\overset{\mathcal{H}}{\longleftrightarrow}}
	%\newcommand{\solution}[2]{\textbf{Solution:}{#1}}
%\newcommand{\solution}{\noindent \textbf{Solution: }}
%\newcommand{\cosec}{\,\text{cosec}\,}
%\providecommand{\dec}[2]{\ensuremath{\overset{#1}{\underset{#2}{\gtrless}}}}
%\newcommand{\myvec}[1]{\ensuremath{\begin{pmatrix}#1\end{pmatrix}}}
%\newcommand{\mydet}[1]{\ensuremath{\begin{vmatrix}#1\end{vmatrix}}}
%\let\vec\mathbf
%\begin{center}
%\title{\textbf{Straight Lines}}
%\date{\vspace{-5ex}} %Not to print date automatically
%\maketitle
%\end{center}
%\setcounter{page}{1}
%\section*{11$^{th}$ Maths - Chapter 10}
%This is Problem-10 from Exercise 10.4
%\begin{enumerate}
%    \item If three lines whose equations are $y=m_1x+c_1$, $y=m_2x+c_2$ and $y=m_3x+c_3$ are concurrent, then show that $m_1(c_2-c_3)+m_2(c_3-c_1)+m_3(c_1-c_2) = 0.$\\
%    \solution 
    Given lines can be written as \begin{align}
       m_1x-y+c_1=0
    \end{align}
    \begin{align}
        m_2x-y+c_2=0
    \end{align}
    \begin{align}
        m_3x-y+c_3=0
        \label{eq:3}
    \end{align}
    
    
   The above lines can be written in the form of \begin{align}
        \Vec{n}^{\top}\Vec{x} = c
    \end{align}
   Therefore,
		\begin{align}
       \myvec{m_1&-1}\vec{x}=c_1
       \label{eq:5}
   \end{align} 
   \begin{align}
       \myvec{m_2&-1}\vec{x}=c_2
       \label{eq:6}
   \end{align}
   \begin{align}
       \myvec{m_3&-1}\vec{x}=c_3
       \label{eq:7}
   \end{align}
   Solving equations \eqref{eq:5}, \eqref{eq:6}and \eqref{eq:7}
		augumented matrix is
 \begin{align}
    \myvec{m_1&-1&c_1\\m_2&-1&c_2\\m_3&-1&c_3}\\
    \xleftrightarrow{R_2 \leftarrow m_1R_2-m_2R_1}
    \myvec{m_1&-1&c_1\\0&m_2-m_1&m_1c_2-m_2c_1\\m_3&-1&c_3}\\
    \xleftrightarrow{R_3 \leftarrow m_1R_3-m_3R_1}
    \myvec{m_1&-1&c_1\\0&m_2-m_1&m_1c_2-m_2c_1\\0&m_3-m_1&m_1c_3-m_3c_1}\\
    \xleftrightarrow{R_3 \leftarrow R_3\frac{m_2-m_1}{m_3-m_1}-R_2}
        \myvec{m_1&-1&c_1\\0&m_2-m_1&m_1c_2-m_2c_1\\0&0&$\brak{m_1c_3-m_3c_1}$$\brak{\frac{m_2-m_1}{m_3-m_1}}$-$\brak{m_1c_2-m_2c_1}$}
\end{align}
Now, for lines to be concurrent, then the third row should be equal to zero. \\

Therefore,
\begin{align}
\brak{m_1c_3-m_3c_1}\brak{\frac{m_2-m_1}{m_3-m_1}}-\brak{m_1c_2-m_2c_1}=0\\
\frac{\brak{m_1c_3-m_3c_1}\brak{m_2-m_1}-\brak{m_1c_2-m_2c_1}\brak{m_3-m_1}}{m_3-m_1}=0\\
\brak{m_1c_3-m_3c_1}\brak{m_2-m_1}-\brak{m_1c_2-m_2c_1}\brak{m_3-m_1}=0\\
m_2c_3-m_1c_3+m_3c_1-m_3c_2+m_1c_2-m_2c_1=0\\
m_1\brak{c_2-c_3}+m_2\brak{c_3-c_1}+m_3\brak{c_1-c_2} = 0
\end{align}
           Hence proved
%\begin{figure}[h]
 %   \centering
  %  \includegraphics[width=\columnwidth]{concurrent-1.png}
   % \caption{Straight Lines}
    %\label{fig:concurrent-1.png}
%\end{figure}
%\end{enumerate}
%\end{document}

 \item  In each of the following cases, determine the direction cosines of the normal to
the plane and the distance from the origin.
\begin{enumerate}
	\item $z=2$ 
	\item $x + y + z = 1$
	\item $2x + 3y – z = 5$
	\item $5y + 8 = 0$
\end{enumerate}
    \solution
		\input{chapters/12/11/3/1/dist.tex}
\item
Find the angle between the lines whose direction ratios are $a,b,c$ and $b-c,c-a,a-b$.

\textbf{Solution :}
    \begin{align}
    \vec{m _1} &= \myvec{a\\b\\c}\\
    \vec{m_2} &= \myvec{b-c\\c-a\\a-b}\\
    \cos{\theta}&= \frac{\vec{m_1}^{\top}\vec{m_2}}{\vec{\norm{m_1}\norm{m_2}}
   } \\
   &=\frac{\myvec{a&b&c}\myvec{b-c\\c-a\\a-b}}{\sqrt{a^2+b^2+c^2}\sqrt{\brak{b-c}^2+\brak{c-a}^2+\brak{a-b}^2}}\\
   &=0\\
   or,\theta&=\frac{\pi}{2}
    \end{align}

\end{enumerate}

\item Find the distance of the point $(-1,1)$ from the line $12\brak{x+6} = 5\brak{y-2}$. 
\label{chapters/11/10/3/4}
%\documentclass[12pt]{article}
%\usepackage[cmex10]{amsmath}
%\usepackage{amsthm}
%\usepackage{mathrsfs}
%\usepackage{txfonts}
%\usepackage{stfloats}
%\usepackage{bm}
%\usepackage{cite}
%\usepackage{cases}
%\usepackage{subfig}
%\usepackage{longtable}
%\usepackage{multirow}
%\usepackage{enumitem}
%\usepackage{mathtools}
%\usepackage{steinmetz}
%\usepackage{tikz}
%\usepackage{circuitikz}
%\usepackage{verbatim}
%\usepackage{tfrupee}
%\usepackage[breaklinks=true]{hyperref}
%\usepackage{tkz-euclide} % loads  TikZ and tkz-base
%\providecommand{\brak}[1]{\ensuremath{\left(#1\right)}}
%\usepackage{atbegshi}
%\AtBeginDocument{\AtBeginShipoutNext{\AtBeginShipoutDiscard}}
%\usetikzlibrary{calc,math}
%\usepackage{listings}
%    \usepackage{color}                                            %%
%    \usepackage{array}                                            %%
 %   \usepackage{longtable}                                        %%
  %  \usepackage{calc}                                             %%
   % \usepackage{multirow}                                         %%
    %\usepackage{hhline}                                           %%
    %\usepackage{ifthen}                                           %%
  %optionally (for landscape tables embedded in another document): %%
    %\usepackage{lscape}     
%\usepackage{multicol}
%\usepackage{chngcntr}

%\DeclareMathOperator*{\Res}{Res}
%\renewcommand{\baselinestretch}{2}
%\renewcommand\thesection{\arabic{section}}
%\renewcommand\thesubsection{\thesection.\arabic{subsection}}
%\renewcommand\thesubsubsection{\thesubsection.\arabic{subsubsection}}


% correct bad hyphenation here
%\hyphenation{op-tical net-works semi-conduc-tor}
%\def\inputGnumericTable{}                                 %%

%\lstset{
%language=C,
%frame=single, 
%breaklines=true,
%columns=fullflexible
%}
%\begin{document}
%\newtheorem{theorem}{Theorem}[section]
%\newtheorem{problem}{Problem}
%\newtheorem{proposition}{Proposition}[section]
%\newtheorem{lemma}{Lemma}[section]
%\newtheorem{corollary}[theorem]{Corollary}
%\newtheorem{example}{Example}[section]
%\newtheorem{definition}[problem]{Definition}
%\newcommand{\BEQA}{\begin{eqnarray}}
%\newcommand{\EEQA}{\end{eqnarray}}
%\newcommand{\define}{\stackrel{\triangle}{=}}

%\bibliographystyle{IEEEtran}
%\bibliographystyle{ieeetr}
%\providecommand{\mbf}{\mathbf}
%\providecommand{\pr}[1]{\ensuremath{\Pr\left(#1\right)}}
%\providecommand{\qfunc}[1]{\ensuremath{Q\left(#1\right)}}
%\providecommand{\sbrak}[1]{\ensuremath{{}\left[#1\right]}}
%\providecommand{\lsbrak}[1]{\ensuremath{{}\left[#1\right.}}
%\providecommand{\rsbrak}[1]{\ensuremath{{}\left.#1\right]}}
%\providecommand{\brak}[1]{\ensuremath{\left(#1\right)}}
%\providecommand{\lbrak}[1]{\ensuremath{\left(#1\right.}}
%\providecommand{\rbrak}[1]{\ensuremath{\left.#1\right)}}
%\providecommand{\cbrak}[1]{\ensuremath{\left\{#1\right\}}}
%\providecommand{\lcbrak}[1]{\ensuremath{\left\{#1\right.}}
%\providecommand{\rcbrak}[1]{\ensuremath{\left.#1\right\}}}
%\theoremstyle{remark}
%\newtheorem{rem}{Remark}
%\newcommand{\sgn}{\mathop{\mathrm{sgn}}}
%\providecommand{\res}[1]{\Res\displaylimits_{#1}} 
%\providecommand{\mtx}[1]{\mathbf{#1}}
%\providecommand{\fourier}{\overset{\mathcal{F}}{\rightleftharpoons}}
%\providecommand{\system}{\overset{\mathcal{H}}{\longleftrightarrow}}
	%\newcommand{\solution}[2]{\textbf{Solution:}{#1}}
%\newcommand{\solution}{\noindent \textbf{Solution: }}
%\newcommand{\cosec}{\,\text{cosec}\,}
%\providecommand{\dec}[2]{\ensuremath{\overset{#1}{\underset{#2}{\gtrless}}}}
%\newcommand{\myvec}[1]{\ensuremath{\begin{pmatrix}#1\end{pmatrix}}}
%\newcommand{\mydet}[1]{\ensuremath{\begin{vmatrix}#1\end{vmatrix}}}
%\let\vec\mathbf
%\begin{center}
%\title{\textbf{Straight Lines}}
%\date{\vspace{-5ex}} %Not to print date automatically
%\maketitle
%\end{center}
%\setcounter{page}{1}
%\section*{11$^{th}$ Maths - Chapter 10}
%This is Problem-10 from Exercise 10.4
%\begin{enumerate}
%    \item If three lines whose equations are $y=m_1x+c_1$, $y=m_2x+c_2$ and $y=m_3x+c_3$ are concurrent, then show that $m_1(c_2-c_3)+m_2(c_3-c_1)+m_3(c_1-c_2) = 0.$\\
%    \solution 
    Given lines can be written as \begin{align}
       m_1x-y+c_1=0
    \end{align}
    \begin{align}
        m_2x-y+c_2=0
    \end{align}
    \begin{align}
        m_3x-y+c_3=0
        \label{eq:3}
    \end{align}
    
    
   The above lines can be written in the form of \begin{align}
        \Vec{n}^{\top}\Vec{x} = c
    \end{align}
   Therefore,
		\begin{align}
       \myvec{m_1&-1}\vec{x}=c_1
       \label{eq:5}
   \end{align} 
   \begin{align}
       \myvec{m_2&-1}\vec{x}=c_2
       \label{eq:6}
   \end{align}
   \begin{align}
       \myvec{m_3&-1}\vec{x}=c_3
       \label{eq:7}
   \end{align}
   Solving equations \eqref{eq:5}, \eqref{eq:6}and \eqref{eq:7}
		augumented matrix is
 \begin{align}
    \myvec{m_1&-1&c_1\\m_2&-1&c_2\\m_3&-1&c_3}\\
    \xleftrightarrow{R_2 \leftarrow m_1R_2-m_2R_1}
    \myvec{m_1&-1&c_1\\0&m_2-m_1&m_1c_2-m_2c_1\\m_3&-1&c_3}\\
    \xleftrightarrow{R_3 \leftarrow m_1R_3-m_3R_1}
    \myvec{m_1&-1&c_1\\0&m_2-m_1&m_1c_2-m_2c_1\\0&m_3-m_1&m_1c_3-m_3c_1}\\
    \xleftrightarrow{R_3 \leftarrow R_3\frac{m_2-m_1}{m_3-m_1}-R_2}
        \myvec{m_1&-1&c_1\\0&m_2-m_1&m_1c_2-m_2c_1\\0&0&$\brak{m_1c_3-m_3c_1}$$\brak{\frac{m_2-m_1}{m_3-m_1}}$-$\brak{m_1c_2-m_2c_1}$}
\end{align}
Now, for lines to be concurrent, then the third row should be equal to zero. \\

Therefore,
\begin{align}
\brak{m_1c_3-m_3c_1}\brak{\frac{m_2-m_1}{m_3-m_1}}-\brak{m_1c_2-m_2c_1}=0\\
\frac{\brak{m_1c_3-m_3c_1}\brak{m_2-m_1}-\brak{m_1c_2-m_2c_1}\brak{m_3-m_1}}{m_3-m_1}=0\\
\brak{m_1c_3-m_3c_1}\brak{m_2-m_1}-\brak{m_1c_2-m_2c_1}\brak{m_3-m_1}=0\\
m_2c_3-m_1c_3+m_3c_1-m_3c_2+m_1c_2-m_2c_1=0\\
m_1\brak{c_2-c_3}+m_2\brak{c_3-c_1}+m_3\brak{c_1-c_2} = 0
\end{align}
           Hence proved
%\begin{figure}[h]
 %   \centering
  %  \includegraphics[width=\columnwidth]{concurrent-1.png}
   % \caption{Straight Lines}
    %\label{fig:concurrent-1.png}
%\end{figure}
%\end{enumerate}
%\end{document}

\item Find the points on the x-axis, whose distances from the line $\frac{x}{3}+\frac{y}{4}=1$ are 4 units.
\label{chapters/11/10/3/5}
	\\
	\solution
%\documentclass[12pt]{article}
%\usepackage[cmex10]{amsmath}
%\usepackage{amsthm}
%\usepackage{mathrsfs}
%\usepackage{txfonts}
%\usepackage{stfloats}
%\usepackage{bm}
%\usepackage{cite}
%\usepackage{cases}
%\usepackage{subfig}
%\usepackage{longtable}
%\usepackage{multirow}
%\usepackage{enumitem}
%\usepackage{mathtools}
%\usepackage{steinmetz}
%\usepackage{tikz}
%\usepackage{circuitikz}
%\usepackage{verbatim}
%\usepackage{tfrupee}
%\usepackage[breaklinks=true]{hyperref}
%\usepackage{tkz-euclide} % loads  TikZ and tkz-base
%\providecommand{\brak}[1]{\ensuremath{\left(#1\right)}}
%\usepackage{atbegshi}
%\AtBeginDocument{\AtBeginShipoutNext{\AtBeginShipoutDiscard}}
%\usetikzlibrary{calc,math}
%\usepackage{listings}
%    \usepackage{color}                                            %%
%    \usepackage{array}                                            %%
 %   \usepackage{longtable}                                        %%
  %  \usepackage{calc}                                             %%
   % \usepackage{multirow}                                         %%
    %\usepackage{hhline}                                           %%
    %\usepackage{ifthen}                                           %%
  %optionally (for landscape tables embedded in another document): %%
    %\usepackage{lscape}     
%\usepackage{multicol}
%\usepackage{chngcntr}

%\DeclareMathOperator*{\Res}{Res}
%\renewcommand{\baselinestretch}{2}
%\renewcommand\thesection{\arabic{section}}
%\renewcommand\thesubsection{\thesection.\arabic{subsection}}
%\renewcommand\thesubsubsection{\thesubsection.\arabic{subsubsection}}


% correct bad hyphenation here
%\hyphenation{op-tical net-works semi-conduc-tor}
%\def\inputGnumericTable{}                                 %%

%\lstset{
%language=C,
%frame=single, 
%breaklines=true,
%columns=fullflexible
%}
%\begin{document}
%\newtheorem{theorem}{Theorem}[section]
%\newtheorem{problem}{Problem}
%\newtheorem{proposition}{Proposition}[section]
%\newtheorem{lemma}{Lemma}[section]
%\newtheorem{corollary}[theorem]{Corollary}
%\newtheorem{example}{Example}[section]
%\newtheorem{definition}[problem]{Definition}
%\newcommand{\BEQA}{\begin{eqnarray}}
%\newcommand{\EEQA}{\end{eqnarray}}
%\newcommand{\define}{\stackrel{\triangle}{=}}

%\bibliographystyle{IEEEtran}
%\bibliographystyle{ieeetr}
%\providecommand{\mbf}{\mathbf}
%\providecommand{\pr}[1]{\ensuremath{\Pr\left(#1\right)}}
%\providecommand{\qfunc}[1]{\ensuremath{Q\left(#1\right)}}
%\providecommand{\sbrak}[1]{\ensuremath{{}\left[#1\right]}}
%\providecommand{\lsbrak}[1]{\ensuremath{{}\left[#1\right.}}
%\providecommand{\rsbrak}[1]{\ensuremath{{}\left.#1\right]}}
%\providecommand{\brak}[1]{\ensuremath{\left(#1\right)}}
%\providecommand{\lbrak}[1]{\ensuremath{\left(#1\right.}}
%\providecommand{\rbrak}[1]{\ensuremath{\left.#1\right)}}
%\providecommand{\cbrak}[1]{\ensuremath{\left\{#1\right\}}}
%\providecommand{\lcbrak}[1]{\ensuremath{\left\{#1\right.}}
%\providecommand{\rcbrak}[1]{\ensuremath{\left.#1\right\}}}
%\theoremstyle{remark}
%\newtheorem{rem}{Remark}
%\newcommand{\sgn}{\mathop{\mathrm{sgn}}}
%\providecommand{\res}[1]{\Res\displaylimits_{#1}} 
%\providecommand{\mtx}[1]{\mathbf{#1}}
%\providecommand{\fourier}{\overset{\mathcal{F}}{\rightleftharpoons}}
%\providecommand{\system}{\overset{\mathcal{H}}{\longleftrightarrow}}
	%\newcommand{\solution}[2]{\textbf{Solution:}{#1}}
%\newcommand{\solution}{\noindent \textbf{Solution: }}
%\newcommand{\cosec}{\,\text{cosec}\,}
%\providecommand{\dec}[2]{\ensuremath{\overset{#1}{\underset{#2}{\gtrless}}}}
%\newcommand{\myvec}[1]{\ensuremath{\begin{pmatrix}#1\end{pmatrix}}}
%\newcommand{\mydet}[1]{\ensuremath{\begin{vmatrix}#1\end{vmatrix}}}
%\let\vec\mathbf
%\begin{center}
%\title{\textbf{Straight Lines}}
%\date{\vspace{-5ex}} %Not to print date automatically
%\maketitle
%\end{center}
%\setcounter{page}{1}
%\section*{11$^{th}$ Maths - Chapter 10}
%This is Problem-10 from Exercise 10.4
%\begin{enumerate}
%    \item If three lines whose equations are $y=m_1x+c_1$, $y=m_2x+c_2$ and $y=m_3x+c_3$ are concurrent, then show that $m_1(c_2-c_3)+m_2(c_3-c_1)+m_3(c_1-c_2) = 0.$\\
%    \solution 
    Given lines can be written as \begin{align}
       m_1x-y+c_1=0
    \end{align}
    \begin{align}
        m_2x-y+c_2=0
    \end{align}
    \begin{align}
        m_3x-y+c_3=0
        \label{eq:3}
    \end{align}
    
    
   The above lines can be written in the form of \begin{align}
        \Vec{n}^{\top}\Vec{x} = c
    \end{align}
   Therefore,
		\begin{align}
       \myvec{m_1&-1}\vec{x}=c_1
       \label{eq:5}
   \end{align} 
   \begin{align}
       \myvec{m_2&-1}\vec{x}=c_2
       \label{eq:6}
   \end{align}
   \begin{align}
       \myvec{m_3&-1}\vec{x}=c_3
       \label{eq:7}
   \end{align}
   Solving equations \eqref{eq:5}, \eqref{eq:6}and \eqref{eq:7}
		augumented matrix is
 \begin{align}
    \myvec{m_1&-1&c_1\\m_2&-1&c_2\\m_3&-1&c_3}\\
    \xleftrightarrow{R_2 \leftarrow m_1R_2-m_2R_1}
    \myvec{m_1&-1&c_1\\0&m_2-m_1&m_1c_2-m_2c_1\\m_3&-1&c_3}\\
    \xleftrightarrow{R_3 \leftarrow m_1R_3-m_3R_1}
    \myvec{m_1&-1&c_1\\0&m_2-m_1&m_1c_2-m_2c_1\\0&m_3-m_1&m_1c_3-m_3c_1}\\
    \xleftrightarrow{R_3 \leftarrow R_3\frac{m_2-m_1}{m_3-m_1}-R_2}
        \myvec{m_1&-1&c_1\\0&m_2-m_1&m_1c_2-m_2c_1\\0&0&$\brak{m_1c_3-m_3c_1}$$\brak{\frac{m_2-m_1}{m_3-m_1}}$-$\brak{m_1c_2-m_2c_1}$}
\end{align}
Now, for lines to be concurrent, then the third row should be equal to zero. \\

Therefore,
\begin{align}
\brak{m_1c_3-m_3c_1}\brak{\frac{m_2-m_1}{m_3-m_1}}-\brak{m_1c_2-m_2c_1}=0\\
\frac{\brak{m_1c_3-m_3c_1}\brak{m_2-m_1}-\brak{m_1c_2-m_2c_1}\brak{m_3-m_1}}{m_3-m_1}=0\\
\brak{m_1c_3-m_3c_1}\brak{m_2-m_1}-\brak{m_1c_2-m_2c_1}\brak{m_3-m_1}=0\\
m_2c_3-m_1c_3+m_3c_1-m_3c_2+m_1c_2-m_2c_1=0\\
m_1\brak{c_2-c_3}+m_2\brak{c_3-c_1}+m_3\brak{c_1-c_2} = 0
\end{align}
           Hence proved
%\begin{figure}[h]
 %   \centering
  %  \includegraphics[width=\columnwidth]{concurrent-1.png}
   % \caption{Straight Lines}
    %\label{fig:concurrent-1.png}
%\end{figure}
%\end{enumerate}
%\end{document}

\item Find the distance between parallel lines
\label{chapters/11/10/3/6}
\begin{enumerate}
	\item $15x+8y-34=0$ and  $15x+8y+31=0$ \\
	\item  $l(x+y)+p=0$ and  $l(x+y)-r=0$
\end{enumerate}
	\solution
\input{chapters/11/10/3/6/dist.tex}
\item Find the coordinates of the foot of the perpendicular from $(-1, 3)$ to the line $3x-4y-16=0$.  
\label{chapters/11/10/3/14}
\\
\solution
%\documentclass[12pt]{article}
%\usepackage[cmex10]{amsmath}
%\usepackage{amsthm}
%\usepackage{mathrsfs}
%\usepackage{txfonts}
%\usepackage{stfloats}
%\usepackage{bm}
%\usepackage{cite}
%\usepackage{cases}
%\usepackage{subfig}
%\usepackage{longtable}
%\usepackage{multirow}
%\usepackage{enumitem}
%\usepackage{mathtools}
%\usepackage{steinmetz}
%\usepackage{tikz}
%\usepackage{circuitikz}
%\usepackage{verbatim}
%\usepackage{tfrupee}
%\usepackage[breaklinks=true]{hyperref}
%\usepackage{tkz-euclide} % loads  TikZ and tkz-base
%\providecommand{\brak}[1]{\ensuremath{\left(#1\right)}}
%\usepackage{atbegshi}
%\AtBeginDocument{\AtBeginShipoutNext{\AtBeginShipoutDiscard}}
%\usetikzlibrary{calc,math}
%\usepackage{listings}
%    \usepackage{color}                                            %%
%    \usepackage{array}                                            %%
 %   \usepackage{longtable}                                        %%
  %  \usepackage{calc}                                             %%
   % \usepackage{multirow}                                         %%
    %\usepackage{hhline}                                           %%
    %\usepackage{ifthen}                                           %%
  %optionally (for landscape tables embedded in another document): %%
    %\usepackage{lscape}     
%\usepackage{multicol}
%\usepackage{chngcntr}

%\DeclareMathOperator*{\Res}{Res}
%\renewcommand{\baselinestretch}{2}
%\renewcommand\thesection{\arabic{section}}
%\renewcommand\thesubsection{\thesection.\arabic{subsection}}
%\renewcommand\thesubsubsection{\thesubsection.\arabic{subsubsection}}


% correct bad hyphenation here
%\hyphenation{op-tical net-works semi-conduc-tor}
%\def\inputGnumericTable{}                                 %%

%\lstset{
%language=C,
%frame=single, 
%breaklines=true,
%columns=fullflexible
%}
%\begin{document}
%\newtheorem{theorem}{Theorem}[section]
%\newtheorem{problem}{Problem}
%\newtheorem{proposition}{Proposition}[section]
%\newtheorem{lemma}{Lemma}[section]
%\newtheorem{corollary}[theorem]{Corollary}
%\newtheorem{example}{Example}[section]
%\newtheorem{definition}[problem]{Definition}
%\newcommand{\BEQA}{\begin{eqnarray}}
%\newcommand{\EEQA}{\end{eqnarray}}
%\newcommand{\define}{\stackrel{\triangle}{=}}

%\bibliographystyle{IEEEtran}
%\bibliographystyle{ieeetr}
%\providecommand{\mbf}{\mathbf}
%\providecommand{\pr}[1]{\ensuremath{\Pr\left(#1\right)}}
%\providecommand{\qfunc}[1]{\ensuremath{Q\left(#1\right)}}
%\providecommand{\sbrak}[1]{\ensuremath{{}\left[#1\right]}}
%\providecommand{\lsbrak}[1]{\ensuremath{{}\left[#1\right.}}
%\providecommand{\rsbrak}[1]{\ensuremath{{}\left.#1\right]}}
%\providecommand{\brak}[1]{\ensuremath{\left(#1\right)}}
%\providecommand{\lbrak}[1]{\ensuremath{\left(#1\right.}}
%\providecommand{\rbrak}[1]{\ensuremath{\left.#1\right)}}
%\providecommand{\cbrak}[1]{\ensuremath{\left\{#1\right\}}}
%\providecommand{\lcbrak}[1]{\ensuremath{\left\{#1\right.}}
%\providecommand{\rcbrak}[1]{\ensuremath{\left.#1\right\}}}
%\theoremstyle{remark}
%\newtheorem{rem}{Remark}
%\newcommand{\sgn}{\mathop{\mathrm{sgn}}}
%\providecommand{\res}[1]{\Res\displaylimits_{#1}} 
%\providecommand{\mtx}[1]{\mathbf{#1}}
%\providecommand{\fourier}{\overset{\mathcal{F}}{\rightleftharpoons}}
%\providecommand{\system}{\overset{\mathcal{H}}{\longleftrightarrow}}
	%\newcommand{\solution}[2]{\textbf{Solution:}{#1}}
%\newcommand{\solution}{\noindent \textbf{Solution: }}
%\newcommand{\cosec}{\,\text{cosec}\,}
%\providecommand{\dec}[2]{\ensuremath{\overset{#1}{\underset{#2}{\gtrless}}}}
%\newcommand{\myvec}[1]{\ensuremath{\begin{pmatrix}#1\end{pmatrix}}}
%\newcommand{\mydet}[1]{\ensuremath{\begin{vmatrix}#1\end{vmatrix}}}
%\let\vec\mathbf
%\begin{center}
%\title{\textbf{Straight Lines}}
%\date{\vspace{-5ex}} %Not to print date automatically
%\maketitle
%\end{center}
%\setcounter{page}{1}
%\section*{11$^{th}$ Maths - Chapter 10}
%This is Problem-10 from Exercise 10.4
%\begin{enumerate}
%    \item If three lines whose equations are $y=m_1x+c_1$, $y=m_2x+c_2$ and $y=m_3x+c_3$ are concurrent, then show that $m_1(c_2-c_3)+m_2(c_3-c_1)+m_3(c_1-c_2) = 0.$\\
%    \solution 
    Given lines can be written as \begin{align}
       m_1x-y+c_1=0
    \end{align}
    \begin{align}
        m_2x-y+c_2=0
    \end{align}
    \begin{align}
        m_3x-y+c_3=0
        \label{eq:3}
    \end{align}
    
    
   The above lines can be written in the form of \begin{align}
        \Vec{n}^{\top}\Vec{x} = c
    \end{align}
   Therefore,
		\begin{align}
       \myvec{m_1&-1}\vec{x}=c_1
       \label{eq:5}
   \end{align} 
   \begin{align}
       \myvec{m_2&-1}\vec{x}=c_2
       \label{eq:6}
   \end{align}
   \begin{align}
       \myvec{m_3&-1}\vec{x}=c_3
       \label{eq:7}
   \end{align}
   Solving equations \eqref{eq:5}, \eqref{eq:6}and \eqref{eq:7}
		augumented matrix is
 \begin{align}
    \myvec{m_1&-1&c_1\\m_2&-1&c_2\\m_3&-1&c_3}\\
    \xleftrightarrow{R_2 \leftarrow m_1R_2-m_2R_1}
    \myvec{m_1&-1&c_1\\0&m_2-m_1&m_1c_2-m_2c_1\\m_3&-1&c_3}\\
    \xleftrightarrow{R_3 \leftarrow m_1R_3-m_3R_1}
    \myvec{m_1&-1&c_1\\0&m_2-m_1&m_1c_2-m_2c_1\\0&m_3-m_1&m_1c_3-m_3c_1}\\
    \xleftrightarrow{R_3 \leftarrow R_3\frac{m_2-m_1}{m_3-m_1}-R_2}
        \myvec{m_1&-1&c_1\\0&m_2-m_1&m_1c_2-m_2c_1\\0&0&$\brak{m_1c_3-m_3c_1}$$\brak{\frac{m_2-m_1}{m_3-m_1}}$-$\brak{m_1c_2-m_2c_1}$}
\end{align}
Now, for lines to be concurrent, then the third row should be equal to zero. \\

Therefore,
\begin{align}
\brak{m_1c_3-m_3c_1}\brak{\frac{m_2-m_1}{m_3-m_1}}-\brak{m_1c_2-m_2c_1}=0\\
\frac{\brak{m_1c_3-m_3c_1}\brak{m_2-m_1}-\brak{m_1c_2-m_2c_1}\brak{m_3-m_1}}{m_3-m_1}=0\\
\brak{m_1c_3-m_3c_1}\brak{m_2-m_1}-\brak{m_1c_2-m_2c_1}\brak{m_3-m_1}=0\\
m_2c_3-m_1c_3+m_3c_1-m_3c_2+m_1c_2-m_2c_1=0\\
m_1\brak{c_2-c_3}+m_2\brak{c_3-c_1}+m_3\brak{c_1-c_2} = 0
\end{align}
           Hence proved
%\begin{figure}[h]
 %   \centering
  %  \includegraphics[width=\columnwidth]{concurrent-1.png}
   % \caption{Straight Lines}
    %\label{fig:concurrent-1.png}
%\end{figure}
%\end{enumerate}
%\end{document}

\item  If ${p}$ and ${q}$ are the lengths of perpendiculars from the origin to the lines ${x}\cos\theta - {y}\sin\theta =  {k}\cos2\theta$ and ${x}\sec\theta + {y}\cosec\theta = {k}$, respectively, prove that ${p}^2 + 4{q}^2 = {k}^2$
\label{chapters/11/10/3/16}
\\
\solution
%\documentclass[12pt]{article}
%\usepackage[cmex10]{amsmath}
%\usepackage{amsthm}
%\usepackage{mathrsfs}
%\usepackage{txfonts}
%\usepackage{stfloats}
%\usepackage{bm}
%\usepackage{cite}
%\usepackage{cases}
%\usepackage{subfig}
%\usepackage{longtable}
%\usepackage{multirow}
%\usepackage{enumitem}
%\usepackage{mathtools}
%\usepackage{steinmetz}
%\usepackage{tikz}
%\usepackage{circuitikz}
%\usepackage{verbatim}
%\usepackage{tfrupee}
%\usepackage[breaklinks=true]{hyperref}
%\usepackage{tkz-euclide} % loads  TikZ and tkz-base
%\providecommand{\brak}[1]{\ensuremath{\left(#1\right)}}
%\usepackage{atbegshi}
%\AtBeginDocument{\AtBeginShipoutNext{\AtBeginShipoutDiscard}}
%\usetikzlibrary{calc,math}
%\usepackage{listings}
%    \usepackage{color}                                            %%
%    \usepackage{array}                                            %%
 %   \usepackage{longtable}                                        %%
  %  \usepackage{calc}                                             %%
   % \usepackage{multirow}                                         %%
    %\usepackage{hhline}                                           %%
    %\usepackage{ifthen}                                           %%
  %optionally (for landscape tables embedded in another document): %%
    %\usepackage{lscape}     
%\usepackage{multicol}
%\usepackage{chngcntr}

%\DeclareMathOperator*{\Res}{Res}
%\renewcommand{\baselinestretch}{2}
%\renewcommand\thesection{\arabic{section}}
%\renewcommand\thesubsection{\thesection.\arabic{subsection}}
%\renewcommand\thesubsubsection{\thesubsection.\arabic{subsubsection}}


% correct bad hyphenation here
%\hyphenation{op-tical net-works semi-conduc-tor}
%\def\inputGnumericTable{}                                 %%

%\lstset{
%language=C,
%frame=single, 
%breaklines=true,
%columns=fullflexible
%}
%\begin{document}
%\newtheorem{theorem}{Theorem}[section]
%\newtheorem{problem}{Problem}
%\newtheorem{proposition}{Proposition}[section]
%\newtheorem{lemma}{Lemma}[section]
%\newtheorem{corollary}[theorem]{Corollary}
%\newtheorem{example}{Example}[section]
%\newtheorem{definition}[problem]{Definition}
%\newcommand{\BEQA}{\begin{eqnarray}}
%\newcommand{\EEQA}{\end{eqnarray}}
%\newcommand{\define}{\stackrel{\triangle}{=}}

%\bibliographystyle{IEEEtran}
%\bibliographystyle{ieeetr}
%\providecommand{\mbf}{\mathbf}
%\providecommand{\pr}[1]{\ensuremath{\Pr\left(#1\right)}}
%\providecommand{\qfunc}[1]{\ensuremath{Q\left(#1\right)}}
%\providecommand{\sbrak}[1]{\ensuremath{{}\left[#1\right]}}
%\providecommand{\lsbrak}[1]{\ensuremath{{}\left[#1\right.}}
%\providecommand{\rsbrak}[1]{\ensuremath{{}\left.#1\right]}}
%\providecommand{\brak}[1]{\ensuremath{\left(#1\right)}}
%\providecommand{\lbrak}[1]{\ensuremath{\left(#1\right.}}
%\providecommand{\rbrak}[1]{\ensuremath{\left.#1\right)}}
%\providecommand{\cbrak}[1]{\ensuremath{\left\{#1\right\}}}
%\providecommand{\lcbrak}[1]{\ensuremath{\left\{#1\right.}}
%\providecommand{\rcbrak}[1]{\ensuremath{\left.#1\right\}}}
%\theoremstyle{remark}
%\newtheorem{rem}{Remark}
%\newcommand{\sgn}{\mathop{\mathrm{sgn}}}
%\providecommand{\res}[1]{\Res\displaylimits_{#1}} 
%\providecommand{\mtx}[1]{\mathbf{#1}}
%\providecommand{\fourier}{\overset{\mathcal{F}}{\rightleftharpoons}}
%\providecommand{\system}{\overset{\mathcal{H}}{\longleftrightarrow}}
	%\newcommand{\solution}[2]{\textbf{Solution:}{#1}}
%\newcommand{\solution}{\noindent \textbf{Solution: }}
%\newcommand{\cosec}{\,\text{cosec}\,}
%\providecommand{\dec}[2]{\ensuremath{\overset{#1}{\underset{#2}{\gtrless}}}}
%\newcommand{\myvec}[1]{\ensuremath{\begin{pmatrix}#1\end{pmatrix}}}
%\newcommand{\mydet}[1]{\ensuremath{\begin{vmatrix}#1\end{vmatrix}}}
%\let\vec\mathbf
%\begin{center}
%\title{\textbf{Straight Lines}}
%\date{\vspace{-5ex}} %Not to print date automatically
%\maketitle
%\end{center}
%\setcounter{page}{1}
%\section*{11$^{th}$ Maths - Chapter 10}
%This is Problem-10 from Exercise 10.4
%\begin{enumerate}
%    \item If three lines whose equations are $y=m_1x+c_1$, $y=m_2x+c_2$ and $y=m_3x+c_3$ are concurrent, then show that $m_1(c_2-c_3)+m_2(c_3-c_1)+m_3(c_1-c_2) = 0.$\\
%    \solution 
    Given lines can be written as \begin{align}
       m_1x-y+c_1=0
    \end{align}
    \begin{align}
        m_2x-y+c_2=0
    \end{align}
    \begin{align}
        m_3x-y+c_3=0
        \label{eq:3}
    \end{align}
    
    
   The above lines can be written in the form of \begin{align}
        \Vec{n}^{\top}\Vec{x} = c
    \end{align}
   Therefore,
		\begin{align}
       \myvec{m_1&-1}\vec{x}=c_1
       \label{eq:5}
   \end{align} 
   \begin{align}
       \myvec{m_2&-1}\vec{x}=c_2
       \label{eq:6}
   \end{align}
   \begin{align}
       \myvec{m_3&-1}\vec{x}=c_3
       \label{eq:7}
   \end{align}
   Solving equations \eqref{eq:5}, \eqref{eq:6}and \eqref{eq:7}
		augumented matrix is
 \begin{align}
    \myvec{m_1&-1&c_1\\m_2&-1&c_2\\m_3&-1&c_3}\\
    \xleftrightarrow{R_2 \leftarrow m_1R_2-m_2R_1}
    \myvec{m_1&-1&c_1\\0&m_2-m_1&m_1c_2-m_2c_1\\m_3&-1&c_3}\\
    \xleftrightarrow{R_3 \leftarrow m_1R_3-m_3R_1}
    \myvec{m_1&-1&c_1\\0&m_2-m_1&m_1c_2-m_2c_1\\0&m_3-m_1&m_1c_3-m_3c_1}\\
    \xleftrightarrow{R_3 \leftarrow R_3\frac{m_2-m_1}{m_3-m_1}-R_2}
        \myvec{m_1&-1&c_1\\0&m_2-m_1&m_1c_2-m_2c_1\\0&0&$\brak{m_1c_3-m_3c_1}$$\brak{\frac{m_2-m_1}{m_3-m_1}}$-$\brak{m_1c_2-m_2c_1}$}
\end{align}
Now, for lines to be concurrent, then the third row should be equal to zero. \\

Therefore,
\begin{align}
\brak{m_1c_3-m_3c_1}\brak{\frac{m_2-m_1}{m_3-m_1}}-\brak{m_1c_2-m_2c_1}=0\\
\frac{\brak{m_1c_3-m_3c_1}\brak{m_2-m_1}-\brak{m_1c_2-m_2c_1}\brak{m_3-m_1}}{m_3-m_1}=0\\
\brak{m_1c_3-m_3c_1}\brak{m_2-m_1}-\brak{m_1c_2-m_2c_1}\brak{m_3-m_1}=0\\
m_2c_3-m_1c_3+m_3c_1-m_3c_2+m_1c_2-m_2c_1=0\\
m_1\brak{c_2-c_3}+m_2\brak{c_3-c_1}+m_3\brak{c_1-c_2} = 0
\end{align}
           Hence proved
%\begin{figure}[h]
 %   \centering
  %  \includegraphics[width=\columnwidth]{concurrent-1.png}
   % \caption{Straight Lines}
    %\label{fig:concurrent-1.png}
%\end{figure}
%\end{enumerate}
%\end{document}

\item In the triangle $ABC$ with vertices $\vec{A} \brak{2, 3}$, $\vec{B} \brak{4, –1}$ and $\vec{C} \brak{1, 2}$, find the equation and length of altitude from the vertex $\vec{A}$.
\label{chapters/11/10/3/17}
\\
\solution
%\documentclass[12pt]{article}
%\usepackage[cmex10]{amsmath}
%\usepackage{amsthm}
%\usepackage{mathrsfs}
%\usepackage{txfonts}
%\usepackage{stfloats}
%\usepackage{bm}
%\usepackage{cite}
%\usepackage{cases}
%\usepackage{subfig}
%\usepackage{longtable}
%\usepackage{multirow}
%\usepackage{enumitem}
%\usepackage{mathtools}
%\usepackage{steinmetz}
%\usepackage{tikz}
%\usepackage{circuitikz}
%\usepackage{verbatim}
%\usepackage{tfrupee}
%\usepackage[breaklinks=true]{hyperref}
%\usepackage{tkz-euclide} % loads  TikZ and tkz-base
%\providecommand{\brak}[1]{\ensuremath{\left(#1\right)}}
%\usepackage{atbegshi}
%\AtBeginDocument{\AtBeginShipoutNext{\AtBeginShipoutDiscard}}
%\usetikzlibrary{calc,math}
%\usepackage{listings}
%    \usepackage{color}                                            %%
%    \usepackage{array}                                            %%
 %   \usepackage{longtable}                                        %%
  %  \usepackage{calc}                                             %%
   % \usepackage{multirow}                                         %%
    %\usepackage{hhline}                                           %%
    %\usepackage{ifthen}                                           %%
  %optionally (for landscape tables embedded in another document): %%
    %\usepackage{lscape}     
%\usepackage{multicol}
%\usepackage{chngcntr}

%\DeclareMathOperator*{\Res}{Res}
%\renewcommand{\baselinestretch}{2}
%\renewcommand\thesection{\arabic{section}}
%\renewcommand\thesubsection{\thesection.\arabic{subsection}}
%\renewcommand\thesubsubsection{\thesubsection.\arabic{subsubsection}}


% correct bad hyphenation here
%\hyphenation{op-tical net-works semi-conduc-tor}
%\def\inputGnumericTable{}                                 %%

%\lstset{
%language=C,
%frame=single, 
%breaklines=true,
%columns=fullflexible
%}
%\begin{document}
%\newtheorem{theorem}{Theorem}[section]
%\newtheorem{problem}{Problem}
%\newtheorem{proposition}{Proposition}[section]
%\newtheorem{lemma}{Lemma}[section]
%\newtheorem{corollary}[theorem]{Corollary}
%\newtheorem{example}{Example}[section]
%\newtheorem{definition}[problem]{Definition}
%\newcommand{\BEQA}{\begin{eqnarray}}
%\newcommand{\EEQA}{\end{eqnarray}}
%\newcommand{\define}{\stackrel{\triangle}{=}}

%\bibliographystyle{IEEEtran}
%\bibliographystyle{ieeetr}
%\providecommand{\mbf}{\mathbf}
%\providecommand{\pr}[1]{\ensuremath{\Pr\left(#1\right)}}
%\providecommand{\qfunc}[1]{\ensuremath{Q\left(#1\right)}}
%\providecommand{\sbrak}[1]{\ensuremath{{}\left[#1\right]}}
%\providecommand{\lsbrak}[1]{\ensuremath{{}\left[#1\right.}}
%\providecommand{\rsbrak}[1]{\ensuremath{{}\left.#1\right]}}
%\providecommand{\brak}[1]{\ensuremath{\left(#1\right)}}
%\providecommand{\lbrak}[1]{\ensuremath{\left(#1\right.}}
%\providecommand{\rbrak}[1]{\ensuremath{\left.#1\right)}}
%\providecommand{\cbrak}[1]{\ensuremath{\left\{#1\right\}}}
%\providecommand{\lcbrak}[1]{\ensuremath{\left\{#1\right.}}
%\providecommand{\rcbrak}[1]{\ensuremath{\left.#1\right\}}}
%\theoremstyle{remark}
%\newtheorem{rem}{Remark}
%\newcommand{\sgn}{\mathop{\mathrm{sgn}}}
%\providecommand{\res}[1]{\Res\displaylimits_{#1}} 
%\providecommand{\mtx}[1]{\mathbf{#1}}
%\providecommand{\fourier}{\overset{\mathcal{F}}{\rightleftharpoons}}
%\providecommand{\system}{\overset{\mathcal{H}}{\longleftrightarrow}}
	%\newcommand{\solution}[2]{\textbf{Solution:}{#1}}
%\newcommand{\solution}{\noindent \textbf{Solution: }}
%\newcommand{\cosec}{\,\text{cosec}\,}
%\providecommand{\dec}[2]{\ensuremath{\overset{#1}{\underset{#2}{\gtrless}}}}
%\newcommand{\myvec}[1]{\ensuremath{\begin{pmatrix}#1\end{pmatrix}}}
%\newcommand{\mydet}[1]{\ensuremath{\begin{vmatrix}#1\end{vmatrix}}}
%\let\vec\mathbf
%\begin{center}
%\title{\textbf{Straight Lines}}
%\date{\vspace{-5ex}} %Not to print date automatically
%\maketitle
%\end{center}
%\setcounter{page}{1}
%\section*{11$^{th}$ Maths - Chapter 10}
%This is Problem-10 from Exercise 10.4
%\begin{enumerate}
%    \item If three lines whose equations are $y=m_1x+c_1$, $y=m_2x+c_2$ and $y=m_3x+c_3$ are concurrent, then show that $m_1(c_2-c_3)+m_2(c_3-c_1)+m_3(c_1-c_2) = 0.$\\
%    \solution 
    Given lines can be written as \begin{align}
       m_1x-y+c_1=0
    \end{align}
    \begin{align}
        m_2x-y+c_2=0
    \end{align}
    \begin{align}
        m_3x-y+c_3=0
        \label{eq:3}
    \end{align}
    
    
   The above lines can be written in the form of \begin{align}
        \Vec{n}^{\top}\Vec{x} = c
    \end{align}
   Therefore,
		\begin{align}
       \myvec{m_1&-1}\vec{x}=c_1
       \label{eq:5}
   \end{align} 
   \begin{align}
       \myvec{m_2&-1}\vec{x}=c_2
       \label{eq:6}
   \end{align}
   \begin{align}
       \myvec{m_3&-1}\vec{x}=c_3
       \label{eq:7}
   \end{align}
   Solving equations \eqref{eq:5}, \eqref{eq:6}and \eqref{eq:7}
		augumented matrix is
 \begin{align}
    \myvec{m_1&-1&c_1\\m_2&-1&c_2\\m_3&-1&c_3}\\
    \xleftrightarrow{R_2 \leftarrow m_1R_2-m_2R_1}
    \myvec{m_1&-1&c_1\\0&m_2-m_1&m_1c_2-m_2c_1\\m_3&-1&c_3}\\
    \xleftrightarrow{R_3 \leftarrow m_1R_3-m_3R_1}
    \myvec{m_1&-1&c_1\\0&m_2-m_1&m_1c_2-m_2c_1\\0&m_3-m_1&m_1c_3-m_3c_1}\\
    \xleftrightarrow{R_3 \leftarrow R_3\frac{m_2-m_1}{m_3-m_1}-R_2}
        \myvec{m_1&-1&c_1\\0&m_2-m_1&m_1c_2-m_2c_1\\0&0&$\brak{m_1c_3-m_3c_1}$$\brak{\frac{m_2-m_1}{m_3-m_1}}$-$\brak{m_1c_2-m_2c_1}$}
\end{align}
Now, for lines to be concurrent, then the third row should be equal to zero. \\

Therefore,
\begin{align}
\brak{m_1c_3-m_3c_1}\brak{\frac{m_2-m_1}{m_3-m_1}}-\brak{m_1c_2-m_2c_1}=0\\
\frac{\brak{m_1c_3-m_3c_1}\brak{m_2-m_1}-\brak{m_1c_2-m_2c_1}\brak{m_3-m_1}}{m_3-m_1}=0\\
\brak{m_1c_3-m_3c_1}\brak{m_2-m_1}-\brak{m_1c_2-m_2c_1}\brak{m_3-m_1}=0\\
m_2c_3-m_1c_3+m_3c_1-m_3c_2+m_1c_2-m_2c_1=0\\
m_1\brak{c_2-c_3}+m_2\brak{c_3-c_1}+m_3\brak{c_1-c_2} = 0
\end{align}
           Hence proved
%\begin{figure}[h]
 %   \centering
  %  \includegraphics[width=\columnwidth]{concurrent-1.png}
   % \caption{Straight Lines}
    %\label{fig:concurrent-1.png}
%\end{figure}
%\end{enumerate}
%\end{document}

\item If $p$ is the length of perpendicular from origin to the line whose intercepts on the axes are $a$ and $b$, then show that 
\begin{align}
	\frac{1}{p^2} = \frac{1}{a^2}+ \frac{1}{b^2}
\end{align}
\label{chapters/11/10/3/18}
\input{chapters/11/10/3/18/dist.tex}
\item What are the points on the y-axis whose distance from the line $\frac{x}{3}+\frac{y}{4}=1$ is 4 units.
\\
\solution
		%\documentclass[12pt]{article}
%\usepackage[cmex10]{amsmath}
%\usepackage{amsthm}
%\usepackage{mathrsfs}
%\usepackage{txfonts}
%\usepackage{stfloats}
%\usepackage{bm}
%\usepackage{cite}
%\usepackage{cases}
%\usepackage{subfig}
%\usepackage{longtable}
%\usepackage{multirow}
%\usepackage{enumitem}
%\usepackage{mathtools}
%\usepackage{steinmetz}
%\usepackage{tikz}
%\usepackage{circuitikz}
%\usepackage{verbatim}
%\usepackage{tfrupee}
%\usepackage[breaklinks=true]{hyperref}
%\usepackage{tkz-euclide} % loads  TikZ and tkz-base
%\providecommand{\brak}[1]{\ensuremath{\left(#1\right)}}
%\usepackage{atbegshi}
%\AtBeginDocument{\AtBeginShipoutNext{\AtBeginShipoutDiscard}}
%\usetikzlibrary{calc,math}
%\usepackage{listings}
%    \usepackage{color}                                            %%
%    \usepackage{array}                                            %%
 %   \usepackage{longtable}                                        %%
  %  \usepackage{calc}                                             %%
   % \usepackage{multirow}                                         %%
    %\usepackage{hhline}                                           %%
    %\usepackage{ifthen}                                           %%
  %optionally (for landscape tables embedded in another document): %%
    %\usepackage{lscape}     
%\usepackage{multicol}
%\usepackage{chngcntr}

%\DeclareMathOperator*{\Res}{Res}
%\renewcommand{\baselinestretch}{2}
%\renewcommand\thesection{\arabic{section}}
%\renewcommand\thesubsection{\thesection.\arabic{subsection}}
%\renewcommand\thesubsubsection{\thesubsection.\arabic{subsubsection}}


% correct bad hyphenation here
%\hyphenation{op-tical net-works semi-conduc-tor}
%\def\inputGnumericTable{}                                 %%

%\lstset{
%language=C,
%frame=single, 
%breaklines=true,
%columns=fullflexible
%}
%\begin{document}
%\newtheorem{theorem}{Theorem}[section]
%\newtheorem{problem}{Problem}
%\newtheorem{proposition}{Proposition}[section]
%\newtheorem{lemma}{Lemma}[section]
%\newtheorem{corollary}[theorem]{Corollary}
%\newtheorem{example}{Example}[section]
%\newtheorem{definition}[problem]{Definition}
%\newcommand{\BEQA}{\begin{eqnarray}}
%\newcommand{\EEQA}{\end{eqnarray}}
%\newcommand{\define}{\stackrel{\triangle}{=}}

%\bibliographystyle{IEEEtran}
%\bibliographystyle{ieeetr}
%\providecommand{\mbf}{\mathbf}
%\providecommand{\pr}[1]{\ensuremath{\Pr\left(#1\right)}}
%\providecommand{\qfunc}[1]{\ensuremath{Q\left(#1\right)}}
%\providecommand{\sbrak}[1]{\ensuremath{{}\left[#1\right]}}
%\providecommand{\lsbrak}[1]{\ensuremath{{}\left[#1\right.}}
%\providecommand{\rsbrak}[1]{\ensuremath{{}\left.#1\right]}}
%\providecommand{\brak}[1]{\ensuremath{\left(#1\right)}}
%\providecommand{\lbrak}[1]{\ensuremath{\left(#1\right.}}
%\providecommand{\rbrak}[1]{\ensuremath{\left.#1\right)}}
%\providecommand{\cbrak}[1]{\ensuremath{\left\{#1\right\}}}
%\providecommand{\lcbrak}[1]{\ensuremath{\left\{#1\right.}}
%\providecommand{\rcbrak}[1]{\ensuremath{\left.#1\right\}}}
%\theoremstyle{remark}
%\newtheorem{rem}{Remark}
%\newcommand{\sgn}{\mathop{\mathrm{sgn}}}
%\providecommand{\res}[1]{\Res\displaylimits_{#1}} 
%\providecommand{\mtx}[1]{\mathbf{#1}}
%\providecommand{\fourier}{\overset{\mathcal{F}}{\rightleftharpoons}}
%\providecommand{\system}{\overset{\mathcal{H}}{\longleftrightarrow}}
	%\newcommand{\solution}[2]{\textbf{Solution:}{#1}}
%\newcommand{\solution}{\noindent \textbf{Solution: }}
%\newcommand{\cosec}{\,\text{cosec}\,}
%\providecommand{\dec}[2]{\ensuremath{\overset{#1}{\underset{#2}{\gtrless}}}}
%\newcommand{\myvec}[1]{\ensuremath{\begin{pmatrix}#1\end{pmatrix}}}
%\newcommand{\mydet}[1]{\ensuremath{\begin{vmatrix}#1\end{vmatrix}}}
%\let\vec\mathbf
%\begin{center}
%\title{\textbf{Straight Lines}}
%\date{\vspace{-5ex}} %Not to print date automatically
%\maketitle
%\end{center}
%\setcounter{page}{1}
%\section*{11$^{th}$ Maths - Chapter 10}
%This is Problem-10 from Exercise 10.4
%\begin{enumerate}
%    \item If three lines whose equations are $y=m_1x+c_1$, $y=m_2x+c_2$ and $y=m_3x+c_3$ are concurrent, then show that $m_1(c_2-c_3)+m_2(c_3-c_1)+m_3(c_1-c_2) = 0.$\\
%    \solution 
    Given lines can be written as \begin{align}
       m_1x-y+c_1=0
    \end{align}
    \begin{align}
        m_2x-y+c_2=0
    \end{align}
    \begin{align}
        m_3x-y+c_3=0
        \label{eq:3}
    \end{align}
    
    
   The above lines can be written in the form of \begin{align}
        \Vec{n}^{\top}\Vec{x} = c
    \end{align}
   Therefore,
		\begin{align}
       \myvec{m_1&-1}\vec{x}=c_1
       \label{eq:5}
   \end{align} 
   \begin{align}
       \myvec{m_2&-1}\vec{x}=c_2
       \label{eq:6}
   \end{align}
   \begin{align}
       \myvec{m_3&-1}\vec{x}=c_3
       \label{eq:7}
   \end{align}
   Solving equations \eqref{eq:5}, \eqref{eq:6}and \eqref{eq:7}
		augumented matrix is
 \begin{align}
    \myvec{m_1&-1&c_1\\m_2&-1&c_2\\m_3&-1&c_3}\\
    \xleftrightarrow{R_2 \leftarrow m_1R_2-m_2R_1}
    \myvec{m_1&-1&c_1\\0&m_2-m_1&m_1c_2-m_2c_1\\m_3&-1&c_3}\\
    \xleftrightarrow{R_3 \leftarrow m_1R_3-m_3R_1}
    \myvec{m_1&-1&c_1\\0&m_2-m_1&m_1c_2-m_2c_1\\0&m_3-m_1&m_1c_3-m_3c_1}\\
    \xleftrightarrow{R_3 \leftarrow R_3\frac{m_2-m_1}{m_3-m_1}-R_2}
        \myvec{m_1&-1&c_1\\0&m_2-m_1&m_1c_2-m_2c_1\\0&0&$\brak{m_1c_3-m_3c_1}$$\brak{\frac{m_2-m_1}{m_3-m_1}}$-$\brak{m_1c_2-m_2c_1}$}
\end{align}
Now, for lines to be concurrent, then the third row should be equal to zero. \\

Therefore,
\begin{align}
\brak{m_1c_3-m_3c_1}\brak{\frac{m_2-m_1}{m_3-m_1}}-\brak{m_1c_2-m_2c_1}=0\\
\frac{\brak{m_1c_3-m_3c_1}\brak{m_2-m_1}-\brak{m_1c_2-m_2c_1}\brak{m_3-m_1}}{m_3-m_1}=0\\
\brak{m_1c_3-m_3c_1}\brak{m_2-m_1}-\brak{m_1c_2-m_2c_1}\brak{m_3-m_1}=0\\
m_2c_3-m_1c_3+m_3c_1-m_3c_2+m_1c_2-m_2c_1=0\\
m_1\brak{c_2-c_3}+m_2\brak{c_3-c_1}+m_3\brak{c_1-c_2} = 0
\end{align}
           Hence proved
%\begin{figure}[h]
 %   \centering
  %  \includegraphics[width=\columnwidth]{concurrent-1.png}
   % \caption{Straight Lines}
    %\label{fig:concurrent-1.png}
%\end{figure}
%\end{enumerate}
%\end{document}

\item Find perpendicular distance from the origin to the line joining the points$(\cos\theta,\sin\theta)$ and $(\cos\phi,\sin\phi)$.
\\
\solution
		\input{chapters/11/10/4/5/dist.tex}
\item Find the equation of line which is equidistant from parallel lines $9x+6y-7=0$ and $3x+2y+6=0$.
\\
\solution
		%\documentclass[12pt]{article}
%\usepackage[cmex10]{amsmath}
%\usepackage{amsthm}
%\usepackage{mathrsfs}
%\usepackage{txfonts}
%\usepackage{stfloats}
%\usepackage{bm}
%\usepackage{cite}
%\usepackage{cases}
%\usepackage{subfig}
%\usepackage{longtable}
%\usepackage{multirow}
%\usepackage{enumitem}
%\usepackage{mathtools}
%\usepackage{steinmetz}
%\usepackage{tikz}
%\usepackage{circuitikz}
%\usepackage{verbatim}
%\usepackage{tfrupee}
%\usepackage[breaklinks=true]{hyperref}
%\usepackage{tkz-euclide} % loads  TikZ and tkz-base
%\providecommand{\brak}[1]{\ensuremath{\left(#1\right)}}
%\usepackage{atbegshi}
%\AtBeginDocument{\AtBeginShipoutNext{\AtBeginShipoutDiscard}}
%\usetikzlibrary{calc,math}
%\usepackage{listings}
%    \usepackage{color}                                            %%
%    \usepackage{array}                                            %%
 %   \usepackage{longtable}                                        %%
  %  \usepackage{calc}                                             %%
   % \usepackage{multirow}                                         %%
    %\usepackage{hhline}                                           %%
    %\usepackage{ifthen}                                           %%
  %optionally (for landscape tables embedded in another document): %%
    %\usepackage{lscape}     
%\usepackage{multicol}
%\usepackage{chngcntr}

%\DeclareMathOperator*{\Res}{Res}
%\renewcommand{\baselinestretch}{2}
%\renewcommand\thesection{\arabic{section}}
%\renewcommand\thesubsection{\thesection.\arabic{subsection}}
%\renewcommand\thesubsubsection{\thesubsection.\arabic{subsubsection}}


% correct bad hyphenation here
%\hyphenation{op-tical net-works semi-conduc-tor}
%\def\inputGnumericTable{}                                 %%

%\lstset{
%language=C,
%frame=single, 
%breaklines=true,
%columns=fullflexible
%}
%\begin{document}
%\newtheorem{theorem}{Theorem}[section]
%\newtheorem{problem}{Problem}
%\newtheorem{proposition}{Proposition}[section]
%\newtheorem{lemma}{Lemma}[section]
%\newtheorem{corollary}[theorem]{Corollary}
%\newtheorem{example}{Example}[section]
%\newtheorem{definition}[problem]{Definition}
%\newcommand{\BEQA}{\begin{eqnarray}}
%\newcommand{\EEQA}{\end{eqnarray}}
%\newcommand{\define}{\stackrel{\triangle}{=}}

%\bibliographystyle{IEEEtran}
%\bibliographystyle{ieeetr}
%\providecommand{\mbf}{\mathbf}
%\providecommand{\pr}[1]{\ensuremath{\Pr\left(#1\right)}}
%\providecommand{\qfunc}[1]{\ensuremath{Q\left(#1\right)}}
%\providecommand{\sbrak}[1]{\ensuremath{{}\left[#1\right]}}
%\providecommand{\lsbrak}[1]{\ensuremath{{}\left[#1\right.}}
%\providecommand{\rsbrak}[1]{\ensuremath{{}\left.#1\right]}}
%\providecommand{\brak}[1]{\ensuremath{\left(#1\right)}}
%\providecommand{\lbrak}[1]{\ensuremath{\left(#1\right.}}
%\providecommand{\rbrak}[1]{\ensuremath{\left.#1\right)}}
%\providecommand{\cbrak}[1]{\ensuremath{\left\{#1\right\}}}
%\providecommand{\lcbrak}[1]{\ensuremath{\left\{#1\right.}}
%\providecommand{\rcbrak}[1]{\ensuremath{\left.#1\right\}}}
%\theoremstyle{remark}
%\newtheorem{rem}{Remark}
%\newcommand{\sgn}{\mathop{\mathrm{sgn}}}
%\providecommand{\res}[1]{\Res\displaylimits_{#1}} 
%\providecommand{\mtx}[1]{\mathbf{#1}}
%\providecommand{\fourier}{\overset{\mathcal{F}}{\rightleftharpoons}}
%\providecommand{\system}{\overset{\mathcal{H}}{\longleftrightarrow}}
	%\newcommand{\solution}[2]{\textbf{Solution:}{#1}}
%\newcommand{\solution}{\noindent \textbf{Solution: }}
%\newcommand{\cosec}{\,\text{cosec}\,}
%\providecommand{\dec}[2]{\ensuremath{\overset{#1}{\underset{#2}{\gtrless}}}}
%\newcommand{\myvec}[1]{\ensuremath{\begin{pmatrix}#1\end{pmatrix}}}
%\newcommand{\mydet}[1]{\ensuremath{\begin{vmatrix}#1\end{vmatrix}}}
%\let\vec\mathbf
%\begin{center}
%\title{\textbf{Straight Lines}}
%\date{\vspace{-5ex}} %Not to print date automatically
%\maketitle
%\end{center}
%\setcounter{page}{1}
%\section*{11$^{th}$ Maths - Chapter 10}
%This is Problem-10 from Exercise 10.4
%\begin{enumerate}
%    \item If three lines whose equations are $y=m_1x+c_1$, $y=m_2x+c_2$ and $y=m_3x+c_3$ are concurrent, then show that $m_1(c_2-c_3)+m_2(c_3-c_1)+m_3(c_1-c_2) = 0.$\\
%    \solution 
    Given lines can be written as \begin{align}
       m_1x-y+c_1=0
    \end{align}
    \begin{align}
        m_2x-y+c_2=0
    \end{align}
    \begin{align}
        m_3x-y+c_3=0
        \label{eq:3}
    \end{align}
    
    
   The above lines can be written in the form of \begin{align}
        \Vec{n}^{\top}\Vec{x} = c
    \end{align}
   Therefore,
		\begin{align}
       \myvec{m_1&-1}\vec{x}=c_1
       \label{eq:5}
   \end{align} 
   \begin{align}
       \myvec{m_2&-1}\vec{x}=c_2
       \label{eq:6}
   \end{align}
   \begin{align}
       \myvec{m_3&-1}\vec{x}=c_3
       \label{eq:7}
   \end{align}
   Solving equations \eqref{eq:5}, \eqref{eq:6}and \eqref{eq:7}
		augumented matrix is
 \begin{align}
    \myvec{m_1&-1&c_1\\m_2&-1&c_2\\m_3&-1&c_3}\\
    \xleftrightarrow{R_2 \leftarrow m_1R_2-m_2R_1}
    \myvec{m_1&-1&c_1\\0&m_2-m_1&m_1c_2-m_2c_1\\m_3&-1&c_3}\\
    \xleftrightarrow{R_3 \leftarrow m_1R_3-m_3R_1}
    \myvec{m_1&-1&c_1\\0&m_2-m_1&m_1c_2-m_2c_1\\0&m_3-m_1&m_1c_3-m_3c_1}\\
    \xleftrightarrow{R_3 \leftarrow R_3\frac{m_2-m_1}{m_3-m_1}-R_2}
        \myvec{m_1&-1&c_1\\0&m_2-m_1&m_1c_2-m_2c_1\\0&0&$\brak{m_1c_3-m_3c_1}$$\brak{\frac{m_2-m_1}{m_3-m_1}}$-$\brak{m_1c_2-m_2c_1}$}
\end{align}
Now, for lines to be concurrent, then the third row should be equal to zero. \\

Therefore,
\begin{align}
\brak{m_1c_3-m_3c_1}\brak{\frac{m_2-m_1}{m_3-m_1}}-\brak{m_1c_2-m_2c_1}=0\\
\frac{\brak{m_1c_3-m_3c_1}\brak{m_2-m_1}-\brak{m_1c_2-m_2c_1}\brak{m_3-m_1}}{m_3-m_1}=0\\
\brak{m_1c_3-m_3c_1}\brak{m_2-m_1}-\brak{m_1c_2-m_2c_1}\brak{m_3-m_1}=0\\
m_2c_3-m_1c_3+m_3c_1-m_3c_2+m_1c_2-m_2c_1=0\\
m_1\brak{c_2-c_3}+m_2\brak{c_3-c_1}+m_3\brak{c_1-c_2} = 0
\end{align}
           Hence proved
%\begin{figure}[h]
 %   \centering
  %  \includegraphics[width=\columnwidth]{concurrent-1.png}
   % \caption{Straight Lines}
    %\label{fig:concurrent-1.png}
%\end{figure}
%\end{enumerate}
%\end{document}

	\item Prove that the products of the lengths of the perpendiculars drawn from the points $\myvec{\sqrt{a^2-b^2}\\0}$ and $\myvec{-\sqrt{a^2-b^2} \\0} $ to the line $\frac{x}{a} \cos{\theta} + \frac{y}{b}\sin{\theta} =1 $ is $ b^2 $.
\\
    \solution 
		\input{chapters/11/10/4/23/dist.tex}
\item Find the equation of line  drawn perpendicular to the line $\frac{x}{4}+\frac{y}{6}=1$ through the point where it meets the y-axis \\
\solution
		%\documentclass[12pt]{article}
%\usepackage[cmex10]{amsmath}
%\usepackage{amsthm}
%\usepackage{mathrsfs}
%\usepackage{txfonts}
%\usepackage{stfloats}
%\usepackage{bm}
%\usepackage{cite}
%\usepackage{cases}
%\usepackage{subfig}
%\usepackage{longtable}
%\usepackage{multirow}
%\usepackage{enumitem}
%\usepackage{mathtools}
%\usepackage{steinmetz}
%\usepackage{tikz}
%\usepackage{circuitikz}
%\usepackage{verbatim}
%\usepackage{tfrupee}
%\usepackage[breaklinks=true]{hyperref}
%\usepackage{tkz-euclide} % loads  TikZ and tkz-base
%\providecommand{\brak}[1]{\ensuremath{\left(#1\right)}}
%\usepackage{atbegshi}
%\AtBeginDocument{\AtBeginShipoutNext{\AtBeginShipoutDiscard}}
%\usetikzlibrary{calc,math}
%\usepackage{listings}
%    \usepackage{color}                                            %%
%    \usepackage{array}                                            %%
 %   \usepackage{longtable}                                        %%
  %  \usepackage{calc}                                             %%
   % \usepackage{multirow}                                         %%
    %\usepackage{hhline}                                           %%
    %\usepackage{ifthen}                                           %%
  %optionally (for landscape tables embedded in another document): %%
    %\usepackage{lscape}     
%\usepackage{multicol}
%\usepackage{chngcntr}

%\DeclareMathOperator*{\Res}{Res}
%\renewcommand{\baselinestretch}{2}
%\renewcommand\thesection{\arabic{section}}
%\renewcommand\thesubsection{\thesection.\arabic{subsection}}
%\renewcommand\thesubsubsection{\thesubsection.\arabic{subsubsection}}


% correct bad hyphenation here
%\hyphenation{op-tical net-works semi-conduc-tor}
%\def\inputGnumericTable{}                                 %%

%\lstset{
%language=C,
%frame=single, 
%breaklines=true,
%columns=fullflexible
%}
%\begin{document}
%\newtheorem{theorem}{Theorem}[section]
%\newtheorem{problem}{Problem}
%\newtheorem{proposition}{Proposition}[section]
%\newtheorem{lemma}{Lemma}[section]
%\newtheorem{corollary}[theorem]{Corollary}
%\newtheorem{example}{Example}[section]
%\newtheorem{definition}[problem]{Definition}
%\newcommand{\BEQA}{\begin{eqnarray}}
%\newcommand{\EEQA}{\end{eqnarray}}
%\newcommand{\define}{\stackrel{\triangle}{=}}

%\bibliographystyle{IEEEtran}
%\bibliographystyle{ieeetr}
%\providecommand{\mbf}{\mathbf}
%\providecommand{\pr}[1]{\ensuremath{\Pr\left(#1\right)}}
%\providecommand{\qfunc}[1]{\ensuremath{Q\left(#1\right)}}
%\providecommand{\sbrak}[1]{\ensuremath{{}\left[#1\right]}}
%\providecommand{\lsbrak}[1]{\ensuremath{{}\left[#1\right.}}
%\providecommand{\rsbrak}[1]{\ensuremath{{}\left.#1\right]}}
%\providecommand{\brak}[1]{\ensuremath{\left(#1\right)}}
%\providecommand{\lbrak}[1]{\ensuremath{\left(#1\right.}}
%\providecommand{\rbrak}[1]{\ensuremath{\left.#1\right)}}
%\providecommand{\cbrak}[1]{\ensuremath{\left\{#1\right\}}}
%\providecommand{\lcbrak}[1]{\ensuremath{\left\{#1\right.}}
%\providecommand{\rcbrak}[1]{\ensuremath{\left.#1\right\}}}
%\theoremstyle{remark}
%\newtheorem{rem}{Remark}
%\newcommand{\sgn}{\mathop{\mathrm{sgn}}}
%\providecommand{\res}[1]{\Res\displaylimits_{#1}} 
%\providecommand{\mtx}[1]{\mathbf{#1}}
%\providecommand{\fourier}{\overset{\mathcal{F}}{\rightleftharpoons}}
%\providecommand{\system}{\overset{\mathcal{H}}{\longleftrightarrow}}
	%\newcommand{\solution}[2]{\textbf{Solution:}{#1}}
%\newcommand{\solution}{\noindent \textbf{Solution: }}
%\newcommand{\cosec}{\,\text{cosec}\,}
%\providecommand{\dec}[2]{\ensuremath{\overset{#1}{\underset{#2}{\gtrless}}}}
%\newcommand{\myvec}[1]{\ensuremath{\begin{pmatrix}#1\end{pmatrix}}}
%\newcommand{\mydet}[1]{\ensuremath{\begin{vmatrix}#1\end{vmatrix}}}
%\let\vec\mathbf
%\begin{center}
%\title{\textbf{Straight Lines}}
%\date{\vspace{-5ex}} %Not to print date automatically
%\maketitle
%\end{center}
%\setcounter{page}{1}
%\section*{11$^{th}$ Maths - Chapter 10}
%This is Problem-10 from Exercise 10.4
%\begin{enumerate}
%    \item If three lines whose equations are $y=m_1x+c_1$, $y=m_2x+c_2$ and $y=m_3x+c_3$ are concurrent, then show that $m_1(c_2-c_3)+m_2(c_3-c_1)+m_3(c_1-c_2) = 0.$\\
%    \solution 
    Given lines can be written as \begin{align}
       m_1x-y+c_1=0
    \end{align}
    \begin{align}
        m_2x-y+c_2=0
    \end{align}
    \begin{align}
        m_3x-y+c_3=0
        \label{eq:3}
    \end{align}
    
    
   The above lines can be written in the form of \begin{align}
        \Vec{n}^{\top}\Vec{x} = c
    \end{align}
   Therefore,
		\begin{align}
       \myvec{m_1&-1}\vec{x}=c_1
       \label{eq:5}
   \end{align} 
   \begin{align}
       \myvec{m_2&-1}\vec{x}=c_2
       \label{eq:6}
   \end{align}
   \begin{align}
       \myvec{m_3&-1}\vec{x}=c_3
       \label{eq:7}
   \end{align}
   Solving equations \eqref{eq:5}, \eqref{eq:6}and \eqref{eq:7}
		augumented matrix is
 \begin{align}
    \myvec{m_1&-1&c_1\\m_2&-1&c_2\\m_3&-1&c_3}\\
    \xleftrightarrow{R_2 \leftarrow m_1R_2-m_2R_1}
    \myvec{m_1&-1&c_1\\0&m_2-m_1&m_1c_2-m_2c_1\\m_3&-1&c_3}\\
    \xleftrightarrow{R_3 \leftarrow m_1R_3-m_3R_1}
    \myvec{m_1&-1&c_1\\0&m_2-m_1&m_1c_2-m_2c_1\\0&m_3-m_1&m_1c_3-m_3c_1}\\
    \xleftrightarrow{R_3 \leftarrow R_3\frac{m_2-m_1}{m_3-m_1}-R_2}
        \myvec{m_1&-1&c_1\\0&m_2-m_1&m_1c_2-m_2c_1\\0&0&$\brak{m_1c_3-m_3c_1}$$\brak{\frac{m_2-m_1}{m_3-m_1}}$-$\brak{m_1c_2-m_2c_1}$}
\end{align}
Now, for lines to be concurrent, then the third row should be equal to zero. \\

Therefore,
\begin{align}
\brak{m_1c_3-m_3c_1}\brak{\frac{m_2-m_1}{m_3-m_1}}-\brak{m_1c_2-m_2c_1}=0\\
\frac{\brak{m_1c_3-m_3c_1}\brak{m_2-m_1}-\brak{m_1c_2-m_2c_1}\brak{m_3-m_1}}{m_3-m_1}=0\\
\brak{m_1c_3-m_3c_1}\brak{m_2-m_1}-\brak{m_1c_2-m_2c_1}\brak{m_3-m_1}=0\\
m_2c_3-m_1c_3+m_3c_1-m_3c_2+m_1c_2-m_2c_1=0\\
m_1\brak{c_2-c_3}+m_2\brak{c_3-c_1}+m_3\brak{c_1-c_2} = 0
\end{align}
           Hence proved
%\begin{figure}[h]
 %   \centering
  %  \includegraphics[width=\columnwidth]{concurrent-1.png}
   % \caption{Straight Lines}
    %\label{fig:concurrent-1.png}
%\end{figure}
%\end{enumerate}
%\end{document}

 \item  In each of the following cases, determine the direction cosines of the normal to
the plane and the distance from the origin.
\begin{enumerate}
	\item $z=2$ 
	\item $x + y + z = 1$
	\item $2x + 3y – z = 5$
	\item $5y + 8 = 0$
\end{enumerate}
    \solution
		\input{chapters/12/11/3/1/dist.tex}
\item
Find the angle between the lines whose direction ratios are $a,b,c$ and $b-c,c-a,a-b$.

\textbf{Solution :}
    \begin{align}
    \vec{m _1} &= \myvec{a\\b\\c}\\
    \vec{m_2} &= \myvec{b-c\\c-a\\a-b}\\
    \cos{\theta}&= \frac{\vec{m_1}^{\top}\vec{m_2}}{\vec{\norm{m_1}\norm{m_2}}
   } \\
   &=\frac{\myvec{a&b&c}\myvec{b-c\\c-a\\a-b}}{\sqrt{a^2+b^2+c^2}\sqrt{\brak{b-c}^2+\brak{c-a}^2+\brak{a-b}^2}}\\
   &=0\\
   or,\theta&=\frac{\pi}{2}
    \end{align}

\end{enumerate}

\item Find the distance of the point $(-1,1)$ from the line $12\brak{x+6} = 5\brak{y-2}$. 
\label{chapters/11/10/3/4}
%\documentclass[12pt]{article}
%\usepackage[cmex10]{amsmath}
%\usepackage{amsthm}
%\usepackage{mathrsfs}
%\usepackage{txfonts}
%\usepackage{stfloats}
%\usepackage{bm}
%\usepackage{cite}
%\usepackage{cases}
%\usepackage{subfig}
%\usepackage{longtable}
%\usepackage{multirow}
%\usepackage{enumitem}
%\usepackage{mathtools}
%\usepackage{steinmetz}
%\usepackage{tikz}
%\usepackage{circuitikz}
%\usepackage{verbatim}
%\usepackage{tfrupee}
%\usepackage[breaklinks=true]{hyperref}
%\usepackage{tkz-euclide} % loads  TikZ and tkz-base
%\providecommand{\brak}[1]{\ensuremath{\left(#1\right)}}
%\usepackage{atbegshi}
%\AtBeginDocument{\AtBeginShipoutNext{\AtBeginShipoutDiscard}}
%\usetikzlibrary{calc,math}
%\usepackage{listings}
%    \usepackage{color}                                            %%
%    \usepackage{array}                                            %%
 %   \usepackage{longtable}                                        %%
  %  \usepackage{calc}                                             %%
   % \usepackage{multirow}                                         %%
    %\usepackage{hhline}                                           %%
    %\usepackage{ifthen}                                           %%
  %optionally (for landscape tables embedded in another document): %%
    %\usepackage{lscape}     
%\usepackage{multicol}
%\usepackage{chngcntr}

%\DeclareMathOperator*{\Res}{Res}
%\renewcommand{\baselinestretch}{2}
%\renewcommand\thesection{\arabic{section}}
%\renewcommand\thesubsection{\thesection.\arabic{subsection}}
%\renewcommand\thesubsubsection{\thesubsection.\arabic{subsubsection}}


% correct bad hyphenation here
%\hyphenation{op-tical net-works semi-conduc-tor}
%\def\inputGnumericTable{}                                 %%

%\lstset{
%language=C,
%frame=single, 
%breaklines=true,
%columns=fullflexible
%}
%\begin{document}
%\newtheorem{theorem}{Theorem}[section]
%\newtheorem{problem}{Problem}
%\newtheorem{proposition}{Proposition}[section]
%\newtheorem{lemma}{Lemma}[section]
%\newtheorem{corollary}[theorem]{Corollary}
%\newtheorem{example}{Example}[section]
%\newtheorem{definition}[problem]{Definition}
%\newcommand{\BEQA}{\begin{eqnarray}}
%\newcommand{\EEQA}{\end{eqnarray}}
%\newcommand{\define}{\stackrel{\triangle}{=}}

%\bibliographystyle{IEEEtran}
%\bibliographystyle{ieeetr}
%\providecommand{\mbf}{\mathbf}
%\providecommand{\pr}[1]{\ensuremath{\Pr\left(#1\right)}}
%\providecommand{\qfunc}[1]{\ensuremath{Q\left(#1\right)}}
%\providecommand{\sbrak}[1]{\ensuremath{{}\left[#1\right]}}
%\providecommand{\lsbrak}[1]{\ensuremath{{}\left[#1\right.}}
%\providecommand{\rsbrak}[1]{\ensuremath{{}\left.#1\right]}}
%\providecommand{\brak}[1]{\ensuremath{\left(#1\right)}}
%\providecommand{\lbrak}[1]{\ensuremath{\left(#1\right.}}
%\providecommand{\rbrak}[1]{\ensuremath{\left.#1\right)}}
%\providecommand{\cbrak}[1]{\ensuremath{\left\{#1\right\}}}
%\providecommand{\lcbrak}[1]{\ensuremath{\left\{#1\right.}}
%\providecommand{\rcbrak}[1]{\ensuremath{\left.#1\right\}}}
%\theoremstyle{remark}
%\newtheorem{rem}{Remark}
%\newcommand{\sgn}{\mathop{\mathrm{sgn}}}
%\providecommand{\res}[1]{\Res\displaylimits_{#1}} 
%\providecommand{\mtx}[1]{\mathbf{#1}}
%\providecommand{\fourier}{\overset{\mathcal{F}}{\rightleftharpoons}}
%\providecommand{\system}{\overset{\mathcal{H}}{\longleftrightarrow}}
	%\newcommand{\solution}[2]{\textbf{Solution:}{#1}}
%\newcommand{\solution}{\noindent \textbf{Solution: }}
%\newcommand{\cosec}{\,\text{cosec}\,}
%\providecommand{\dec}[2]{\ensuremath{\overset{#1}{\underset{#2}{\gtrless}}}}
%\newcommand{\myvec}[1]{\ensuremath{\begin{pmatrix}#1\end{pmatrix}}}
%\newcommand{\mydet}[1]{\ensuremath{\begin{vmatrix}#1\end{vmatrix}}}
%\let\vec\mathbf
%\begin{center}
%\title{\textbf{Straight Lines}}
%\date{\vspace{-5ex}} %Not to print date automatically
%\maketitle
%\end{center}
%\setcounter{page}{1}
%\section*{11$^{th}$ Maths - Chapter 10}
%This is Problem-10 from Exercise 10.4
%\begin{enumerate}
%    \item If three lines whose equations are $y=m_1x+c_1$, $y=m_2x+c_2$ and $y=m_3x+c_3$ are concurrent, then show that $m_1(c_2-c_3)+m_2(c_3-c_1)+m_3(c_1-c_2) = 0.$\\
%    \solution 
    Given lines can be written as \begin{align}
       m_1x-y+c_1=0
    \end{align}
    \begin{align}
        m_2x-y+c_2=0
    \end{align}
    \begin{align}
        m_3x-y+c_3=0
        \label{eq:3}
    \end{align}
    
    
   The above lines can be written in the form of \begin{align}
        \Vec{n}^{\top}\Vec{x} = c
    \end{align}
   Therefore,
		\begin{align}
       \myvec{m_1&-1}\vec{x}=c_1
       \label{eq:5}
   \end{align} 
   \begin{align}
       \myvec{m_2&-1}\vec{x}=c_2
       \label{eq:6}
   \end{align}
   \begin{align}
       \myvec{m_3&-1}\vec{x}=c_3
       \label{eq:7}
   \end{align}
   Solving equations \eqref{eq:5}, \eqref{eq:6}and \eqref{eq:7}
		augumented matrix is
 \begin{align}
    \myvec{m_1&-1&c_1\\m_2&-1&c_2\\m_3&-1&c_3}\\
    \xleftrightarrow{R_2 \leftarrow m_1R_2-m_2R_1}
    \myvec{m_1&-1&c_1\\0&m_2-m_1&m_1c_2-m_2c_1\\m_3&-1&c_3}\\
    \xleftrightarrow{R_3 \leftarrow m_1R_3-m_3R_1}
    \myvec{m_1&-1&c_1\\0&m_2-m_1&m_1c_2-m_2c_1\\0&m_3-m_1&m_1c_3-m_3c_1}\\
    \xleftrightarrow{R_3 \leftarrow R_3\frac{m_2-m_1}{m_3-m_1}-R_2}
        \myvec{m_1&-1&c_1\\0&m_2-m_1&m_1c_2-m_2c_1\\0&0&$\brak{m_1c_3-m_3c_1}$$\brak{\frac{m_2-m_1}{m_3-m_1}}$-$\brak{m_1c_2-m_2c_1}$}
\end{align}
Now, for lines to be concurrent, then the third row should be equal to zero. \\

Therefore,
\begin{align}
\brak{m_1c_3-m_3c_1}\brak{\frac{m_2-m_1}{m_3-m_1}}-\brak{m_1c_2-m_2c_1}=0\\
\frac{\brak{m_1c_3-m_3c_1}\brak{m_2-m_1}-\brak{m_1c_2-m_2c_1}\brak{m_3-m_1}}{m_3-m_1}=0\\
\brak{m_1c_3-m_3c_1}\brak{m_2-m_1}-\brak{m_1c_2-m_2c_1}\brak{m_3-m_1}=0\\
m_2c_3-m_1c_3+m_3c_1-m_3c_2+m_1c_2-m_2c_1=0\\
m_1\brak{c_2-c_3}+m_2\brak{c_3-c_1}+m_3\brak{c_1-c_2} = 0
\end{align}
           Hence proved
%\begin{figure}[h]
 %   \centering
  %  \includegraphics[width=\columnwidth]{concurrent-1.png}
   % \caption{Straight Lines}
    %\label{fig:concurrent-1.png}
%\end{figure}
%\end{enumerate}
%\end{document}

\item Find the points on the x-axis, whose distances from the line $\frac{x}{3}+\frac{y}{4}=1$ are 4 units.
\label{chapters/11/10/3/5}
	\\
	\solution
%\documentclass[12pt]{article}
%\usepackage[cmex10]{amsmath}
%\usepackage{amsthm}
%\usepackage{mathrsfs}
%\usepackage{txfonts}
%\usepackage{stfloats}
%\usepackage{bm}
%\usepackage{cite}
%\usepackage{cases}
%\usepackage{subfig}
%\usepackage{longtable}
%\usepackage{multirow}
%\usepackage{enumitem}
%\usepackage{mathtools}
%\usepackage{steinmetz}
%\usepackage{tikz}
%\usepackage{circuitikz}
%\usepackage{verbatim}
%\usepackage{tfrupee}
%\usepackage[breaklinks=true]{hyperref}
%\usepackage{tkz-euclide} % loads  TikZ and tkz-base
%\providecommand{\brak}[1]{\ensuremath{\left(#1\right)}}
%\usepackage{atbegshi}
%\AtBeginDocument{\AtBeginShipoutNext{\AtBeginShipoutDiscard}}
%\usetikzlibrary{calc,math}
%\usepackage{listings}
%    \usepackage{color}                                            %%
%    \usepackage{array}                                            %%
 %   \usepackage{longtable}                                        %%
  %  \usepackage{calc}                                             %%
   % \usepackage{multirow}                                         %%
    %\usepackage{hhline}                                           %%
    %\usepackage{ifthen}                                           %%
  %optionally (for landscape tables embedded in another document): %%
    %\usepackage{lscape}     
%\usepackage{multicol}
%\usepackage{chngcntr}

%\DeclareMathOperator*{\Res}{Res}
%\renewcommand{\baselinestretch}{2}
%\renewcommand\thesection{\arabic{section}}
%\renewcommand\thesubsection{\thesection.\arabic{subsection}}
%\renewcommand\thesubsubsection{\thesubsection.\arabic{subsubsection}}


% correct bad hyphenation here
%\hyphenation{op-tical net-works semi-conduc-tor}
%\def\inputGnumericTable{}                                 %%

%\lstset{
%language=C,
%frame=single, 
%breaklines=true,
%columns=fullflexible
%}
%\begin{document}
%\newtheorem{theorem}{Theorem}[section]
%\newtheorem{problem}{Problem}
%\newtheorem{proposition}{Proposition}[section]
%\newtheorem{lemma}{Lemma}[section]
%\newtheorem{corollary}[theorem]{Corollary}
%\newtheorem{example}{Example}[section]
%\newtheorem{definition}[problem]{Definition}
%\newcommand{\BEQA}{\begin{eqnarray}}
%\newcommand{\EEQA}{\end{eqnarray}}
%\newcommand{\define}{\stackrel{\triangle}{=}}

%\bibliographystyle{IEEEtran}
%\bibliographystyle{ieeetr}
%\providecommand{\mbf}{\mathbf}
%\providecommand{\pr}[1]{\ensuremath{\Pr\left(#1\right)}}
%\providecommand{\qfunc}[1]{\ensuremath{Q\left(#1\right)}}
%\providecommand{\sbrak}[1]{\ensuremath{{}\left[#1\right]}}
%\providecommand{\lsbrak}[1]{\ensuremath{{}\left[#1\right.}}
%\providecommand{\rsbrak}[1]{\ensuremath{{}\left.#1\right]}}
%\providecommand{\brak}[1]{\ensuremath{\left(#1\right)}}
%\providecommand{\lbrak}[1]{\ensuremath{\left(#1\right.}}
%\providecommand{\rbrak}[1]{\ensuremath{\left.#1\right)}}
%\providecommand{\cbrak}[1]{\ensuremath{\left\{#1\right\}}}
%\providecommand{\lcbrak}[1]{\ensuremath{\left\{#1\right.}}
%\providecommand{\rcbrak}[1]{\ensuremath{\left.#1\right\}}}
%\theoremstyle{remark}
%\newtheorem{rem}{Remark}
%\newcommand{\sgn}{\mathop{\mathrm{sgn}}}
%\providecommand{\res}[1]{\Res\displaylimits_{#1}} 
%\providecommand{\mtx}[1]{\mathbf{#1}}
%\providecommand{\fourier}{\overset{\mathcal{F}}{\rightleftharpoons}}
%\providecommand{\system}{\overset{\mathcal{H}}{\longleftrightarrow}}
	%\newcommand{\solution}[2]{\textbf{Solution:}{#1}}
%\newcommand{\solution}{\noindent \textbf{Solution: }}
%\newcommand{\cosec}{\,\text{cosec}\,}
%\providecommand{\dec}[2]{\ensuremath{\overset{#1}{\underset{#2}{\gtrless}}}}
%\newcommand{\myvec}[1]{\ensuremath{\begin{pmatrix}#1\end{pmatrix}}}
%\newcommand{\mydet}[1]{\ensuremath{\begin{vmatrix}#1\end{vmatrix}}}
%\let\vec\mathbf
%\begin{center}
%\title{\textbf{Straight Lines}}
%\date{\vspace{-5ex}} %Not to print date automatically
%\maketitle
%\end{center}
%\setcounter{page}{1}
%\section*{11$^{th}$ Maths - Chapter 10}
%This is Problem-10 from Exercise 10.4
%\begin{enumerate}
%    \item If three lines whose equations are $y=m_1x+c_1$, $y=m_2x+c_2$ and $y=m_3x+c_3$ are concurrent, then show that $m_1(c_2-c_3)+m_2(c_3-c_1)+m_3(c_1-c_2) = 0.$\\
%    \solution 
    Given lines can be written as \begin{align}
       m_1x-y+c_1=0
    \end{align}
    \begin{align}
        m_2x-y+c_2=0
    \end{align}
    \begin{align}
        m_3x-y+c_3=0
        \label{eq:3}
    \end{align}
    
    
   The above lines can be written in the form of \begin{align}
        \Vec{n}^{\top}\Vec{x} = c
    \end{align}
   Therefore,
		\begin{align}
       \myvec{m_1&-1}\vec{x}=c_1
       \label{eq:5}
   \end{align} 
   \begin{align}
       \myvec{m_2&-1}\vec{x}=c_2
       \label{eq:6}
   \end{align}
   \begin{align}
       \myvec{m_3&-1}\vec{x}=c_3
       \label{eq:7}
   \end{align}
   Solving equations \eqref{eq:5}, \eqref{eq:6}and \eqref{eq:7}
		augumented matrix is
 \begin{align}
    \myvec{m_1&-1&c_1\\m_2&-1&c_2\\m_3&-1&c_3}\\
    \xleftrightarrow{R_2 \leftarrow m_1R_2-m_2R_1}
    \myvec{m_1&-1&c_1\\0&m_2-m_1&m_1c_2-m_2c_1\\m_3&-1&c_3}\\
    \xleftrightarrow{R_3 \leftarrow m_1R_3-m_3R_1}
    \myvec{m_1&-1&c_1\\0&m_2-m_1&m_1c_2-m_2c_1\\0&m_3-m_1&m_1c_3-m_3c_1}\\
    \xleftrightarrow{R_3 \leftarrow R_3\frac{m_2-m_1}{m_3-m_1}-R_2}
        \myvec{m_1&-1&c_1\\0&m_2-m_1&m_1c_2-m_2c_1\\0&0&$\brak{m_1c_3-m_3c_1}$$\brak{\frac{m_2-m_1}{m_3-m_1}}$-$\brak{m_1c_2-m_2c_1}$}
\end{align}
Now, for lines to be concurrent, then the third row should be equal to zero. \\

Therefore,
\begin{align}
\brak{m_1c_3-m_3c_1}\brak{\frac{m_2-m_1}{m_3-m_1}}-\brak{m_1c_2-m_2c_1}=0\\
\frac{\brak{m_1c_3-m_3c_1}\brak{m_2-m_1}-\brak{m_1c_2-m_2c_1}\brak{m_3-m_1}}{m_3-m_1}=0\\
\brak{m_1c_3-m_3c_1}\brak{m_2-m_1}-\brak{m_1c_2-m_2c_1}\brak{m_3-m_1}=0\\
m_2c_3-m_1c_3+m_3c_1-m_3c_2+m_1c_2-m_2c_1=0\\
m_1\brak{c_2-c_3}+m_2\brak{c_3-c_1}+m_3\brak{c_1-c_2} = 0
\end{align}
           Hence proved
%\begin{figure}[h]
 %   \centering
  %  \includegraphics[width=\columnwidth]{concurrent-1.png}
   % \caption{Straight Lines}
    %\label{fig:concurrent-1.png}
%\end{figure}
%\end{enumerate}
%\end{document}

\item Find the distance between parallel lines
\label{chapters/11/10/3/6}
\begin{enumerate}
	\item $15x+8y-34=0$ and  $15x+8y+31=0$ \\
	\item  $l(x+y)+p=0$ and  $l(x+y)-r=0$
\end{enumerate}
	\solution
\input{chapters/11/10/3/6/dist.tex}
\item Find the coordinates of the foot of the perpendicular from $(-1, 3)$ to the line $3x-4y-16=0$.  
\label{chapters/11/10/3/14}
\\
\solution
%\documentclass[12pt]{article}
%\usepackage[cmex10]{amsmath}
%\usepackage{amsthm}
%\usepackage{mathrsfs}
%\usepackage{txfonts}
%\usepackage{stfloats}
%\usepackage{bm}
%\usepackage{cite}
%\usepackage{cases}
%\usepackage{subfig}
%\usepackage{longtable}
%\usepackage{multirow}
%\usepackage{enumitem}
%\usepackage{mathtools}
%\usepackage{steinmetz}
%\usepackage{tikz}
%\usepackage{circuitikz}
%\usepackage{verbatim}
%\usepackage{tfrupee}
%\usepackage[breaklinks=true]{hyperref}
%\usepackage{tkz-euclide} % loads  TikZ and tkz-base
%\providecommand{\brak}[1]{\ensuremath{\left(#1\right)}}
%\usepackage{atbegshi}
%\AtBeginDocument{\AtBeginShipoutNext{\AtBeginShipoutDiscard}}
%\usetikzlibrary{calc,math}
%\usepackage{listings}
%    \usepackage{color}                                            %%
%    \usepackage{array}                                            %%
 %   \usepackage{longtable}                                        %%
  %  \usepackage{calc}                                             %%
   % \usepackage{multirow}                                         %%
    %\usepackage{hhline}                                           %%
    %\usepackage{ifthen}                                           %%
  %optionally (for landscape tables embedded in another document): %%
    %\usepackage{lscape}     
%\usepackage{multicol}
%\usepackage{chngcntr}

%\DeclareMathOperator*{\Res}{Res}
%\renewcommand{\baselinestretch}{2}
%\renewcommand\thesection{\arabic{section}}
%\renewcommand\thesubsection{\thesection.\arabic{subsection}}
%\renewcommand\thesubsubsection{\thesubsection.\arabic{subsubsection}}


% correct bad hyphenation here
%\hyphenation{op-tical net-works semi-conduc-tor}
%\def\inputGnumericTable{}                                 %%

%\lstset{
%language=C,
%frame=single, 
%breaklines=true,
%columns=fullflexible
%}
%\begin{document}
%\newtheorem{theorem}{Theorem}[section]
%\newtheorem{problem}{Problem}
%\newtheorem{proposition}{Proposition}[section]
%\newtheorem{lemma}{Lemma}[section]
%\newtheorem{corollary}[theorem]{Corollary}
%\newtheorem{example}{Example}[section]
%\newtheorem{definition}[problem]{Definition}
%\newcommand{\BEQA}{\begin{eqnarray}}
%\newcommand{\EEQA}{\end{eqnarray}}
%\newcommand{\define}{\stackrel{\triangle}{=}}

%\bibliographystyle{IEEEtran}
%\bibliographystyle{ieeetr}
%\providecommand{\mbf}{\mathbf}
%\providecommand{\pr}[1]{\ensuremath{\Pr\left(#1\right)}}
%\providecommand{\qfunc}[1]{\ensuremath{Q\left(#1\right)}}
%\providecommand{\sbrak}[1]{\ensuremath{{}\left[#1\right]}}
%\providecommand{\lsbrak}[1]{\ensuremath{{}\left[#1\right.}}
%\providecommand{\rsbrak}[1]{\ensuremath{{}\left.#1\right]}}
%\providecommand{\brak}[1]{\ensuremath{\left(#1\right)}}
%\providecommand{\lbrak}[1]{\ensuremath{\left(#1\right.}}
%\providecommand{\rbrak}[1]{\ensuremath{\left.#1\right)}}
%\providecommand{\cbrak}[1]{\ensuremath{\left\{#1\right\}}}
%\providecommand{\lcbrak}[1]{\ensuremath{\left\{#1\right.}}
%\providecommand{\rcbrak}[1]{\ensuremath{\left.#1\right\}}}
%\theoremstyle{remark}
%\newtheorem{rem}{Remark}
%\newcommand{\sgn}{\mathop{\mathrm{sgn}}}
%\providecommand{\res}[1]{\Res\displaylimits_{#1}} 
%\providecommand{\mtx}[1]{\mathbf{#1}}
%\providecommand{\fourier}{\overset{\mathcal{F}}{\rightleftharpoons}}
%\providecommand{\system}{\overset{\mathcal{H}}{\longleftrightarrow}}
	%\newcommand{\solution}[2]{\textbf{Solution:}{#1}}
%\newcommand{\solution}{\noindent \textbf{Solution: }}
%\newcommand{\cosec}{\,\text{cosec}\,}
%\providecommand{\dec}[2]{\ensuremath{\overset{#1}{\underset{#2}{\gtrless}}}}
%\newcommand{\myvec}[1]{\ensuremath{\begin{pmatrix}#1\end{pmatrix}}}
%\newcommand{\mydet}[1]{\ensuremath{\begin{vmatrix}#1\end{vmatrix}}}
%\let\vec\mathbf
%\begin{center}
%\title{\textbf{Straight Lines}}
%\date{\vspace{-5ex}} %Not to print date automatically
%\maketitle
%\end{center}
%\setcounter{page}{1}
%\section*{11$^{th}$ Maths - Chapter 10}
%This is Problem-10 from Exercise 10.4
%\begin{enumerate}
%    \item If three lines whose equations are $y=m_1x+c_1$, $y=m_2x+c_2$ and $y=m_3x+c_3$ are concurrent, then show that $m_1(c_2-c_3)+m_2(c_3-c_1)+m_3(c_1-c_2) = 0.$\\
%    \solution 
    Given lines can be written as \begin{align}
       m_1x-y+c_1=0
    \end{align}
    \begin{align}
        m_2x-y+c_2=0
    \end{align}
    \begin{align}
        m_3x-y+c_3=0
        \label{eq:3}
    \end{align}
    
    
   The above lines can be written in the form of \begin{align}
        \Vec{n}^{\top}\Vec{x} = c
    \end{align}
   Therefore,
		\begin{align}
       \myvec{m_1&-1}\vec{x}=c_1
       \label{eq:5}
   \end{align} 
   \begin{align}
       \myvec{m_2&-1}\vec{x}=c_2
       \label{eq:6}
   \end{align}
   \begin{align}
       \myvec{m_3&-1}\vec{x}=c_3
       \label{eq:7}
   \end{align}
   Solving equations \eqref{eq:5}, \eqref{eq:6}and \eqref{eq:7}
		augumented matrix is
 \begin{align}
    \myvec{m_1&-1&c_1\\m_2&-1&c_2\\m_3&-1&c_3}\\
    \xleftrightarrow{R_2 \leftarrow m_1R_2-m_2R_1}
    \myvec{m_1&-1&c_1\\0&m_2-m_1&m_1c_2-m_2c_1\\m_3&-1&c_3}\\
    \xleftrightarrow{R_3 \leftarrow m_1R_3-m_3R_1}
    \myvec{m_1&-1&c_1\\0&m_2-m_1&m_1c_2-m_2c_1\\0&m_3-m_1&m_1c_3-m_3c_1}\\
    \xleftrightarrow{R_3 \leftarrow R_3\frac{m_2-m_1}{m_3-m_1}-R_2}
        \myvec{m_1&-1&c_1\\0&m_2-m_1&m_1c_2-m_2c_1\\0&0&$\brak{m_1c_3-m_3c_1}$$\brak{\frac{m_2-m_1}{m_3-m_1}}$-$\brak{m_1c_2-m_2c_1}$}
\end{align}
Now, for lines to be concurrent, then the third row should be equal to zero. \\

Therefore,
\begin{align}
\brak{m_1c_3-m_3c_1}\brak{\frac{m_2-m_1}{m_3-m_1}}-\brak{m_1c_2-m_2c_1}=0\\
\frac{\brak{m_1c_3-m_3c_1}\brak{m_2-m_1}-\brak{m_1c_2-m_2c_1}\brak{m_3-m_1}}{m_3-m_1}=0\\
\brak{m_1c_3-m_3c_1}\brak{m_2-m_1}-\brak{m_1c_2-m_2c_1}\brak{m_3-m_1}=0\\
m_2c_3-m_1c_3+m_3c_1-m_3c_2+m_1c_2-m_2c_1=0\\
m_1\brak{c_2-c_3}+m_2\brak{c_3-c_1}+m_3\brak{c_1-c_2} = 0
\end{align}
           Hence proved
%\begin{figure}[h]
 %   \centering
  %  \includegraphics[width=\columnwidth]{concurrent-1.png}
   % \caption{Straight Lines}
    %\label{fig:concurrent-1.png}
%\end{figure}
%\end{enumerate}
%\end{document}

\item  If ${p}$ and ${q}$ are the lengths of perpendiculars from the origin to the lines ${x}\cos\theta - {y}\sin\theta =  {k}\cos2\theta$ and ${x}\sec\theta + {y}\cosec\theta = {k}$, respectively, prove that ${p}^2 + 4{q}^2 = {k}^2$
\label{chapters/11/10/3/16}
\\
\solution
%\documentclass[12pt]{article}
%\usepackage[cmex10]{amsmath}
%\usepackage{amsthm}
%\usepackage{mathrsfs}
%\usepackage{txfonts}
%\usepackage{stfloats}
%\usepackage{bm}
%\usepackage{cite}
%\usepackage{cases}
%\usepackage{subfig}
%\usepackage{longtable}
%\usepackage{multirow}
%\usepackage{enumitem}
%\usepackage{mathtools}
%\usepackage{steinmetz}
%\usepackage{tikz}
%\usepackage{circuitikz}
%\usepackage{verbatim}
%\usepackage{tfrupee}
%\usepackage[breaklinks=true]{hyperref}
%\usepackage{tkz-euclide} % loads  TikZ and tkz-base
%\providecommand{\brak}[1]{\ensuremath{\left(#1\right)}}
%\usepackage{atbegshi}
%\AtBeginDocument{\AtBeginShipoutNext{\AtBeginShipoutDiscard}}
%\usetikzlibrary{calc,math}
%\usepackage{listings}
%    \usepackage{color}                                            %%
%    \usepackage{array}                                            %%
 %   \usepackage{longtable}                                        %%
  %  \usepackage{calc}                                             %%
   % \usepackage{multirow}                                         %%
    %\usepackage{hhline}                                           %%
    %\usepackage{ifthen}                                           %%
  %optionally (for landscape tables embedded in another document): %%
    %\usepackage{lscape}     
%\usepackage{multicol}
%\usepackage{chngcntr}

%\DeclareMathOperator*{\Res}{Res}
%\renewcommand{\baselinestretch}{2}
%\renewcommand\thesection{\arabic{section}}
%\renewcommand\thesubsection{\thesection.\arabic{subsection}}
%\renewcommand\thesubsubsection{\thesubsection.\arabic{subsubsection}}


% correct bad hyphenation here
%\hyphenation{op-tical net-works semi-conduc-tor}
%\def\inputGnumericTable{}                                 %%

%\lstset{
%language=C,
%frame=single, 
%breaklines=true,
%columns=fullflexible
%}
%\begin{document}
%\newtheorem{theorem}{Theorem}[section]
%\newtheorem{problem}{Problem}
%\newtheorem{proposition}{Proposition}[section]
%\newtheorem{lemma}{Lemma}[section]
%\newtheorem{corollary}[theorem]{Corollary}
%\newtheorem{example}{Example}[section]
%\newtheorem{definition}[problem]{Definition}
%\newcommand{\BEQA}{\begin{eqnarray}}
%\newcommand{\EEQA}{\end{eqnarray}}
%\newcommand{\define}{\stackrel{\triangle}{=}}

%\bibliographystyle{IEEEtran}
%\bibliographystyle{ieeetr}
%\providecommand{\mbf}{\mathbf}
%\providecommand{\pr}[1]{\ensuremath{\Pr\left(#1\right)}}
%\providecommand{\qfunc}[1]{\ensuremath{Q\left(#1\right)}}
%\providecommand{\sbrak}[1]{\ensuremath{{}\left[#1\right]}}
%\providecommand{\lsbrak}[1]{\ensuremath{{}\left[#1\right.}}
%\providecommand{\rsbrak}[1]{\ensuremath{{}\left.#1\right]}}
%\providecommand{\brak}[1]{\ensuremath{\left(#1\right)}}
%\providecommand{\lbrak}[1]{\ensuremath{\left(#1\right.}}
%\providecommand{\rbrak}[1]{\ensuremath{\left.#1\right)}}
%\providecommand{\cbrak}[1]{\ensuremath{\left\{#1\right\}}}
%\providecommand{\lcbrak}[1]{\ensuremath{\left\{#1\right.}}
%\providecommand{\rcbrak}[1]{\ensuremath{\left.#1\right\}}}
%\theoremstyle{remark}
%\newtheorem{rem}{Remark}
%\newcommand{\sgn}{\mathop{\mathrm{sgn}}}
%\providecommand{\res}[1]{\Res\displaylimits_{#1}} 
%\providecommand{\mtx}[1]{\mathbf{#1}}
%\providecommand{\fourier}{\overset{\mathcal{F}}{\rightleftharpoons}}
%\providecommand{\system}{\overset{\mathcal{H}}{\longleftrightarrow}}
	%\newcommand{\solution}[2]{\textbf{Solution:}{#1}}
%\newcommand{\solution}{\noindent \textbf{Solution: }}
%\newcommand{\cosec}{\,\text{cosec}\,}
%\providecommand{\dec}[2]{\ensuremath{\overset{#1}{\underset{#2}{\gtrless}}}}
%\newcommand{\myvec}[1]{\ensuremath{\begin{pmatrix}#1\end{pmatrix}}}
%\newcommand{\mydet}[1]{\ensuremath{\begin{vmatrix}#1\end{vmatrix}}}
%\let\vec\mathbf
%\begin{center}
%\title{\textbf{Straight Lines}}
%\date{\vspace{-5ex}} %Not to print date automatically
%\maketitle
%\end{center}
%\setcounter{page}{1}
%\section*{11$^{th}$ Maths - Chapter 10}
%This is Problem-10 from Exercise 10.4
%\begin{enumerate}
%    \item If three lines whose equations are $y=m_1x+c_1$, $y=m_2x+c_2$ and $y=m_3x+c_3$ are concurrent, then show that $m_1(c_2-c_3)+m_2(c_3-c_1)+m_3(c_1-c_2) = 0.$\\
%    \solution 
    Given lines can be written as \begin{align}
       m_1x-y+c_1=0
    \end{align}
    \begin{align}
        m_2x-y+c_2=0
    \end{align}
    \begin{align}
        m_3x-y+c_3=0
        \label{eq:3}
    \end{align}
    
    
   The above lines can be written in the form of \begin{align}
        \Vec{n}^{\top}\Vec{x} = c
    \end{align}
   Therefore,
		\begin{align}
       \myvec{m_1&-1}\vec{x}=c_1
       \label{eq:5}
   \end{align} 
   \begin{align}
       \myvec{m_2&-1}\vec{x}=c_2
       \label{eq:6}
   \end{align}
   \begin{align}
       \myvec{m_3&-1}\vec{x}=c_3
       \label{eq:7}
   \end{align}
   Solving equations \eqref{eq:5}, \eqref{eq:6}and \eqref{eq:7}
		augumented matrix is
 \begin{align}
    \myvec{m_1&-1&c_1\\m_2&-1&c_2\\m_3&-1&c_3}\\
    \xleftrightarrow{R_2 \leftarrow m_1R_2-m_2R_1}
    \myvec{m_1&-1&c_1\\0&m_2-m_1&m_1c_2-m_2c_1\\m_3&-1&c_3}\\
    \xleftrightarrow{R_3 \leftarrow m_1R_3-m_3R_1}
    \myvec{m_1&-1&c_1\\0&m_2-m_1&m_1c_2-m_2c_1\\0&m_3-m_1&m_1c_3-m_3c_1}\\
    \xleftrightarrow{R_3 \leftarrow R_3\frac{m_2-m_1}{m_3-m_1}-R_2}
        \myvec{m_1&-1&c_1\\0&m_2-m_1&m_1c_2-m_2c_1\\0&0&$\brak{m_1c_3-m_3c_1}$$\brak{\frac{m_2-m_1}{m_3-m_1}}$-$\brak{m_1c_2-m_2c_1}$}
\end{align}
Now, for lines to be concurrent, then the third row should be equal to zero. \\

Therefore,
\begin{align}
\brak{m_1c_3-m_3c_1}\brak{\frac{m_2-m_1}{m_3-m_1}}-\brak{m_1c_2-m_2c_1}=0\\
\frac{\brak{m_1c_3-m_3c_1}\brak{m_2-m_1}-\brak{m_1c_2-m_2c_1}\brak{m_3-m_1}}{m_3-m_1}=0\\
\brak{m_1c_3-m_3c_1}\brak{m_2-m_1}-\brak{m_1c_2-m_2c_1}\brak{m_3-m_1}=0\\
m_2c_3-m_1c_3+m_3c_1-m_3c_2+m_1c_2-m_2c_1=0\\
m_1\brak{c_2-c_3}+m_2\brak{c_3-c_1}+m_3\brak{c_1-c_2} = 0
\end{align}
           Hence proved
%\begin{figure}[h]
 %   \centering
  %  \includegraphics[width=\columnwidth]{concurrent-1.png}
   % \caption{Straight Lines}
    %\label{fig:concurrent-1.png}
%\end{figure}
%\end{enumerate}
%\end{document}

\item In the triangle $ABC$ with vertices $\vec{A} \brak{2, 3}$, $\vec{B} \brak{4, –1}$ and $\vec{C} \brak{1, 2}$, find the equation and length of altitude from the vertex $\vec{A}$.
\label{chapters/11/10/3/17}
\\
\solution
%\documentclass[12pt]{article}
%\usepackage[cmex10]{amsmath}
%\usepackage{amsthm}
%\usepackage{mathrsfs}
%\usepackage{txfonts}
%\usepackage{stfloats}
%\usepackage{bm}
%\usepackage{cite}
%\usepackage{cases}
%\usepackage{subfig}
%\usepackage{longtable}
%\usepackage{multirow}
%\usepackage{enumitem}
%\usepackage{mathtools}
%\usepackage{steinmetz}
%\usepackage{tikz}
%\usepackage{circuitikz}
%\usepackage{verbatim}
%\usepackage{tfrupee}
%\usepackage[breaklinks=true]{hyperref}
%\usepackage{tkz-euclide} % loads  TikZ and tkz-base
%\providecommand{\brak}[1]{\ensuremath{\left(#1\right)}}
%\usepackage{atbegshi}
%\AtBeginDocument{\AtBeginShipoutNext{\AtBeginShipoutDiscard}}
%\usetikzlibrary{calc,math}
%\usepackage{listings}
%    \usepackage{color}                                            %%
%    \usepackage{array}                                            %%
 %   \usepackage{longtable}                                        %%
  %  \usepackage{calc}                                             %%
   % \usepackage{multirow}                                         %%
    %\usepackage{hhline}                                           %%
    %\usepackage{ifthen}                                           %%
  %optionally (for landscape tables embedded in another document): %%
    %\usepackage{lscape}     
%\usepackage{multicol}
%\usepackage{chngcntr}

%\DeclareMathOperator*{\Res}{Res}
%\renewcommand{\baselinestretch}{2}
%\renewcommand\thesection{\arabic{section}}
%\renewcommand\thesubsection{\thesection.\arabic{subsection}}
%\renewcommand\thesubsubsection{\thesubsection.\arabic{subsubsection}}


% correct bad hyphenation here
%\hyphenation{op-tical net-works semi-conduc-tor}
%\def\inputGnumericTable{}                                 %%

%\lstset{
%language=C,
%frame=single, 
%breaklines=true,
%columns=fullflexible
%}
%\begin{document}
%\newtheorem{theorem}{Theorem}[section]
%\newtheorem{problem}{Problem}
%\newtheorem{proposition}{Proposition}[section]
%\newtheorem{lemma}{Lemma}[section]
%\newtheorem{corollary}[theorem]{Corollary}
%\newtheorem{example}{Example}[section]
%\newtheorem{definition}[problem]{Definition}
%\newcommand{\BEQA}{\begin{eqnarray}}
%\newcommand{\EEQA}{\end{eqnarray}}
%\newcommand{\define}{\stackrel{\triangle}{=}}

%\bibliographystyle{IEEEtran}
%\bibliographystyle{ieeetr}
%\providecommand{\mbf}{\mathbf}
%\providecommand{\pr}[1]{\ensuremath{\Pr\left(#1\right)}}
%\providecommand{\qfunc}[1]{\ensuremath{Q\left(#1\right)}}
%\providecommand{\sbrak}[1]{\ensuremath{{}\left[#1\right]}}
%\providecommand{\lsbrak}[1]{\ensuremath{{}\left[#1\right.}}
%\providecommand{\rsbrak}[1]{\ensuremath{{}\left.#1\right]}}
%\providecommand{\brak}[1]{\ensuremath{\left(#1\right)}}
%\providecommand{\lbrak}[1]{\ensuremath{\left(#1\right.}}
%\providecommand{\rbrak}[1]{\ensuremath{\left.#1\right)}}
%\providecommand{\cbrak}[1]{\ensuremath{\left\{#1\right\}}}
%\providecommand{\lcbrak}[1]{\ensuremath{\left\{#1\right.}}
%\providecommand{\rcbrak}[1]{\ensuremath{\left.#1\right\}}}
%\theoremstyle{remark}
%\newtheorem{rem}{Remark}
%\newcommand{\sgn}{\mathop{\mathrm{sgn}}}
%\providecommand{\res}[1]{\Res\displaylimits_{#1}} 
%\providecommand{\mtx}[1]{\mathbf{#1}}
%\providecommand{\fourier}{\overset{\mathcal{F}}{\rightleftharpoons}}
%\providecommand{\system}{\overset{\mathcal{H}}{\longleftrightarrow}}
	%\newcommand{\solution}[2]{\textbf{Solution:}{#1}}
%\newcommand{\solution}{\noindent \textbf{Solution: }}
%\newcommand{\cosec}{\,\text{cosec}\,}
%\providecommand{\dec}[2]{\ensuremath{\overset{#1}{\underset{#2}{\gtrless}}}}
%\newcommand{\myvec}[1]{\ensuremath{\begin{pmatrix}#1\end{pmatrix}}}
%\newcommand{\mydet}[1]{\ensuremath{\begin{vmatrix}#1\end{vmatrix}}}
%\let\vec\mathbf
%\begin{center}
%\title{\textbf{Straight Lines}}
%\date{\vspace{-5ex}} %Not to print date automatically
%\maketitle
%\end{center}
%\setcounter{page}{1}
%\section*{11$^{th}$ Maths - Chapter 10}
%This is Problem-10 from Exercise 10.4
%\begin{enumerate}
%    \item If three lines whose equations are $y=m_1x+c_1$, $y=m_2x+c_2$ and $y=m_3x+c_3$ are concurrent, then show that $m_1(c_2-c_3)+m_2(c_3-c_1)+m_3(c_1-c_2) = 0.$\\
%    \solution 
    Given lines can be written as \begin{align}
       m_1x-y+c_1=0
    \end{align}
    \begin{align}
        m_2x-y+c_2=0
    \end{align}
    \begin{align}
        m_3x-y+c_3=0
        \label{eq:3}
    \end{align}
    
    
   The above lines can be written in the form of \begin{align}
        \Vec{n}^{\top}\Vec{x} = c
    \end{align}
   Therefore,
		\begin{align}
       \myvec{m_1&-1}\vec{x}=c_1
       \label{eq:5}
   \end{align} 
   \begin{align}
       \myvec{m_2&-1}\vec{x}=c_2
       \label{eq:6}
   \end{align}
   \begin{align}
       \myvec{m_3&-1}\vec{x}=c_3
       \label{eq:7}
   \end{align}
   Solving equations \eqref{eq:5}, \eqref{eq:6}and \eqref{eq:7}
		augumented matrix is
 \begin{align}
    \myvec{m_1&-1&c_1\\m_2&-1&c_2\\m_3&-1&c_3}\\
    \xleftrightarrow{R_2 \leftarrow m_1R_2-m_2R_1}
    \myvec{m_1&-1&c_1\\0&m_2-m_1&m_1c_2-m_2c_1\\m_3&-1&c_3}\\
    \xleftrightarrow{R_3 \leftarrow m_1R_3-m_3R_1}
    \myvec{m_1&-1&c_1\\0&m_2-m_1&m_1c_2-m_2c_1\\0&m_3-m_1&m_1c_3-m_3c_1}\\
    \xleftrightarrow{R_3 \leftarrow R_3\frac{m_2-m_1}{m_3-m_1}-R_2}
        \myvec{m_1&-1&c_1\\0&m_2-m_1&m_1c_2-m_2c_1\\0&0&$\brak{m_1c_3-m_3c_1}$$\brak{\frac{m_2-m_1}{m_3-m_1}}$-$\brak{m_1c_2-m_2c_1}$}
\end{align}
Now, for lines to be concurrent, then the third row should be equal to zero. \\

Therefore,
\begin{align}
\brak{m_1c_3-m_3c_1}\brak{\frac{m_2-m_1}{m_3-m_1}}-\brak{m_1c_2-m_2c_1}=0\\
\frac{\brak{m_1c_3-m_3c_1}\brak{m_2-m_1}-\brak{m_1c_2-m_2c_1}\brak{m_3-m_1}}{m_3-m_1}=0\\
\brak{m_1c_3-m_3c_1}\brak{m_2-m_1}-\brak{m_1c_2-m_2c_1}\brak{m_3-m_1}=0\\
m_2c_3-m_1c_3+m_3c_1-m_3c_2+m_1c_2-m_2c_1=0\\
m_1\brak{c_2-c_3}+m_2\brak{c_3-c_1}+m_3\brak{c_1-c_2} = 0
\end{align}
           Hence proved
%\begin{figure}[h]
 %   \centering
  %  \includegraphics[width=\columnwidth]{concurrent-1.png}
   % \caption{Straight Lines}
    %\label{fig:concurrent-1.png}
%\end{figure}
%\end{enumerate}
%\end{document}

\item If $p$ is the length of perpendicular from origin to the line whose intercepts on the axes are $a$ and $b$, then show that 
\begin{align}
	\frac{1}{p^2} = \frac{1}{a^2}+ \frac{1}{b^2}
\end{align}
\label{chapters/11/10/3/18}
\input{chapters/11/10/3/18/dist.tex}
\item What are the points on the y-axis whose distance from the line $\frac{x}{3}+\frac{y}{4}=1$ is 4 units.
\\
\solution
		%\documentclass[12pt]{article}
%\usepackage[cmex10]{amsmath}
%\usepackage{amsthm}
%\usepackage{mathrsfs}
%\usepackage{txfonts}
%\usepackage{stfloats}
%\usepackage{bm}
%\usepackage{cite}
%\usepackage{cases}
%\usepackage{subfig}
%\usepackage{longtable}
%\usepackage{multirow}
%\usepackage{enumitem}
%\usepackage{mathtools}
%\usepackage{steinmetz}
%\usepackage{tikz}
%\usepackage{circuitikz}
%\usepackage{verbatim}
%\usepackage{tfrupee}
%\usepackage[breaklinks=true]{hyperref}
%\usepackage{tkz-euclide} % loads  TikZ and tkz-base
%\providecommand{\brak}[1]{\ensuremath{\left(#1\right)}}
%\usepackage{atbegshi}
%\AtBeginDocument{\AtBeginShipoutNext{\AtBeginShipoutDiscard}}
%\usetikzlibrary{calc,math}
%\usepackage{listings}
%    \usepackage{color}                                            %%
%    \usepackage{array}                                            %%
 %   \usepackage{longtable}                                        %%
  %  \usepackage{calc}                                             %%
   % \usepackage{multirow}                                         %%
    %\usepackage{hhline}                                           %%
    %\usepackage{ifthen}                                           %%
  %optionally (for landscape tables embedded in another document): %%
    %\usepackage{lscape}     
%\usepackage{multicol}
%\usepackage{chngcntr}

%\DeclareMathOperator*{\Res}{Res}
%\renewcommand{\baselinestretch}{2}
%\renewcommand\thesection{\arabic{section}}
%\renewcommand\thesubsection{\thesection.\arabic{subsection}}
%\renewcommand\thesubsubsection{\thesubsection.\arabic{subsubsection}}


% correct bad hyphenation here
%\hyphenation{op-tical net-works semi-conduc-tor}
%\def\inputGnumericTable{}                                 %%

%\lstset{
%language=C,
%frame=single, 
%breaklines=true,
%columns=fullflexible
%}
%\begin{document}
%\newtheorem{theorem}{Theorem}[section]
%\newtheorem{problem}{Problem}
%\newtheorem{proposition}{Proposition}[section]
%\newtheorem{lemma}{Lemma}[section]
%\newtheorem{corollary}[theorem]{Corollary}
%\newtheorem{example}{Example}[section]
%\newtheorem{definition}[problem]{Definition}
%\newcommand{\BEQA}{\begin{eqnarray}}
%\newcommand{\EEQA}{\end{eqnarray}}
%\newcommand{\define}{\stackrel{\triangle}{=}}

%\bibliographystyle{IEEEtran}
%\bibliographystyle{ieeetr}
%\providecommand{\mbf}{\mathbf}
%\providecommand{\pr}[1]{\ensuremath{\Pr\left(#1\right)}}
%\providecommand{\qfunc}[1]{\ensuremath{Q\left(#1\right)}}
%\providecommand{\sbrak}[1]{\ensuremath{{}\left[#1\right]}}
%\providecommand{\lsbrak}[1]{\ensuremath{{}\left[#1\right.}}
%\providecommand{\rsbrak}[1]{\ensuremath{{}\left.#1\right]}}
%\providecommand{\brak}[1]{\ensuremath{\left(#1\right)}}
%\providecommand{\lbrak}[1]{\ensuremath{\left(#1\right.}}
%\providecommand{\rbrak}[1]{\ensuremath{\left.#1\right)}}
%\providecommand{\cbrak}[1]{\ensuremath{\left\{#1\right\}}}
%\providecommand{\lcbrak}[1]{\ensuremath{\left\{#1\right.}}
%\providecommand{\rcbrak}[1]{\ensuremath{\left.#1\right\}}}
%\theoremstyle{remark}
%\newtheorem{rem}{Remark}
%\newcommand{\sgn}{\mathop{\mathrm{sgn}}}
%\providecommand{\res}[1]{\Res\displaylimits_{#1}} 
%\providecommand{\mtx}[1]{\mathbf{#1}}
%\providecommand{\fourier}{\overset{\mathcal{F}}{\rightleftharpoons}}
%\providecommand{\system}{\overset{\mathcal{H}}{\longleftrightarrow}}
	%\newcommand{\solution}[2]{\textbf{Solution:}{#1}}
%\newcommand{\solution}{\noindent \textbf{Solution: }}
%\newcommand{\cosec}{\,\text{cosec}\,}
%\providecommand{\dec}[2]{\ensuremath{\overset{#1}{\underset{#2}{\gtrless}}}}
%\newcommand{\myvec}[1]{\ensuremath{\begin{pmatrix}#1\end{pmatrix}}}
%\newcommand{\mydet}[1]{\ensuremath{\begin{vmatrix}#1\end{vmatrix}}}
%\let\vec\mathbf
%\begin{center}
%\title{\textbf{Straight Lines}}
%\date{\vspace{-5ex}} %Not to print date automatically
%\maketitle
%\end{center}
%\setcounter{page}{1}
%\section*{11$^{th}$ Maths - Chapter 10}
%This is Problem-10 from Exercise 10.4
%\begin{enumerate}
%    \item If three lines whose equations are $y=m_1x+c_1$, $y=m_2x+c_2$ and $y=m_3x+c_3$ are concurrent, then show that $m_1(c_2-c_3)+m_2(c_3-c_1)+m_3(c_1-c_2) = 0.$\\
%    \solution 
    Given lines can be written as \begin{align}
       m_1x-y+c_1=0
    \end{align}
    \begin{align}
        m_2x-y+c_2=0
    \end{align}
    \begin{align}
        m_3x-y+c_3=0
        \label{eq:3}
    \end{align}
    
    
   The above lines can be written in the form of \begin{align}
        \Vec{n}^{\top}\Vec{x} = c
    \end{align}
   Therefore,
		\begin{align}
       \myvec{m_1&-1}\vec{x}=c_1
       \label{eq:5}
   \end{align} 
   \begin{align}
       \myvec{m_2&-1}\vec{x}=c_2
       \label{eq:6}
   \end{align}
   \begin{align}
       \myvec{m_3&-1}\vec{x}=c_3
       \label{eq:7}
   \end{align}
   Solving equations \eqref{eq:5}, \eqref{eq:6}and \eqref{eq:7}
		augumented matrix is
 \begin{align}
    \myvec{m_1&-1&c_1\\m_2&-1&c_2\\m_3&-1&c_3}\\
    \xleftrightarrow{R_2 \leftarrow m_1R_2-m_2R_1}
    \myvec{m_1&-1&c_1\\0&m_2-m_1&m_1c_2-m_2c_1\\m_3&-1&c_3}\\
    \xleftrightarrow{R_3 \leftarrow m_1R_3-m_3R_1}
    \myvec{m_1&-1&c_1\\0&m_2-m_1&m_1c_2-m_2c_1\\0&m_3-m_1&m_1c_3-m_3c_1}\\
    \xleftrightarrow{R_3 \leftarrow R_3\frac{m_2-m_1}{m_3-m_1}-R_2}
        \myvec{m_1&-1&c_1\\0&m_2-m_1&m_1c_2-m_2c_1\\0&0&$\brak{m_1c_3-m_3c_1}$$\brak{\frac{m_2-m_1}{m_3-m_1}}$-$\brak{m_1c_2-m_2c_1}$}
\end{align}
Now, for lines to be concurrent, then the third row should be equal to zero. \\

Therefore,
\begin{align}
\brak{m_1c_3-m_3c_1}\brak{\frac{m_2-m_1}{m_3-m_1}}-\brak{m_1c_2-m_2c_1}=0\\
\frac{\brak{m_1c_3-m_3c_1}\brak{m_2-m_1}-\brak{m_1c_2-m_2c_1}\brak{m_3-m_1}}{m_3-m_1}=0\\
\brak{m_1c_3-m_3c_1}\brak{m_2-m_1}-\brak{m_1c_2-m_2c_1}\brak{m_3-m_1}=0\\
m_2c_3-m_1c_3+m_3c_1-m_3c_2+m_1c_2-m_2c_1=0\\
m_1\brak{c_2-c_3}+m_2\brak{c_3-c_1}+m_3\brak{c_1-c_2} = 0
\end{align}
           Hence proved
%\begin{figure}[h]
 %   \centering
  %  \includegraphics[width=\columnwidth]{concurrent-1.png}
   % \caption{Straight Lines}
    %\label{fig:concurrent-1.png}
%\end{figure}
%\end{enumerate}
%\end{document}

\item Find perpendicular distance from the origin to the line joining the points$(\cos\theta,\sin\theta)$ and $(\cos\phi,\sin\phi)$.
\\
\solution
		\input{chapters/11/10/4/5/dist.tex}
\item Find the equation of line which is equidistant from parallel lines $9x+6y-7=0$ and $3x+2y+6=0$.
\\
\solution
		%\documentclass[12pt]{article}
%\usepackage[cmex10]{amsmath}
%\usepackage{amsthm}
%\usepackage{mathrsfs}
%\usepackage{txfonts}
%\usepackage{stfloats}
%\usepackage{bm}
%\usepackage{cite}
%\usepackage{cases}
%\usepackage{subfig}
%\usepackage{longtable}
%\usepackage{multirow}
%\usepackage{enumitem}
%\usepackage{mathtools}
%\usepackage{steinmetz}
%\usepackage{tikz}
%\usepackage{circuitikz}
%\usepackage{verbatim}
%\usepackage{tfrupee}
%\usepackage[breaklinks=true]{hyperref}
%\usepackage{tkz-euclide} % loads  TikZ and tkz-base
%\providecommand{\brak}[1]{\ensuremath{\left(#1\right)}}
%\usepackage{atbegshi}
%\AtBeginDocument{\AtBeginShipoutNext{\AtBeginShipoutDiscard}}
%\usetikzlibrary{calc,math}
%\usepackage{listings}
%    \usepackage{color}                                            %%
%    \usepackage{array}                                            %%
 %   \usepackage{longtable}                                        %%
  %  \usepackage{calc}                                             %%
   % \usepackage{multirow}                                         %%
    %\usepackage{hhline}                                           %%
    %\usepackage{ifthen}                                           %%
  %optionally (for landscape tables embedded in another document): %%
    %\usepackage{lscape}     
%\usepackage{multicol}
%\usepackage{chngcntr}

%\DeclareMathOperator*{\Res}{Res}
%\renewcommand{\baselinestretch}{2}
%\renewcommand\thesection{\arabic{section}}
%\renewcommand\thesubsection{\thesection.\arabic{subsection}}
%\renewcommand\thesubsubsection{\thesubsection.\arabic{subsubsection}}


% correct bad hyphenation here
%\hyphenation{op-tical net-works semi-conduc-tor}
%\def\inputGnumericTable{}                                 %%

%\lstset{
%language=C,
%frame=single, 
%breaklines=true,
%columns=fullflexible
%}
%\begin{document}
%\newtheorem{theorem}{Theorem}[section]
%\newtheorem{problem}{Problem}
%\newtheorem{proposition}{Proposition}[section]
%\newtheorem{lemma}{Lemma}[section]
%\newtheorem{corollary}[theorem]{Corollary}
%\newtheorem{example}{Example}[section]
%\newtheorem{definition}[problem]{Definition}
%\newcommand{\BEQA}{\begin{eqnarray}}
%\newcommand{\EEQA}{\end{eqnarray}}
%\newcommand{\define}{\stackrel{\triangle}{=}}

%\bibliographystyle{IEEEtran}
%\bibliographystyle{ieeetr}
%\providecommand{\mbf}{\mathbf}
%\providecommand{\pr}[1]{\ensuremath{\Pr\left(#1\right)}}
%\providecommand{\qfunc}[1]{\ensuremath{Q\left(#1\right)}}
%\providecommand{\sbrak}[1]{\ensuremath{{}\left[#1\right]}}
%\providecommand{\lsbrak}[1]{\ensuremath{{}\left[#1\right.}}
%\providecommand{\rsbrak}[1]{\ensuremath{{}\left.#1\right]}}
%\providecommand{\brak}[1]{\ensuremath{\left(#1\right)}}
%\providecommand{\lbrak}[1]{\ensuremath{\left(#1\right.}}
%\providecommand{\rbrak}[1]{\ensuremath{\left.#1\right)}}
%\providecommand{\cbrak}[1]{\ensuremath{\left\{#1\right\}}}
%\providecommand{\lcbrak}[1]{\ensuremath{\left\{#1\right.}}
%\providecommand{\rcbrak}[1]{\ensuremath{\left.#1\right\}}}
%\theoremstyle{remark}
%\newtheorem{rem}{Remark}
%\newcommand{\sgn}{\mathop{\mathrm{sgn}}}
%\providecommand{\res}[1]{\Res\displaylimits_{#1}} 
%\providecommand{\mtx}[1]{\mathbf{#1}}
%\providecommand{\fourier}{\overset{\mathcal{F}}{\rightleftharpoons}}
%\providecommand{\system}{\overset{\mathcal{H}}{\longleftrightarrow}}
	%\newcommand{\solution}[2]{\textbf{Solution:}{#1}}
%\newcommand{\solution}{\noindent \textbf{Solution: }}
%\newcommand{\cosec}{\,\text{cosec}\,}
%\providecommand{\dec}[2]{\ensuremath{\overset{#1}{\underset{#2}{\gtrless}}}}
%\newcommand{\myvec}[1]{\ensuremath{\begin{pmatrix}#1\end{pmatrix}}}
%\newcommand{\mydet}[1]{\ensuremath{\begin{vmatrix}#1\end{vmatrix}}}
%\let\vec\mathbf
%\begin{center}
%\title{\textbf{Straight Lines}}
%\date{\vspace{-5ex}} %Not to print date automatically
%\maketitle
%\end{center}
%\setcounter{page}{1}
%\section*{11$^{th}$ Maths - Chapter 10}
%This is Problem-10 from Exercise 10.4
%\begin{enumerate}
%    \item If three lines whose equations are $y=m_1x+c_1$, $y=m_2x+c_2$ and $y=m_3x+c_3$ are concurrent, then show that $m_1(c_2-c_3)+m_2(c_3-c_1)+m_3(c_1-c_2) = 0.$\\
%    \solution 
    Given lines can be written as \begin{align}
       m_1x-y+c_1=0
    \end{align}
    \begin{align}
        m_2x-y+c_2=0
    \end{align}
    \begin{align}
        m_3x-y+c_3=0
        \label{eq:3}
    \end{align}
    
    
   The above lines can be written in the form of \begin{align}
        \Vec{n}^{\top}\Vec{x} = c
    \end{align}
   Therefore,
		\begin{align}
       \myvec{m_1&-1}\vec{x}=c_1
       \label{eq:5}
   \end{align} 
   \begin{align}
       \myvec{m_2&-1}\vec{x}=c_2
       \label{eq:6}
   \end{align}
   \begin{align}
       \myvec{m_3&-1}\vec{x}=c_3
       \label{eq:7}
   \end{align}
   Solving equations \eqref{eq:5}, \eqref{eq:6}and \eqref{eq:7}
		augumented matrix is
 \begin{align}
    \myvec{m_1&-1&c_1\\m_2&-1&c_2\\m_3&-1&c_3}\\
    \xleftrightarrow{R_2 \leftarrow m_1R_2-m_2R_1}
    \myvec{m_1&-1&c_1\\0&m_2-m_1&m_1c_2-m_2c_1\\m_3&-1&c_3}\\
    \xleftrightarrow{R_3 \leftarrow m_1R_3-m_3R_1}
    \myvec{m_1&-1&c_1\\0&m_2-m_1&m_1c_2-m_2c_1\\0&m_3-m_1&m_1c_3-m_3c_1}\\
    \xleftrightarrow{R_3 \leftarrow R_3\frac{m_2-m_1}{m_3-m_1}-R_2}
        \myvec{m_1&-1&c_1\\0&m_2-m_1&m_1c_2-m_2c_1\\0&0&$\brak{m_1c_3-m_3c_1}$$\brak{\frac{m_2-m_1}{m_3-m_1}}$-$\brak{m_1c_2-m_2c_1}$}
\end{align}
Now, for lines to be concurrent, then the third row should be equal to zero. \\

Therefore,
\begin{align}
\brak{m_1c_3-m_3c_1}\brak{\frac{m_2-m_1}{m_3-m_1}}-\brak{m_1c_2-m_2c_1}=0\\
\frac{\brak{m_1c_3-m_3c_1}\brak{m_2-m_1}-\brak{m_1c_2-m_2c_1}\brak{m_3-m_1}}{m_3-m_1}=0\\
\brak{m_1c_3-m_3c_1}\brak{m_2-m_1}-\brak{m_1c_2-m_2c_1}\brak{m_3-m_1}=0\\
m_2c_3-m_1c_3+m_3c_1-m_3c_2+m_1c_2-m_2c_1=0\\
m_1\brak{c_2-c_3}+m_2\brak{c_3-c_1}+m_3\brak{c_1-c_2} = 0
\end{align}
           Hence proved
%\begin{figure}[h]
 %   \centering
  %  \includegraphics[width=\columnwidth]{concurrent-1.png}
   % \caption{Straight Lines}
    %\label{fig:concurrent-1.png}
%\end{figure}
%\end{enumerate}
%\end{document}

	\item Prove that the products of the lengths of the perpendiculars drawn from the points $\myvec{\sqrt{a^2-b^2}\\0}$ and $\myvec{-\sqrt{a^2-b^2} \\0} $ to the line $\frac{x}{a} \cos{\theta} + \frac{y}{b}\sin{\theta} =1 $ is $ b^2 $.
\\
    \solution 
		\input{chapters/11/10/4/23/dist.tex}
\item Find the equation of line  drawn perpendicular to the line $\frac{x}{4}+\frac{y}{6}=1$ through the point where it meets the y-axis \\
\solution
		%\documentclass[12pt]{article}
%\usepackage[cmex10]{amsmath}
%\usepackage{amsthm}
%\usepackage{mathrsfs}
%\usepackage{txfonts}
%\usepackage{stfloats}
%\usepackage{bm}
%\usepackage{cite}
%\usepackage{cases}
%\usepackage{subfig}
%\usepackage{longtable}
%\usepackage{multirow}
%\usepackage{enumitem}
%\usepackage{mathtools}
%\usepackage{steinmetz}
%\usepackage{tikz}
%\usepackage{circuitikz}
%\usepackage{verbatim}
%\usepackage{tfrupee}
%\usepackage[breaklinks=true]{hyperref}
%\usepackage{tkz-euclide} % loads  TikZ and tkz-base
%\providecommand{\brak}[1]{\ensuremath{\left(#1\right)}}
%\usepackage{atbegshi}
%\AtBeginDocument{\AtBeginShipoutNext{\AtBeginShipoutDiscard}}
%\usetikzlibrary{calc,math}
%\usepackage{listings}
%    \usepackage{color}                                            %%
%    \usepackage{array}                                            %%
 %   \usepackage{longtable}                                        %%
  %  \usepackage{calc}                                             %%
   % \usepackage{multirow}                                         %%
    %\usepackage{hhline}                                           %%
    %\usepackage{ifthen}                                           %%
  %optionally (for landscape tables embedded in another document): %%
    %\usepackage{lscape}     
%\usepackage{multicol}
%\usepackage{chngcntr}

%\DeclareMathOperator*{\Res}{Res}
%\renewcommand{\baselinestretch}{2}
%\renewcommand\thesection{\arabic{section}}
%\renewcommand\thesubsection{\thesection.\arabic{subsection}}
%\renewcommand\thesubsubsection{\thesubsection.\arabic{subsubsection}}


% correct bad hyphenation here
%\hyphenation{op-tical net-works semi-conduc-tor}
%\def\inputGnumericTable{}                                 %%

%\lstset{
%language=C,
%frame=single, 
%breaklines=true,
%columns=fullflexible
%}
%\begin{document}
%\newtheorem{theorem}{Theorem}[section]
%\newtheorem{problem}{Problem}
%\newtheorem{proposition}{Proposition}[section]
%\newtheorem{lemma}{Lemma}[section]
%\newtheorem{corollary}[theorem]{Corollary}
%\newtheorem{example}{Example}[section]
%\newtheorem{definition}[problem]{Definition}
%\newcommand{\BEQA}{\begin{eqnarray}}
%\newcommand{\EEQA}{\end{eqnarray}}
%\newcommand{\define}{\stackrel{\triangle}{=}}

%\bibliographystyle{IEEEtran}
%\bibliographystyle{ieeetr}
%\providecommand{\mbf}{\mathbf}
%\providecommand{\pr}[1]{\ensuremath{\Pr\left(#1\right)}}
%\providecommand{\qfunc}[1]{\ensuremath{Q\left(#1\right)}}
%\providecommand{\sbrak}[1]{\ensuremath{{}\left[#1\right]}}
%\providecommand{\lsbrak}[1]{\ensuremath{{}\left[#1\right.}}
%\providecommand{\rsbrak}[1]{\ensuremath{{}\left.#1\right]}}
%\providecommand{\brak}[1]{\ensuremath{\left(#1\right)}}
%\providecommand{\lbrak}[1]{\ensuremath{\left(#1\right.}}
%\providecommand{\rbrak}[1]{\ensuremath{\left.#1\right)}}
%\providecommand{\cbrak}[1]{\ensuremath{\left\{#1\right\}}}
%\providecommand{\lcbrak}[1]{\ensuremath{\left\{#1\right.}}
%\providecommand{\rcbrak}[1]{\ensuremath{\left.#1\right\}}}
%\theoremstyle{remark}
%\newtheorem{rem}{Remark}
%\newcommand{\sgn}{\mathop{\mathrm{sgn}}}
%\providecommand{\res}[1]{\Res\displaylimits_{#1}} 
%\providecommand{\mtx}[1]{\mathbf{#1}}
%\providecommand{\fourier}{\overset{\mathcal{F}}{\rightleftharpoons}}
%\providecommand{\system}{\overset{\mathcal{H}}{\longleftrightarrow}}
	%\newcommand{\solution}[2]{\textbf{Solution:}{#1}}
%\newcommand{\solution}{\noindent \textbf{Solution: }}
%\newcommand{\cosec}{\,\text{cosec}\,}
%\providecommand{\dec}[2]{\ensuremath{\overset{#1}{\underset{#2}{\gtrless}}}}
%\newcommand{\myvec}[1]{\ensuremath{\begin{pmatrix}#1\end{pmatrix}}}
%\newcommand{\mydet}[1]{\ensuremath{\begin{vmatrix}#1\end{vmatrix}}}
%\let\vec\mathbf
%\begin{center}
%\title{\textbf{Straight Lines}}
%\date{\vspace{-5ex}} %Not to print date automatically
%\maketitle
%\end{center}
%\setcounter{page}{1}
%\section*{11$^{th}$ Maths - Chapter 10}
%This is Problem-10 from Exercise 10.4
%\begin{enumerate}
%    \item If three lines whose equations are $y=m_1x+c_1$, $y=m_2x+c_2$ and $y=m_3x+c_3$ are concurrent, then show that $m_1(c_2-c_3)+m_2(c_3-c_1)+m_3(c_1-c_2) = 0.$\\
%    \solution 
    Given lines can be written as \begin{align}
       m_1x-y+c_1=0
    \end{align}
    \begin{align}
        m_2x-y+c_2=0
    \end{align}
    \begin{align}
        m_3x-y+c_3=0
        \label{eq:3}
    \end{align}
    
    
   The above lines can be written in the form of \begin{align}
        \Vec{n}^{\top}\Vec{x} = c
    \end{align}
   Therefore,
		\begin{align}
       \myvec{m_1&-1}\vec{x}=c_1
       \label{eq:5}
   \end{align} 
   \begin{align}
       \myvec{m_2&-1}\vec{x}=c_2
       \label{eq:6}
   \end{align}
   \begin{align}
       \myvec{m_3&-1}\vec{x}=c_3
       \label{eq:7}
   \end{align}
   Solving equations \eqref{eq:5}, \eqref{eq:6}and \eqref{eq:7}
		augumented matrix is
 \begin{align}
    \myvec{m_1&-1&c_1\\m_2&-1&c_2\\m_3&-1&c_3}\\
    \xleftrightarrow{R_2 \leftarrow m_1R_2-m_2R_1}
    \myvec{m_1&-1&c_1\\0&m_2-m_1&m_1c_2-m_2c_1\\m_3&-1&c_3}\\
    \xleftrightarrow{R_3 \leftarrow m_1R_3-m_3R_1}
    \myvec{m_1&-1&c_1\\0&m_2-m_1&m_1c_2-m_2c_1\\0&m_3-m_1&m_1c_3-m_3c_1}\\
    \xleftrightarrow{R_3 \leftarrow R_3\frac{m_2-m_1}{m_3-m_1}-R_2}
        \myvec{m_1&-1&c_1\\0&m_2-m_1&m_1c_2-m_2c_1\\0&0&$\brak{m_1c_3-m_3c_1}$$\brak{\frac{m_2-m_1}{m_3-m_1}}$-$\brak{m_1c_2-m_2c_1}$}
\end{align}
Now, for lines to be concurrent, then the third row should be equal to zero. \\

Therefore,
\begin{align}
\brak{m_1c_3-m_3c_1}\brak{\frac{m_2-m_1}{m_3-m_1}}-\brak{m_1c_2-m_2c_1}=0\\
\frac{\brak{m_1c_3-m_3c_1}\brak{m_2-m_1}-\brak{m_1c_2-m_2c_1}\brak{m_3-m_1}}{m_3-m_1}=0\\
\brak{m_1c_3-m_3c_1}\brak{m_2-m_1}-\brak{m_1c_2-m_2c_1}\brak{m_3-m_1}=0\\
m_2c_3-m_1c_3+m_3c_1-m_3c_2+m_1c_2-m_2c_1=0\\
m_1\brak{c_2-c_3}+m_2\brak{c_3-c_1}+m_3\brak{c_1-c_2} = 0
\end{align}
           Hence proved
%\begin{figure}[h]
 %   \centering
  %  \includegraphics[width=\columnwidth]{concurrent-1.png}
   % \caption{Straight Lines}
    %\label{fig:concurrent-1.png}
%\end{figure}
%\end{enumerate}
%\end{document}

 \item  In each of the following cases, determine the direction cosines of the normal to
the plane and the distance from the origin.
\begin{enumerate}
	\item $z=2$ 
	\item $x + y + z = 1$
	\item $2x + 3y – z = 5$
	\item $5y + 8 = 0$
\end{enumerate}
    \solution
		\input{chapters/12/11/3/1/dist.tex}
\item
Find the angle between the lines whose direction ratios are $a,b,c$ and $b-c,c-a,a-b$.

\textbf{Solution :}
    \begin{align}
    \vec{m _1} &= \myvec{a\\b\\c}\\
    \vec{m_2} &= \myvec{b-c\\c-a\\a-b}\\
    \cos{\theta}&= \frac{\vec{m_1}^{\top}\vec{m_2}}{\vec{\norm{m_1}\norm{m_2}}
   } \\
   &=\frac{\myvec{a&b&c}\myvec{b-c\\c-a\\a-b}}{\sqrt{a^2+b^2+c^2}\sqrt{\brak{b-c}^2+\brak{c-a}^2+\brak{a-b}^2}}\\
   &=0\\
   or,\theta&=\frac{\pi}{2}
    \end{align}

\end{enumerate}
