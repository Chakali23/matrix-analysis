%\documentclass[12pt]{article}
%\usepackage[cmex10]{amsmath}
%\usepackage{amsthm}
%\usepackage{mathrsfs}
%\usepackage{txfonts}
%\usepackage{stfloats}
%\usepackage{bm}
%\usepackage{cite}
%\usepackage{cases}
%\usepackage{subfig}
%\usepackage{longtable}
%\usepackage{multirow}
%\usepackage{enumitem}
%\usepackage{mathtools}
%\usepackage{steinmetz}
%\usepackage{tikz}
%\usepackage{circuitikz}
%\usepackage{verbatim}
%\usepackage{tfrupee}
%\usepackage[breaklinks=true]{hyperref}
%\usepackage{tkz-euclide} % loads  TikZ and tkz-base
%\providecommand{\brak}[1]{\ensuremath{\left(#1\right)}}
%\usepackage{atbegshi}
%\AtBeginDocument{\AtBeginShipoutNext{\AtBeginShipoutDiscard}}
%\usetikzlibrary{calc,math}
%\usepackage{listings}
%    \usepackage{color}                                            %%
%    \usepackage{array}                                            %%
 %   \usepackage{longtable}                                        %%
  %  \usepackage{calc}                                             %%
   % \usepackage{multirow}                                         %%
    %\usepackage{hhline}                                           %%
    %\usepackage{ifthen}                                           %%
  %optionally (for landscape tables embedded in another document): %%
    %\usepackage{lscape}     
%\usepackage{multicol}
%\usepackage{chngcntr}

%\DeclareMathOperator*{\Res}{Res}
%\renewcommand{\baselinestretch}{2}
%\renewcommand\thesection{\arabic{section}}
%\renewcommand\thesubsection{\thesection.\arabic{subsection}}
%\renewcommand\thesubsubsection{\thesubsection.\arabic{subsubsection}}


% correct bad hyphenation here
%\hyphenation{op-tical net-works semi-conduc-tor}
%\def\inputGnumericTable{}                                 %%

%\lstset{
%language=C,
%frame=single, 
%breaklines=true,
%columns=fullflexible
%}
%\begin{document}
%\newtheorem{theorem}{Theorem}[section]
%\newtheorem{problem}{Problem}
%\newtheorem{proposition}{Proposition}[section]
%\newtheorem{lemma}{Lemma}[section]
%\newtheorem{corollary}[theorem]{Corollary}
%\newtheorem{example}{Example}[section]
%\newtheorem{definition}[problem]{Definition}
%\newcommand{\BEQA}{\begin{eqnarray}}
%\newcommand{\EEQA}{\end{eqnarray}}
%\newcommand{\define}{\stackrel{\triangle}{=}}

%\bibliographystyle{IEEEtran}
%\bibliographystyle{ieeetr}
%\providecommand{\mbf}{\mathbf}
%\providecommand{\pr}[1]{\ensuremath{\Pr\left(#1\right)}}
%\providecommand{\qfunc}[1]{\ensuremath{Q\left(#1\right)}}
%\providecommand{\sbrak}[1]{\ensuremath{{}\left[#1\right]}}
%\providecommand{\lsbrak}[1]{\ensuremath{{}\left[#1\right.}}
%\providecommand{\rsbrak}[1]{\ensuremath{{}\left.#1\right]}}
%\providecommand{\brak}[1]{\ensuremath{\left(#1\right)}}
%\providecommand{\lbrak}[1]{\ensuremath{\left(#1\right.}}
%\providecommand{\rbrak}[1]{\ensuremath{\left.#1\right)}}
%\providecommand{\cbrak}[1]{\ensuremath{\left\{#1\right\}}}
%\providecommand{\lcbrak}[1]{\ensuremath{\left\{#1\right.}}
%\providecommand{\rcbrak}[1]{\ensuremath{\left.#1\right\}}}
%\theoremstyle{remark}
%\newtheorem{rem}{Remark}
%\newcommand{\sgn}{\mathop{\mathrm{sgn}}}
%\providecommand{\res}[1]{\Res\displaylimits_{#1}} 
%\providecommand{\mtx}[1]{\mathbf{#1}}
%\providecommand{\fourier}{\overset{\mathcal{F}}{\rightleftharpoons}}
%\providecommand{\system}{\overset{\mathcal{H}}{\longleftrightarrow}}
	%\newcommand{\solution}[2]{\textbf{Solution:}{#1}}
%\newcommand{\solution}{\noindent \textbf{Solution: }}
%\newcommand{\cosec}{\,\text{cosec}\,}
%\providecommand{\dec}[2]{\ensuremath{\overset{#1}{\underset{#2}{\gtrless}}}}
%\newcommand{\myvec}[1]{\ensuremath{\begin{pmatrix}#1\end{pmatrix}}}
%\newcommand{\mydet}[1]{\ensuremath{\begin{vmatrix}#1\end{vmatrix}}}
%\let\vec\mathbf
%\begin{center}
%\title{\textbf{Straight Lines}}
%\date{\vspace{-5ex}} %Not to print date automatically
%\maketitle
%\end{center}
%\setcounter{page}{1}
%\section*{11$^{th}$ Maths - Chapter 10}
%This is Problem-10 from Exercise 10.4
%\begin{enumerate}
%    \item If three lines whose equations are $y=m_1x+c_1$, $y=m_2x+c_2$ and $y=m_3x+c_3$ are concurrent, then show that $m_1(c_2-c_3)+m_2(c_3-c_1)+m_3(c_1-c_2) = 0.$\\
%    \solution 
    Given lines can be written as \begin{align}
       m_1x-y+c_1=0
    \end{align}
    \begin{align}
        m_2x-y+c_2=0
    \end{align}
    \begin{align}
        m_3x-y+c_3=0
        \label{eq:line3}
    \end{align}
    
    
   The above lines can be written in the form of \begin{align}
        \Vec{n}^{\top}\Vec{x} = c
    \end{align}
   Therefore,
		\begin{align}
       \myvec{m_1&-1}\vec{x}=c_1
       \label{eq:line5}
   \end{align} 
   \begin{align}
       \myvec{m_2&-1}\vec{x}=c_2
       \label{eq:line6}
   \end{align}
   \begin{align}
       \myvec{m_3&-1}\vec{x}=c_3
       \label{eq:line7}
   \end{align}
   Solving equations \eqref{eq:line5}, \eqref{eq:line6}and \eqref{eq:line7}
		augumented matrix is
 \begin{align}
    \myvec{m_1&-1&c_1\\m_2&-1&c_2\\m_3&-1&c_3}\\
    \xleftrightarrow{R_2 \leftarrow m_1R_2-m_2R_1}
    \myvec{m_1&-1&c_1\\0&m_2-m_1&m_1c_2-m_2c_1\\m_3&-1&c_3}\\
    \xleftrightarrow{R_3 \leftarrow m_1R_3-m_3R_1}
    \myvec{m_1&-1&c_1\\0&m_2-m_1&m_1c_2-m_2c_1\\0&m_3-m_1&m_1c_3-m_3c_1}\\
    \xleftrightarrow{R_3 \leftarrow R_3\frac{m_2-m_1}{m_3-m_1}-R_2}
        \myvec{m_1&-1&c_1\\0&m_2-m_1&m_1c_2-m_2c_1\\0&0&$\brak{m_1c_3-m_3c_1}$$\brak{\frac{m_2-m_1}{m_3-m_1}}$-$\brak{m_1c_2-m_2c_1}$}
\end{align}
Now, for lines to be concurrent, then the third row should be equal to zero. \\

Therefore,
\begin{align}
\brak{m_1c_3-m_3c_1}\brak{\frac{m_2-m_1}{m_3-m_1}}-\brak{m_1c_2-m_2c_1}=0\\
\frac{\brak{m_1c_3-m_3c_1}\brak{m_2-m_1}-\brak{m_1c_2-m_2c_1}\brak{m_3-m_1}}{m_3-m_1}=0\\
\brak{m_1c_3-m_3c_1}\brak{m_2-m_1}-\brak{m_1c_2-m_2c_1}\brak{m_3-m_1}=0\\
m_2c_3-m_1c_3+m_3c_1-m_3c_2+m_1c_2-m_2c_1=0\\
m_1\brak{c_2-c_3}+m_2\brak{c_3-c_1}+m_3\brak{c_1-c_2} = 0
\end{align}
           Hence proved
%\begin{figure}[h]
 %   \centering
  %  \includegraphics[width=\columnwidth]{concurrent-1.png}
   % \caption{Straight Lines}
    %\label{fig:concurrent-1.png}
%\end{figure}
%\end{enumerate}
%\end{document}
